\documentclass[12pt,a4paper]{book}

% ====== Paketi ======
\usepackage[utf8]{inputenc}
\usepackage[T1]{fontenc}
\usepackage[croatian]{babel}
\usepackage{lmodern}
\usepackage{graphicx}
\usepackage{amsmath, amssymb}
\usepackage{hyperref}
\usepackage{listings}
\usepackage{color}
\usepackage{geometry}
\usepackage{setspace}
\usepackage{caption}

% ====== Postavke ======
\geometry{margin=2.5cm}
\onehalfspacing
\hypersetup{
	colorlinks=true,
	linkcolor=blue,
	urlcolor=blue,
	citecolor=blue,
	pdftitle={Ekološko modeliranje i predviđanje},
	pdfauthor={Branimir Hackenberger}
}

% ====== Okvir za kod ======
\definecolor{lightgray}{gray}{0.95}
\lstset{
	basicstyle=\ttfamily\small,
	backgroundcolor=\color{lightgray},
	frame=single,
	breaklines=true,
	captionpos=b,
	numbers=left,
	numberstyle=\tiny,
	keywordstyle=\color{blue},
	commentstyle=\color{green!50!black}
}

% ====== Naslovnica ======
\title{\Huge Ekološko modeliranje i predviđanje \\[1cm]
	\Large Sveučilišni udžbenik}
\author{Branimir Hackenberger}
\date{2025}

\begin{document}
	\maketitle
	\tableofcontents
	\chapter*{Predgovor}
	Razlozi za pisanje udžbenika, ciljana publika, način korištenja.
	
	% ====== Glavni dijelovi ======
	
	\part{Osnove}
	\chapter{Uvod u ekološko modeliranje}
	\section{Što je model?}
	\section{Povijesni razvoj modeliranja}
	\section{Vrste modela u ekologiji}
	\section{Prednosti i ograničenja modela}
	
	\chapter{Matematičke i računalne osnove}
	\section{Osnovne funkcije i jednadžbe}
	\section{Diferencijalne i diskretne jednadžbe}
	\section{Matrični modeli (Leslie, Lefkovitch)}
	\section{Stohastički procesi}
	\section{Računalni alati (R, Python, Matlab, NetLogo)}
	
	\part{Populacije i zajednice}
	\chapter{Populacijski modeli}
	\section{Eksponencijalni i logistički rast}
	\section{Lotka-Volterra modeli}
	\section{Dobno-strukturni modeli}
	\section{Metapopulacijski modeli}
	
	\chapter{Modeli zajednica}
	\section{Ekološke mreže}
	\section{Stabilnost zajednica}
	\section{Sukcesija i bioraznolikost}
	
	\part{Ekosustavi i prostor}
	\chapter{Modeli ekosustava}
	\section{Kruženje tvari i energije}
	\section{Bioenergetski modeli}
	\section{DEB teorija}
	\section{Biogeokemijski ciklusi}
	\section{Ecosystem services modeli}
	
	\chapter{Prostorno-ekološki modeli}
	\section{GIS i prostorna analiza}
	\section{Modeli rasprostiranja vrsta (SDM, ENM)}
	\section{Fragmentacija staništa}
	\section{Individualno temeljeni modeli (IBM)}
	
	\part{Predviđanje i primjena}
	\chapter{Prediktivni modeli i scenariji}
	\section{Scenariji klimatskih promjena}
	\section{Predviđanje invazija}
	\section{Rani sustavi upozorenja}
	\section{Kombinacija mehanističkih i statističkih modela}
	
	\chapter{Statistički i AI pristupi}
	\section{Frekventistički i Bayesovski modeli}
	\section{Strojno učenje (RF, XGBoost, SVM)}
	\section{Duboko učenje i neuronske mreže}
	\section{Digitalni blizanci ekosustava}
	
	\chapter{Validacija i nesigurnost}
	\section{Kalibracija i verifikacija}
	\section{Osjetljivost i robusnost modela}
	\section{Nesigurnost i etika predviđanja}
	
	\chapter{Primjene u upravljanju okolišem}
	\section{Zaštita prirode i bioraznolikosti}
	\section{Poljoprivreda i šumarstvo}
	\section{Ekotoksikologija i procjena rizika}
	\section{Klimatska politika i održivi razvoj}
	
	\part{Studije slučaja i praksa}
	\chapter{Studije slučaja}
	\section{Modeliranje populacije gujavica u agroekosustavu}
	\section{Predviđanje širenja komaraca}
	\section{Eutrofikacija riječnog ekosustava}
	\section{Fragmentacija šuma i ptice metapopulacija}
	\section{Bayesove mreže u ekotoksikologiji}
	
	\chapter{Praktične vježbe u R-u i Pythonu}
	\section{Uvod u R i Python}
	\section{Leslie matrica u R-u}
	\section{Lotka-Volterra simulacija u Pythonu}
	\section{SDM u R-u (paket \texttt{dismo})}
	\section{Random Forest za invazije}
	\section{DEB simulacije}
	
	\appendix
	\chapter{Dodatci}
	\section{Matematički prilozi}
	\section{Upute za instalaciju softvera}
	\section{Primjeri koda}
	\section{Literatura i mrežni izvori}
	
\end{document}
