% !TEX TS-program = lualatex
\documentclass[12pt,oneside]{book}

% ====== JEZIK I KODNA STRANA ======
\usepackage{fontspec} % za LuaLaTeX/XeLaTeX; uklonite i dodajte inputenc+fontenc ako koristite pdfLaTeX
\defaultfontfeatures{Ligatures=TeX}
\setmainfont{Latin Modern Roman}
\usepackage[croatian]{babel}

% ====== GEOMETRIJA I OSNOVNI PAKETI ======
\usepackage[a4paper,margin=2.8cm]{geometry}
\usepackage{graphicx}
\usepackage{microtype}
\usepackage{csquotes}
\usepackage{booktabs}
\usepackage{longtable}
\usepackage{array}
\usepackage{amsmath,amssymb}
\usepackage{siunitx}
\usepackage{enumitem}
\setlist{itemsep=2pt,topsep=2pt}

% ====== HIPERLINKOVI I NASLOV ======
\usepackage[hidelinks]{hyperref}
\usepackage[capitalise,nameinlink]{cleveref}

\usepackage[most]{tcolorbox}

% ====== KÔD I LISTINZI ======
\usepackage{listings}
\lstset{
	basicstyle=\ttfamily\small,
	columns=fullflexible,
	frame=single,
	breaklines=true,
	showstringspaces=false
}

% ====== BIBLIOGRAFIJA (Biber) ======
\usepackage[
backend=biber,
style=authoryear,
sorting=nyt,
giveninits=true,
maxbibnames=6
]{biblatex}
\addbibresource{literatura.bib} % napravite datoteku literatura.bib

% ====== META PODACI ======
\title{Ekološko modeliranje i predviđanje\\[4pt]\large Sveobuhvatni koncept sveučilišnog udžbenika}
\author{Branimir Hackenberger}
\date{2025}

% ====== POMOĆNI MAKROI ======
\newcommand{\R}{\textsf{R}}
\newcommand{\Python}{\textsf{Python}}

\begin{document}
	\frontmatter
	\maketitle
	\tableofcontents
	
	\chapter*{Predgovor}
	Ovaj udžbenik sintetizira teorijske temelje, metode i praktične primjene ekološkog
	modeliranja s posebnim naglaskom na validaciju, nesigurnost i prediktivne scenarije.
	Struktura je objedinjena iz dvije početne skice i prilagođena kao koherentan kurikulum
	za preddiplomske i diplomske studije ekologije, biologije i znanosti o okolišu, te kao
	referentni priručnik za istraživače i praktičare.\footnote{Temeljni raspored poglavlja,
		definicije i okviri preuzeti su i integrirani iz \textit{Koncept.pdf} i \textit{Koncept2.pdf}.}
	
	\mainmatter
	
	% =====================================================
	\part{Osnove i matematički okvir}
	% =====================================================
	
	\chapter{Uvod u ekološko modeliranje}
	\label{ch:uvod}
	
	\section{Što je ekološko modeliranje?}
	\label{sec:sto-je-modeliranje}
	
	Ekološko modeliranje je sustavno korištenje matematičkih, statističkih i računalnih alata za \emph{formaliziranje} naše spoznaje o organizmima, populacijama, zajednicama i ekosustavima te za \emph{predviđanje} njihovog ponašanja u različitim uvjetima. Model je pojednostavljenje stvarnosti: on izdvaja bitne komponente i procese kako bi omogućio razumijevanje (\emph{zašto} se nešto događa), kvantifikaciju (\emph{koliko} i \emph{koliko brzo}) i prosudbu (\emph{što ako} promijenimo uvjete).
	
	% --- CALL OUT BOX ---
	\begin{tcolorbox}[colback=blue!5,colframe=blue!50!black,title={Savjet}]
		Dobar model ne mora biti složen: često je jednostavniji model koji hvata ključne procese
		vrijedniji od kompliciranog modela s teško dostupnim parametrima.
	\end{tcolorbox}
	
	% --- EXERCISE BOX ---
	\begin{tcolorbox}[colback=green!5,colframe=green!40!black,title={Zadatak 1}]
		Napišite vlastitim riječima kako biste modelirali populaciju žaba u baru: 
		koje varijable stanja biste uveli, koji parametri bi bili važni, i kako biste opisali dinamiku?
	\end{tcolorbox}
	
	\section{Definicije, opseg i uloga}
	\label{sec:definicije-ulogea}
	
	Ekološko modeliranje obuhvaća četiri komplementarne uloge:
	
	\begin{itemize}
		\item \textbf{Deskriptivna} --- opis obrazaca (npr. trend populacije).
		\item \textbf{Eksplanatorna} --- uzročno-posljedične hipoteze.
		\item \textbf{Prediktivna} --- predviđanja izvan opaženog raspona.
		\item \textbf{Preskriptivna} --- preporuke za upravljanje i politiku.
	\end{itemize}
	
	\begin{tcolorbox}[colback=yellow!10,colframe=orange!80!black,title={Savjet}]
		Prije nego što izradite model, pitajte se: 
		\emph{Je li svrha mog modela opis, objašnjenje, predviđanje ili preporuka?} 
		Odgovor određuje tip modela i potrebnu razinu složenosti.
	\end{tcolorbox}
	
	\begin{tcolorbox}[colback=green!5,colframe=green!40!black,title={Zadatak 2}]
		Za sljedeće primjere odredite kojoj kategoriji modeliranja pripadaju:  
		\begin{enumerate}
			\item Projekcija širenja invazivne vrste pod klimatskim promjenama.  
			\item Analiza učinka temperature na brzinu metabolizma riba.  
			\item Karta raspodjele šumskih tipova prema satelitskim snimkama.  
			\item Model koji preporučuje kvote ribolova radi održivosti.  
		\end{enumerate}
	\end{tcolorbox}
	
	\section{Povijesni pregled}
	\label{sec:povijesni-pregled}
	
	Razvoj ekološkog modeliranja može se podijeliti u nekoliko ključnih etapa:
	\begin{itemize}
		\item \textbf{18.–19. st.}: Malthus (eksponencijalni rast), Verhulst (logistički rast).
		\item \textbf{20. st. rana polovica}: Lotka–Volterra (grabljivac–plijen), Gause (kompetitivna isključenost), Holling (funkcijski odgovori).
		\item \textbf{Sredina 20. st.}: Leslie (dobno strukturirane matrice), Lefkovitch (stadiji).
		\item \textbf{1960.–1980.}: Levins (metapopulacije), May (složenost i stabilnost, kaos).
		\item \textbf{1990.–danas}: SDM/ENM, Bayes, strojno učenje, DEB, digitalni blizanci.
	\end{itemize}
	
	\begin{tcolorbox}[colback=blue!5,colframe=blue!70!black,title={Savjet}]
		Povijesni pregled nije samo lista imena --- on pokazuje kako se pitanja ekologije
		razvijaju u skladu s dostupnim matematičkim alatima i tehnologijom.
	\end{tcolorbox}
	
	\begin{tcolorbox}[colback=green!5,colframe=green!40!black,title={Zadatak 3}]
		Istražite:  
		\begin{enumerate}
			\item Koji je bio glavni nedostatak Malthusovog modela u odnosu na stvarne populacije?  
			\item Kako je Verhulstov logistički model popravio taj nedostatak?  
			\item U kojim uvjetima Lotka–Volterra model gubi stabilnost i prelazi u oscilacije?  
		\end{enumerate}
	\end{tcolorbox}
	
	
	\chapter{Matematičke i statističke osnove}
	\section{Linearna algebra i matrični modeli}
	Leslie i Lefkovitch matrice; dominantna vlastita vrijednost $\lambda_1$ određuje asimptotsku stopu rasta.
	\section{Diferencijalne i diskretne jednadžbe}
	Eksponencijalni i logistički rast:
	\begin{equation}
		\frac{dN}{dt} = rN\Bigl(1-\frac{N}{K}\Bigr), \qquad
		N(t)=\frac{K}{1+\Bigl(\frac{K-N_0}{N_0}\Bigr)e^{-rt}}.
	\end{equation}
	\section{Vjerojatnost, statistika i stohastičnost}
	Demografska, okolišna i katastrofična stohastičnost; maksimumna vjerojatnost, Bayesov pristup.
	\section{Računalni alati}
	\R, \Python, Matlab, NetLogo; reproducibilnost i upravljanje paketima.
	
	% =====================================================
	\part{Populacije, zajednice i ekosustavi}
	% =====================================================
	
	\chapter{Modeli populacijske dinamike}
	\section{Jednostavni modeli}
	Eksponencijalni i logistički rast; diskretni modeli (Ricker, Beverton--Holt).
	\section{Dobno/stanovno strukturirani modeli}
	Leslie/Lefkovitch; tranzicijske matrice; elastičnosti i osjetljivosti.
	\section{Metapopulacijski modeli}
	Levinsov okvir, kolonizacija i izumiranje, fragmentacija.
	
	\chapter{Međuspecijske interakcije i modeli zajednica}
	\section{Grabljivac--plijen i funkcijski odgovori}
	Lotka--Volterra; Holling tip I--III.
	\section{Konkurencija i koegzistencija}
	Uvjete koegzistencije i stabilnost; niše i preklapanje resursa.
	\section{Ekološke mreže i stabilnost zajednica}
	Topologija mreža, otpornost, kaskadni učinci.
	
	\chapter{Modeli ekosustava i bioenergetika}
	\section{Kruženje tvari i energije}
	Bilance, tokovi i pretvorbe; biogeokemijski ciklusi.
	\section{Bioenergetski i DEB modeli}
	DEB teorija za protok energije i tvari kroz organizme; povezivanje s razinama bioraznolikosti.
	\section{Ecosystem services}
	Kvantifikacija usluga ekosustava i trade-off analize.
	
	% =====================================================
	\part{Prostor, predviđanje i AI}
	% =====================================================
	
	\chapter{Prostorno-ekološko modeliranje}
	\section{GIS, daljinska istraživanja i skale}
	Lokalna \textless{}1 km, krajobrazna 1--100 km, regionalna 100--1000 km, kontinentalna \textgreater{}1000 km.
	\section{Modeli rasprostiranja vrsta (SDM/ENM)}
	Klimatske varijable, bioklimatski slojevi, pristranost uzorkovanja.
	\section{IBM modeli}
	Individualno temeljeni pristupi i emergentna svojstva.
	
	\chapter{Prediktivno modeliranje i scenariji}
	\section{Scenariji klimatskih promjena}
	Korištenje scenarija (npr. SSP/RCP) u ekološkim projekcijama.
	\section{Predviđanje invazija i rani sustavi upozorenja}
	Integracija podataka nadzora i modela rizika; pragovi upozorenja.
	\section{Spajanje mehanističkih i statističkih pristupa}
	Hibridni modeli i kalibracija.
	
	\chapter{Statistički i AI pristupi}
	\section{Klasična i Bayesovska statistika}
	GLM/GLMM, MCMC, kredibilni intervali.
	\section{Strojno učenje}
	RF, XGBoost, SVM; značajke, regularizacija, unakrsna provjera.
	\section{Duboko učenje i digitalni blizanci ekosustava}
	Grafovi, konvolucijske i sekvencijske mreže; digital twins za \emph{what-if} scenarije.
	
	% =====================================================
	\part{Validacija, nesigurnost i primjene}
	% =====================================================
	
	\chapter{Validacija, verifikacija i osjetljivost}
	\section{Kalibracija i verifikacija}
	Podjela podataka, \emph{out-of-sample} procjene, nezavisni skupovi.
	\section{Analize osjetljivosti i robustnosti}
	Lokalne i globalne metode; Monte Carlo; propagacija nesigurnosti.
	\section{Etika i komunikacija nesigurnosti}
	Transparentnost i odgovornost modelara.
	
	\chapter{Primjene u upravljanju okolišem}
	\section{Zaštita prirode i bioraznolikosti}
	Prioritizacija očuvanja, planovi upravljanja.
	\section{Poljoprivreda i šumarstvo}
	Produktivnost, štetnici, otpornost agroekosustava.
	\section{Ekotoksikologija i procjena rizika}
	LC/EC metri\linebreak ke, PNEC, scenariji izloženosti; integracija s populacijskim modelima.
	\section{Klimatske politike i održivi razvoj}
	Podrška odlučivanju, socio-ekonomske poveznice.
	
	% =====================================================
	\part{Studije slučaja i praktične vježbe}
	% =====================================================
	
	\chapter{Studije slučaja}
	\section{Gujavice u agroekosustavima}
	Dobno strukturirani modeli, elastičnost i scenariji upravljanja.
	\section{Širenje komaraca}
	SDM + meteorološki pogonjeni prediktori; rani sustavi upozorenja.
	\section{Eutrofikacija riječnog ekosustava}
	Kutije (box) modeli i validacija na nizvodnim mjerenjima.
	\section{Fragmentacija šuma i ptice metapopulacija}
	Povezanost staništa i pragovi propusnosti krajolika.
	\section{Bayesove mreže u ekotoksikologiji}
	Kauzalni grafikoni i inverzna inferencija.
	
	\chapter{Praktične vježbe (\R{} i \Python{})}
	\section{Uvod u \R{} i \Python{} alate}
	Instalacija, radna okolina, reproducibilnost.
	\section{Leslie matrica u \R{}}
	\begin{lstlisting}[language=R,caption={Leslie matrica i projekcija populacije u R-u}]
		F <- c(0, 0.3, 1.2, 1.5)
		P <- c(0.6, 0.7, 0.8)
		L <- matrix(c(F,
		P[1],0,0,0,
		0,P[2],0,0,
		0,0,P[3],0), nrow=4, byrow=TRUE)
		n0 <- c(50, 40, 20, 10)
		proj <- function(L, n, t=20){
			N <- matrix(NA, nrow=length(n), ncol=t+1)
			N[,1] <- n
			for(i in 1:t) N[,i+1] <- L %*% N[,i]
			N
		}
		N <- proj(L, n0, t=30)
		colSums(N) -> Tot
		plot(Tot, type="l", xlab="Vrijeme", ylab="Uk. veličina")
	\end{lstlisting}
	
	\section{Lotka--Volterra simulacija u \Python{}}
	\begin{lstlisting}[language=Python,caption={Jednostavni LV model u Pythonu (scipy.integrate)}]
		import numpy as np
		from scipy.integrate import solve_ivp
		import matplotlib.pyplot as plt
		
		def lv(t, z, r, a, e, m):
		N, P = z
		dN = r*N - a*N*P
		dP = e*a*N*P - m*P
		return [dN, dP]
		
		pars = dict(r=0.8, a=0.02, e=0.1, m=0.3)
		sol = solve_ivp(lambda t,z: lv(t,z,**pars),
		[0, 200], [40, 9], dense_output=True)
		t = np.linspace(0,200,1000)
		N, P = sol.sol(t)
		plt.plot(t, N, label="Plijen")
		plt.plot(t, P, label="Grabežljivac")
		plt.xlabel("Vrijeme"); plt.ylabel("Gustoća"); plt.legend(); plt.show()
	\end{lstlisting}
	
	\section{SDM u \R{} (skica s \texttt{dismo})}
	\begin{lstlisting}[language=R,caption={SDM skica s bioklimatskim varijablama}]
		library(dismo); library(raster)
		# bioclim <- getData('worldclim', var='bio', res=10) # primjer dohvaćanja
		# occ <- read.csv("occ_points.csv")  # popisi opažanja (lon, lat)
		# m <- maxent(bioclim, occ)
		# p <- predict(bioclim, m)
		# plot(p)
	\end{lstlisting}
	
	\section{Random Forest za invazije}
	\begin{lstlisting}[language=R,caption={RF klasifikator s unakrsnom provjerom}]
		library(tidymodels)
		set.seed(1)
		# df: response ~ predictors
		split <- initial_split(df, prop=0.8, strata=response)
		train <- training(split); test <- testing(split)
		rf_spec <- rand_forest(trees=500) %>% set_engine("ranger") %>% set_mode("classification")
		rec <- recipe(response ~ ., data=train) %>% step_zv(all_predictors())
		wf <- workflow() %>% add_model(rf_spec) %>% add_recipe(rec)
		res <- wf %>% fit_resamples(vfold_cv(train, v=5, strata=response), metrics=metric_set(roc_auc,accuracy))
		collect_metrics(res)
	\end{lstlisting}
	
	\section{DEB simulacije (skica)}
	\begin{lstlisting}[language=R,caption={Minimalna DEB skica (konceptualno)}]
		# Ovo je conceptual stub: definirajte parametre i tokove E, V...
		pars <- list(p_Am=1, v=0.02, kappa=0.8)
		state <- c(E=1, V=0.1)
		deb_model <- function(t, y, p){
			with(as.list(c(y,p)), {
				dE <- p_Am - v*E
				dV <- kappa*v*E - 0.01*V
				list(c(dE,dV))
			})
		}
		# solve s deSolve::ode(...)
	\end{lstlisting}
	
	% =====================================================
	\appendix
	% =====================================================
	
	\chapter{Matematički prilozi}
	\section{Osnovni populacijski modeli}
	Sažeti popis formula (eksponencijalni, logistički, Ricker, Beverton--Holt).
	
	\chapter{Instalacija softvera}
	Upute za \R/\Python{} okruženja, pakete i reproducibilnost.
	
	\chapter{Primjeri koda}
	Proširene skripte za poglavlja iz \emph{Praktičnih vježbi}.
	
	\chapter{Rječnik pojmova}
	Bioraznolikost, nosivost staništa, metapopulacija, ekološka niša, stohastičnost.
	
	\backmatter
	\printbibliography
	
\end{document}
