\documentclass[12pt, a4paper]{book}

% --- PAKETI ---
\usepackage[utf8]{inputenc}
\usepackage[T1]{fontenc}
\usepackage[croatian]{babel}
\usepackage{amsmath}          % Za napredne matematičke formule
\usepackage{amssymb}          % Za dodatne matematičke simbole
\usepackage{graphicx}         % Za umetanje slika
\usepackage{geometry}         % Za podešavanje margina
\usepackage{hyperref}         % Za kreiranje linkova unutar dokumenta
\usepackage{tocbibind}        % Za dodavanje sadržaja, literature itd. u sadržaj

% --- PODEŠAVANJA DOKUMENTA ---
\geometry{a4paper, top=2.5cm, bottom=2.5cm, left=3cm, right=2.5cm}
\hypersetup{
	colorlinks=true,
	linkcolor=blue,
	filecolor=magenta,      
	urlcolor=cyan,
	pdftitle={Ekološko modeliranje i predviđanje},
	pdfpagemode=FullScreen,
}

% --- NASLOVNA STRANICA ---
\title{Ekološko modeliranje i predviđanje}
\author{Ime Prezime Autora}
\date{\today}

% --- POČETAK DOKUMENTA ---
\begin{document}
	
	\maketitle
	\frontmatter % Dio prije glavnog sadržaja (rimski brojevi stranica)
	
	\tableofcontents
	
	\mainmatter % Glavni sadržaj (arapski brojevi stranica)
	
	% --- POGLAVLJE 1: UVOD ---
	\chapter{Uvod u ekološko modeliranje}
	
	\section{Što je ekološki model?}
	Ekološki model je apstraktna, obično matematička, reprezentacija ekološkog sustava. Cilj modela je pojednostaviti složene interakcije u prirodi kako bi se lakše razumjele, analizirale i predvidjele. Modeli mogu varirati od jednostavnih jednadžbi koje opisuju rast populacije do složenih simulacija koje obuhvaćaju cijele ekosustave.
	
	\section{Povijest ekološkog modeliranja}
	Povijest ekološkog modeliranja seže u rano 20. stoljeće s radovima Lotke i Volterre na modelima predator-plijen. Razvoj računalne tehnologije u drugoj polovici 20. stoljeća omogućio je razvoj znatno složenijih simulacijskih modela.
	
	\section{Svrha i primjena modela}
	Modeli se u ekologiji koriste za:
	\begin{itemize}
		\item Razumijevanje temeljnih ekoloških procesa.
		\item Predviđanje budućih stanja ekosustava (npr. utjecaj klimatskih promjena).
		\item Upravljanje resursima (npr. održivo ribarstvo).
		\item Testiranje hipoteza koje je teško ili nemoguće provjeriti u stvarnom svijetu.
	\end{itemize}
	
	\section{Vrste modela}
	Ekološke modele možemo podijeliti u nekoliko kategorija:
	\begin{itemize}
		\item \textbf{Analitički modeli:} Sastoje se od jednadžbi čija se rješenja mogu pronaći u zatvorenom obliku.
		\item \textbf{Simulacijski (numerički) modeli:} Rješavaju se pomoću računala korak po korak.
		\item \textbf{Deterministički vs. stohastički modeli:} Deterministički modeli daju isti izlaz za iste ulazne parametre, dok stohastički uključuju element slučajnosti.
		\item \textbf{Prostorno eksplicitni vs. implicitni modeli:} Eksplicitni modeli uzimaju u obzir točnu lokaciju jedinki ili resursa.
	\end{itemize}
	
	% --- POGLAVLJE 2: MATEMATIČKE OSNOVE ---
	\chapter{Matematičke osnove za modeliranje}
	
	\section{Diferencijalne jednadžbe}
	Mnogi ekološki procesi, poput rasta populacije, opisuju se kao promjene tijekom vremena. Diferencijalne jednadžbe su ključan alat za to. Primjerice, eksponencijalni rast populacije ($N$) opisuje se jednadžbom:
	$$ \frac{dN}{dt} = rN $$
	gdje je $r$ stopa rasta.
	
	\section{Matrična algebra}
	Matrična algebra je neophodna za modele koji prate populacije strukturirane po dobi ili stadijima (npr. Leslie matrice). Matrica prijelaza opisuje vjerojatnosti preživljavanja i reprodukcije različitih dobnih skupina.
	
	\section{Osnove vjerojatnosti i statistike}
	Statistički alati koriste se za procjenu parametara modela iz stvarnih podataka i za analizu nesigurnosti. Stohastički modeli izravno ugrađuju vjerojatnosne procese.
	
	% --- POGLAVLJE 3: MODELI POPULACIJSKE DINAMIKE ---
	\chapter{Modeli populacijske dinamike}
	
	\section{Eksponencijalni rast}
	Model neograničenog rasta, primjenjiv u idealnim uvjetima bez ograničenja resursa.
	$$ N(t) = N_0 e^{rt} $$
	gdje je $N_0$ početna veličina populacije.
	
	\section{Logistički rast}
	Realističniji model koji uključuje kapacitet okoliša ($K$). Rast se usporava kako se populacija približava kapacitetu.
	$$ \frac{dN}{dt} = rN \left(1 - \frac{N}{K}\right) $$
	
	\section{Modeli Lotka-Volterra (predator-plijen)}
	Klasičan model koji opisuje cikličke oscilacije u populacijama predatora ($P$) i plijena ($V$):
	\begin{align*}
		\frac{dV}{dt} &= rV - \alpha VP \\
		\frac{dP}{dt} &= \beta VP - qP
	\end{align*}
	
	\section{Strukturirani modeli populacija}
	Ovi modeli dijele populaciju u klase (npr. dob, veličina) i prate dinamiku svake klase zasebno, što omogućuje detaljniji uvid u populacijsku dinamiku.
	
	% --- POGLAVLJE 4: MODELI INTERAKCIJA ---
	\chapter{Modeli interakcija među vrstama}
	\section{Kompeticija}
	Modeli kompeticije, poput proširenog Lotka-Volterra modela, analiziraju kako dvije ili više vrsta koje koriste iste ograničene resurse utječu jedna na drugu.
	
	\section{Mutualizam i parazitizam}
	Matematički modeli mogu opisati i pozitivne interakcije (mutualizam) ili interakcije gdje jedna vrsta ima koristi na štetu druge (parazitizam).
	
	% --- POGLAVLJE 5: MODELI EKOSUSTAVA ---
	\chapter{Modeli ekosustava}
	\section{Protok energije i kruženje tvari}
	Ovi modeli prate kretanje energije i nutrijenata (npr. ugljika, dušika) kroz različite trofičke razine i komponente ekosustava.
	
	\section{Modeli hranidbenih mreža}
	Analiziraju strukturu i stabilnost složenih mreža interakcija predatora i plijena unutar zajednice.
	
	\section{Prostorni modeli}
	Uključuju prostornu dimenziju, modelirajući kako se populacije i resursi raspoređuju i kreću u prostoru (npr. metapopulacijski modeli, modeli krajolika).
	
	% --- POGLAVLJE 6: VALIDACIJA I ANALIZA MODELA ---
	\chapter{Validacija i analiza modela}
	\section{Kalibracija modela}
	Proces podešavanja parametara modela kako bi izlaz modela što bolje odgovarao opaženim, stvarnim podacima.
	
	\section{Validacija i verifikacija}
	Validacija je provjera podudara li se model sa stvarnim svijetom, dok je verifikacija provjera radi li model kako je zamišljeno (npr. je li kod ispravan).
	
	\section{Analiza osjetljivosti i nesigurnosti}
	Analiza osjetljivosti ispituje kako promjene u ulaznim parametrima utječu na izlaz modela. Analiza nesigurnosti kvantificira pouzdanost predviđanja modela.
	
	% --- POGLAVLJE 7: PREDVIĐANJE U EKOLOGIJI ---
	\chapter{Predviđanje u ekologiji}
	\section{Izazovi ekološkog predviđanja}
	Ekološki sustavi su složeni, nelinearni i podložni slučajnim događajima, što predviđanje čini iznimno teškim.
	
	\section{Modeliranje scenarija}
	Koristi se za istraživanje mogućih budućih ishoda pod različitim pretpostavkama (npr. različiti scenariji klimatskih promjena, politike upravljanja).
	
	\section{Primjeri: klimatske promjene i biodiverzitet}
	Modeli se koriste za predviđanje kako će klimatske promjene utjecati na rasprostranjenost vrsta, stabilnost ekosustava i bioraznolikost.
	
	% --- DODATAK I LITERATURA ---
	\appendix
	\chapter{Osnove softvera za modeliranje}
	Kratak pregled popularnih alata za ekološko modeliranje:
	\begin{itemize}
		\item \textbf{R:} Snažan statistički jezik s brojnim paketima za ekološko modeliranje.
		\item \textbf{Python:} Svestran programski jezik s bibliotekama kao što su NumPy i SciPy.
		\item \textbf{NetLogo:} Platforma za modeliranje temeljeno na agentima.
	\end{itemize}
	
	\backmatter
	\begin{thebibliography}{9}
		\bibitem{lotka}
		Lotka, A. J. (1925). \textit{Elements of physical biology}. Williams \& Wilkins Company.
		
		\bibitem{volterra}
		Volterra, V. (1926). Variazioni e fluttuazioni del numero d'individui in specie animali conviventi. \textit{Mem. R. Accad. Naz. dei Lincei}, 2, 31-113.
		
		\bibitem{gotelli}
		Gotelli, N. J. (2008). \textit{A primer of ecology}. Sinauer Associates.
		
	\end{thebibliography}
	
\end{document}
