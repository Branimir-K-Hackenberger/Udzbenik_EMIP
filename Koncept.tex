\documentclass[12pt,a4paper,twoside]{book}

% Paketi za hrvatski jezik
\usepackage[utf8]{inputenc}
\usepackage[croatian]{babel}
\usepackage[T1]{fontenc}

% Ostali potrebni paketi
\usepackage{amsmath,amsfonts,amssymb}
\usepackage{graphicx}
\usepackage{fancyhdr}
\usepackage{geometry}
\usepackage{setspace}
\usepackage{titlesec}
\usepackage{tocloft}
\usepackage{hyperref}
\usepackage{listings}
\usepackage{xcolor}
\usepackage{float}
\usepackage{caption}
\usepackage{subcaption}
\usepackage{booktabs}
\usepackage{longtable}
\usepackage{enumerate}
\usepackage{tikz}
\usepackage{pgfplots}

% Postavke stranice
\geometry{
	left=3cm,
	right=2.5cm,
	top=2.5cm,
	bottom=2.5cm,
	headheight=16pt
}

% Postavke za kodove
\lstset{
	basicstyle=\ttfamily\small,
	keywordstyle=\color{blue},
	commentstyle=\color{gray},
	stringstyle=\color{red},
	numbers=left,
	numberstyle=\tiny,
	breaklines=true,
	frame=single,
	language=R
}

% Hyperref postavke
\hypersetup{
	colorlinks=true,
	linkcolor=blue,
	filecolor=magenta,
	urlcolor=cyan,
	citecolor=red,
	pdftitle={Ekološko modeliranje i predviđanje},
	pdfauthor={Autor},
	pdfsubject={Ekologija, Modeliranje},
	pdfkeywords={ekologija, modeliranje, predviđanje, bioraznolikost}
}

% Zaglavlja i podnožja
\pagestyle{fancy}
\fancyhf{}
\fancyhead[LE,RO]{\thepage}
\fancyhead[LO]{\rightmark}
\fancyhead[RE]{\leftmark}

% Naslovni stil
\titleformat{\chapter}[display]
{\normalfont\huge\bfseries}{\chaptertitlename\ \thechapter}{20pt}{\Huge}
\titlespacing*{\chapter}{0pt}{50pt}{40pt}

\begin{document}
	
	%% NASLOVNICA
	\begin{titlepage}
		\centering
		\vspace*{2cm}
		
		{\huge\bfseries Ekološko modeliranje i predviđanje}
		
		\vspace{1.5cm}
		
		{\Large Udžbenik za studente ekologije, biologije i znanosti o okolišu}
		
		\vspace{3cm}
		
		{\large Autor}
		
		\vspace{2cm}
		
		% \includegraphics[width=0.3\textwidth]{logo.png} % Ukloniti komentar i dodati logo
		
		\vspace{2cm}
		
		{\large Sveučilište}\\
		{\large Fakultet}\\
		{\large Godina}
		
	\end{titlepage}
	
	%% SADRŽAJ
	\frontmatter
	\tableofcontents
	\listoffigures
	\listoftables
	
	%% PREDGOVOR
	\chapter*{Predgovor}
	\addcontentsline{toc}{chapter}{Predgovor}
	
	Ekološko modeliranje predstavlja ključnu metodologiju za razumijevanje složenih procesa u prirodi i predviđanje budućih promjena u ekološkim sustavima. Ovaj udžbenik nastao je s ciljem pružanja sveobuhvatnog uvida u teorijske osnove i praktične aplikacije modeliranja u ekologiji.
	
	Udžbenik je namijenjen studentima preddiplomskih i diplomskih studija ekologije, biologije, znanosti o okolišu te srodnih disciplina. Također može služiti kao referentni materijal za istraživače i praktičare koji se bave zaštitom okoliša i upravljanjem prirodnim resursima.
	
	\mainmatter
	
	%% DIO I: OSNOVE EKOLOŠKOG MODELIRANJA
	\part{Osnove ekološkog modeliranja}
	
	\chapter{Uvod u ekološko modeliranje}
	
	\section{Što je ekološko modeliranje?}
	
	Ekološko modeliranje predstavlja interdisciplinarnu znanstvenu metodologiju koja koristi matematičke, statističke i računalne alate za opisivanje, razumijevanje i predviđanje ekoloških procesa i obrazaca.
	
	\subsection{Definicija i opseg}
	Model u ekologiji možemo definirati kao pojednostavljenu reprezentaciju stvarnog ekološkog sustava koja nam omogućuje:
	\begin{itemize}
		\item Testiranje hipoteza o funkcioniranju ekoloških procesa
		\item Predviđanje odgovora sustava na promjene
		\item Integraciju znanja iz različitih izvora
		\item Identifikaciju ključnih procesa i varijabli
	\end{itemize}
	
	\subsection{Matematički okvir}
	Osnovni matematički pristup ekološkom modeliranju možemo izraziti kao:
	\begin{equation}
		\frac{dN}{dt} = f(N, t, \theta, \epsilon)
	\end{equation}
	gdje je $N$ varijabla stanja (npr. broj jedinki), $t$ vrijeme, $\theta$ parametri modela, a $\epsilon$ stohastička komponenta.
	
	\section{Povijesni razvoj ekološkog modeliranja}
	
	\subsection{Pioniri ekološkog modeliranja}
	\begin{description}
		\item[Malthus (1798)] Prvi eksponencijalni model rasta populacije
		\item[Verhulst (1838)] Logistički model rasta
		\item[Lotka-Volterra (1925-1926)] Modeli grabljivac-plijen
		\item[Leslie (1945)] Matrični modeli populacije
	\end{description}
	
	\section{Vrste ekoloških modela}
	
	\subsection{Klasifikacija prema pristupu}
	\begin{enumerate}
		\item \textbf{Empirijski modeli} - temelje se na statističkim odnosima u podacima
		\item \textbf{Mehanistički modeli} - uključuju eksplicitne ekološke procese
		\item \textbf{Fenomenološki modeli} - opisuju obrasce bez eksplicitnih mehanizama
	\end{enumerate}
	
	\subsection{Klasifikacija prema vremenu}
	\begin{itemize}
		\item Statički modeli
		\item Dinamički modeli
		\item Stohastički modeli
	\end{itemize}
	
	\section{Uloga modeliranja u ekološkim istraživanjima}
	
	Modeliranje u ekologiji služi nekoliko ključnih funkcija:
	
	\begin{table}[H]
		\centering
		\caption{Funkcije ekološkog modeliranja}
		\begin{tabular}{@{}ll@{}}
			\toprule
			Funkcija & Opis \\
			\midrule
			Deskriptivna & Opisivanje postojećih obrazaca \\
			Eksplanatorna & Objašnjavanje uzročno-posljedičnih veza \\
			Prediktivna & Predviđanje budućih stanja \\
			Preskriptivna & Preporučivanje upravljačkih mjera \\
			\bottomrule
		\end{tabular}
	\end{table}
	
	\section{Etičke i metodološke napomene}
	
	\subsection{Odgovornost modelara}
	Znanstvenici koji razvijaju ekološke modele trebaju:
	\begin{itemize}
		\item Jasno komunicirati ograničenja modela
		\item Transparentno prikazivati nesigurnosti
		\item Izbjegavati prekompliciranje bez opravdanja
		\item Validirati modele na nezavisnim podacima
	\end{itemize}
	
	\chapter{Matematičke osnove}
	
	\section{Linearna algebra u ekologiji}
	
	\subsection{Matrični modeli populacije}
	Leslie matrica za strukturirane populacije:
	\begin{equation}
		\mathbf{n}_{t+1} = \mathbf{L} \mathbf{n}_t
	\end{equation}
	
	gdje je:
	\begin{equation}
		\mathbf{L} = \begin{pmatrix}
			F_1 & F_2 & F_3 & \cdots & F_n \\
			P_1 & 0 & 0 & \cdots & 0 \\
			0 & P_2 & 0 & \cdots & 0 \\
			\vdots & \vdots & \ddots & \ddots & \vdots \\
			0 & 0 & 0 & P_{n-1} & 0
		\end{pmatrix}
	\end{equation}
	
	\subsection{Vlastite vrijednosti i vektori}
	Dominantna vlastita vrijednost $\lambda_1$ predstavlja asimptotsku stopu rasta populacije:
	\begin{equation}
		\lambda_1 = \lim_{t \to \infty} \frac{N_{t+1}}{N_t}
	\end{equation}
	
	\section{Diferencijalne jednadžbe}
	
	\subsection{Obične diferencijalne jednadžbe}
	Osnovni oblik:
	\begin{equation}
		\frac{dy}{dt} = f(t, y)
	\end{equation}
	
	\subsection{Logistički rast}
	\begin{equation}
		\frac{dN}{dt} = rN\left(1 - \frac{N}{K}\right)
	\end{equation}
	
	Rješenje:
	\begin{equation}
		N(t) = \frac{K}{1 + \left(\frac{K-N_0}{N_0}\right)e^{-rt}}
	\end{equation}
	
	\section{Teorija vjerojatnosti i statistika}
	
	\subsection{Osnovni koncepti}
	\begin{itemize}
		\item Slučajne varijable i distribucije
		\item Bayesovska vs. frekventistička statistika
		\item Maximum likelihood procjena
		\item Interval povjerenja vs. kredibilni interval
	\end{itemize}
	
	\subsection{Stohastičnost u ekološkim modelima}
	Tri tipa stohastičnosti:
	\begin{enumerate}
		\item \textbf{Demografska stohastičnost} - varijabilnost na razini pojedinaca
		\item \textbf{Okolišna stohastičnost} - varijabilnost parametara kroz vrijeme
		\item \textbf{Stohastičnost katastrofa} - rijetki događaji velikog utjecaja
	\end{enumerate}
	
	\chapter{Tipovi ekoloških modela}
	
	\section{Empirijski modeli}
	
	Empirijski modeli oslanjaju se primarno na statističke odnose u podacima bez eksplicitnog modeliranja ekoloških procesa.
	
	\subsection{Regresijski modeli}
	\begin{equation}
		y = \beta_0 + \beta_1 x_1 + \beta_2 x_2 + \cdots + \beta_p x_p + \epsilon
	\end{equation}
	
	\subsection{Generalizirani linearni modeli (GLM)}
	\begin{equation}
		g(\mu_i) = \mathbf{x}_i^T \boldsymbol{\beta}
	\end{equation}
	
	gdje je $g$ link funkcija, $\mu_i = E[Y_i]$, a $\mathbf{x}_i$ vektor kovarijata.
	
	\section{Mehanistički modeli}
	
	Mehanistički modeli eksplicitno uključuju ekološke procese kao što su rođenje, smrt, migracija i interakcije između vrsta.
	
	\subsection{Prednosti mehanističkih modela}
	\begin{itemize}
		\item Bolje razumijevanje uzročno-posljedičnih veza
		\item Mogućnost ekstrapolacije izvan raspon podataka
		\item Testiranje različitih scenarija
		\item Identificiranje ključnih procesa
	\end{itemize}
	
	\subsection{Izazovi}
	\begin{itemize}
		\item Složenost parametrizacije
		\item Potreba za detaljnim podacima
		\item Računalna zahtjevnost
		\item Nesigurnost parametara
	\end{itemize}
	
	%% DIO II: MODELI POPULACIJSKE DINAMIKE
	\part{Modeli populacijske dinamike}
	
	\chapter{Jednostavni populacijski modeli}
	
	\section{Eksponencijalni rast}
	
	Najjednostavniji model kontinuiranog rasta populacije:
	\begin{equation}
		\frac{dN}{dt} = rN
	\end{equation}
	
	Rješenje:
	\begin{equation}
		N(t) = N_0 e^{rt}
	\end{equation}
	
	\subsection{Diskretni eksponencijalni model}
	\begin{equation}
		N_{t+1} = \lambda N_t
	\end{equation}
	
	gdje je $\lambda = e^r$ konačna stopa rasta.
	
	\section{Logistički rast}
	
	Model koji uključuje nosivost staništa:
	\begin{equation}
		\frac{dN}{dt} = rN\left(1 - \frac{N}{K}\right)
	\end{equation}
	
	\subsection{Stabilnost ravnotežnih točaka}
	\begin{itemize}
		\item $N^* = 0$ (nestabilna ravnoteža)
		\item $N^* = K$ (stabilna ravnoteža)
	\end{itemize}
	
	\section{Diskretni populacijski modeli}
	
	\subsection{Generalizirani diskretni model}
	\begin{equation}
		N_{t+1} = f(N_t)
	\end{equation}
	
	\subsection{Ricker model}
	\begin{equation}
		N_{t+1} = N_t e^{r(1-N_t/K)}
	\end{equation}
	
	\subsection{Beverton-Holt model}
	\begin{equation}
		N_{t+1} = \frac{\lambda N_t}{1 + \frac{\lambda - 1}{K}N_t}
	\end{equation}
	
	\chapter{Međuspecijske interakcije}
	
	\section{Modeli grabljivac-plijen}
	
	\subsection{Lotka-Volterra model}
	Sustav jednadžbi:
	\begin{align}
		\frac{dN}{dt} &= rN - aNP \\
		\frac{dP}{dt} &= eaNP - mP
	\end{align}
	
	gdje su:
	\begin{itemize}
		\item $N$ - gustoća plijena
		\item $P$ - gustoća grabljivca
		\item $r$ - intrinzična stopa rasta plijena
		\item $a$ - stopa napada
		\item $e$ - efikasnost konverzije
		\item $m$ - stopa smrtnosti grabljivca
	\end{itemize}
	
	\subsection{Holling funkcijska odgovor}
	Tip I (linearan):
	\begin{equation}
		f(N) = aN
	\end{equation}
	
	Tip II (zasićen):
	\begin{equation}
		f(N) = \frac{aN}{1 + ahN}
	\end{equation}
	
	Tip III (sigmoidni):
	\begin{equation}
		f(N) = \frac{aN^2}{1 + ahN^2}
	\end{equation}
	
	\section{Modeli konkurencije}
	
	\subsection{Lotka-Volterra konkurencija}
	\begin{align}
		\frac{dN_1}{dt} &= r_1 N_1 \left(1 - \frac{N_1 + \alpha_{12}N_2}{K_1}\right) \\
		\frac{dN_2}{dt} &= r_2 N_2 \left(1 - \frac{N_2 + \alpha_{21}N_1}{K_2}\right)
	\end{align}
	
	\subsection{Uvjeti koegzistencije}
	Koegzistencija je moguća kada:
	\begin{equation}
		\alpha_{12} < \frac{K_1}{K_2} \quad \text{i} \quad \alpha_{21} < \frac{K_2}{K_1}
	\end{equation}
	
	%% DIO III: PROSTORNI EKOLOŠKI MODELI
	\part{Prostorni ekološki modeli}
	
	\chapter{Uvod u prostorno modeliranje}
	
	\section{Prostorna heterogenost u ekologiji}
	
	Prostorna heterogenost ključna je karakteristika ekoloških sustava koja utječe na:
	\begin{itemize}
		\item Distribuciju vrsta
		\item Populacijsku dinamiku
		\item Međuspecijske interakcije
		\item Procese na razini zajednice
	\end{itemize}
	
	\subsection{Skale prostorne heterogenosti}
	\begin{table}[H]
		\centering
		\caption{Prostorne skale u ekologiji}
		\begin{tabular}{@{}lll@{}}
			\toprule
			Skala & Razmjer & Procesi \\
			\midrule
			Lokalna & $<$ 1 km & Mikroklima, konkurencija \\
			Krajobrazna & 1-100 km & Metapopulacije, fragmentacija \\
			Regionalna & 100-1000 km & Biogeografija, migracije \\
			Kontinentalna & $>$ 1000 km & Filogenetska raznolikost \\
			\bottomrule
		\end{tabular}
	\end{table}
	
	\section{GIS i daljinsko istraživanje}
	
	\subsection{Geoinformacijski sustavi (GIS)}
	GIS omogućuje:
	\begin{itemize}
		\item Prostornu analizu ekoloških podataka
		\item Integraciju različitih tipova podataka
		\item Vizualizaciju prostornih obrazaca
		\item Modeliranje prostornih procesa
	\end{itemize}
	
	\subsection{Daljinsko istraživanje}
	Ključni izvori podataka:
	\begin{itemize}
		\item Landsat sateliti
		\item MODIS (Moderate Resolution Imaging Spectroradiometer)
		\item Sentinel programi
		\item LIDAR podaci
	\end{itemize}
	
	%% DODATCI
	\appendix
	
	\chapter{Matematički popis formula}
	
	\section{Osnovni populacijski modeli}
	
	\begin{table}[H]
		\centering
		\caption{Pregled osnovnih formula}
		\begin{tabular}{@{}ll@{}}
			\toprule
			Model & Formula \\
			\midrule
			Eksponencijalni rast & $\frac{dN}{dt} = rN$ \\
			Logistički rast & $\frac{dN}{dt} = rN(1-N/K)$ \\
			Ricker model & $N_{t+1} = N_t e^{r(1-N_t/K)}$ \\
			Beverton-Holt & $N_{t+1} = \frac{\lambda N_t}{1 + (\lambda-1)N_t/K}$ \\
			\bottomrule
		\end{tabular}
	\end{table}
	
	\chapter{Softverski kodovi i skripte}
	
	\section{R kod za logistički rast}
	
	\begin{lstlisting}[language=R, caption=Implementacija logističkog modela u R]
		# Parametri modela
		r <- 0.1      # Intrinzicka stopa rasta
		K <- 1000     # Nosivost staništa
		N0 <- 10      # Pocetna velicina populacije
		t_max <- 100  # Maksimalno vrijeme
		
		# Vremenska serija
		t <- seq(0, t_max, by = 0.1)
		
		# Analiticko rjesenje
		N_analytical <- K / (1 + ((K - N0) / N0) * exp(-r * t))
		
		# Numericka integracija
		library(deSolve)
		
		logistic_model <- function(t, state, parameters) {
			with(as.list(c(state, parameters)), {
				dN <- r * N * (1 - N / K)
				return(list(dN))
			})
		}
		
		parameters <- c(r = r, K = K)
		state <- c(N = N0)
		
		N_numerical <- ode(y = state, times = t, 
		func = logistic_model, 
		parms = parameters)
		
		# Vizualizacija
		plot(t, N_analytical, type = "l", col = "blue", 
		xlab = "Vrijeme", ylab = "Velicina populacije",
		main = "Logisticki rast populacije")
		lines(N_numerical[, "time"], N_numerical[, "N"], 
		col = "red", lty = 2)
		legend("bottomright", legend = c("Analiticko", "Numericko"),
		col = c("blue", "red"), lty = c(1, 2))
	\end{lstlisting}
	
	\chapter{Baze podataka i resursi}
	
	\section{Online baze podataka}
	\begin{itemize}
		\item GBIF (Global Biodiversity Information Facility)
		\item WorldClim - klimatski podaci
		\item IUCN Red List
		\item eBird
		\item Ocean Biogeographic Information System (OBIS)
	\end{itemize}
	
	\section{Softverski paketi}
	\subsection{R paketi}
	\begin{itemize}
		\item \texttt{deSolve} - rješavanje diferencijalnih jednadžbi
		\item \texttt{popbio} - populacijska biologija
		\item \texttt{dismo} - modeli distribucije vrsta
		\item \texttt{vegan} - analiza zajednica
		\item \texttt{adehabitat} - analiza staništa
	\end{itemize}
	
	\chapter{Rječnik pojmova}
	
	\begin{description}
		\item[Bioraznolikost] Varijabilnost živih organizama na genskoj, vrstarskoj i ekosistemskoj razini
		\item[Nosivost staništa] Maksimalna veličina populacije koju određeno stanište može podržati
		\item[Metapopulacija] Skup lokalnih populacija povezanih migracijama
		\item[Ekološka niša] Multidimenzionalni prostor uvjeta i resursa potrebnih vrsti za preživljavanje
		\item[Stohastičnost] Slučajnost u ekološkim procesima
	\end{description}
	
	%% LITERATURA
	\backmatter
	
	\begin{thebibliography}{99}
		
		\bibitem{Caswell2001}
		Caswell, H. (2001). \textit{Matrix Population Models: Construction, Analysis, and Interpretation}. Sinauer Associates.
		
		\bibitem{Gotelli2008}
		Gotelli, N. J. (2008). \textit{A Primer of Ecology}. Sinauer Associates.
		
		\bibitem{Hastings1997}
		Hastings, A. (1997). \textit{Population Biology: Concepts and Models}. Springer-Verlag.
		
		\bibitem{Hilborn1997}
		Hilborn, R., \& Mangel, M. (1997). \textit{The Ecological Detective: Confronting Models with Data}. Princeton University Press.
		
		\bibitem{May2001}
		May, R. M. (2001). \textit{Stability and Complexity in Model Ecosystems}. Princeton University Press.
		
		\bibitem{Odum1971}
		Odum, E. P. (1971). \textit{Fundamentals of Ecology}. W.B. Saunders Company.
		
		\bibitem{Roughgarden1998}
		Roughgarden, J. (1998). \textit{Primer of Ecological Theory}. Prentice Hall.
		
		\bibitem{Tilman1982}
		Tilman, D. (1982). \textit{Resource Competition and Community Structure}. Princeton University Press.
		
	\end{thebibliography}
	
\end{document}