% !TEX TS-program = lualatex
\documentclass[11pt,oneside]{book}

% ====== JEZIK I KODNA STRANA ======
\usepackage{fontspec} % za LuaLaTeX/XeLaTeX; uklonite i dodajte inputenc+fontenc ako koristite pdfLaTeX
\defaultfontfeatures{Ligatures=TeX}
\setmainfont{Latin Modern Roman}
\usepackage[croatian]{babel}

% ====== GEOMETRIJA I OSNOVNI PAKETI ======
\usepackage[a4paper,margin=2.8cm]{geometry}
\usepackage{graphicx}
\usepackage{microtype}
\usepackage{csquotes}
\usepackage{booktabs}
\usepackage{longtable}
\usepackage{array}
\usepackage{amsmath,amssymb}
\usepackage{siunitx}
\usepackage{enumitem}
\setlist{itemsep=2pt,topsep=2pt}

% ====== HIPERLINKOVI I NASLOV ======
\usepackage[hidelinks]{hyperref}
\usepackage[capitalise,nameinlink]{cleveref}

% ====== KÔD I LISTINZI ======
\usepackage{listings}
\lstset{
	basicstyle=\ttfamily\small,
	columns=fullflexible,
	frame=single,
	breaklines=true,
	showstringspaces=false
}

% ====== BIBLIOGRAFIJA (Biber) ======
\usepackage[
backend=biber,
style=authoryear,
sorting=nyt,
giveninits=true,
maxbibnames=6
]{biblatex}
\addbibresource{literatura.bib} % napravite datoteku literatura.bib

% ====== META PODACI ======
\title{Ekološko modeliranje i predviđanje\\[4pt]\large Sveobuhvatni koncept sveučilišnog udžbenika}
\author{Branimir Hackenberger}
\date{2025}

% ====== POMOĆNI MAKROI ======
\newcommand{\R}{\textsf{R}}
\newcommand{\Python}{\textsf{Python}}

\begin{document}
	\frontmatter
	\maketitle
	\tableofcontents
	
	\chapter*{Predgovor}
	Ovaj udžbenik sintetizira teorijske temelje, metode i praktične primjene ekološkog
	modeliranja s posebnim naglaskom na validaciju, nesigurnost i prediktivne scenarije.
	Struktura je objedinjena iz dvije početne skice i prilagođena kao koherentan kurikulum
	za preddiplomske i diplomske studije ekologije, biologije i znanosti o okolišu, te kao
	referentni priručnik za istraživače i praktičare.\footnote{Temeljni raspored poglavlja,
		definicije i okviri preuzeti su i integrirani iz \textit{Koncept.pdf} i \textit{Koncept2.pdf}.}
	
	\mainmatter
	
	% =====================================================
	\part{Osnove i matematički okvir}
	% =====================================================
	
	\chapter{Uvod u ekološko modeliranje}
\section{Što je ekološko modeliranje?}

Ekološko modeliranje predstavlja interdisciplinarno područje koje koristi matematičke, statističke i računalne alate za opisivanje, razumijevanje i predviđanje ekoloških procesa i obrazaca. Ova definicija, premda koncizna, skriva složenost i širinu discipline koja je postala nezaobilazan dio moderne ekološke znanosti.

U svojoj srži, ekološko modeliranje nastoji prevesti složene prirodne procese u formalne, kvantitativne okvire koji omogućavaju sistemsku analizu i razumijevanje. Ova transformacija složenih ekoloških sustava u matematičke reprezentacije nije samo akademska vježba, već predstavlja moćan alat za rješavanje praktičnih problema u zaštiti okoliša, upravljanju prirodnim resursima i predviđanju ekoloških promjena.

Važnost ekološkog modeliranja proizlazi iz fundamentalne prirode ekoloških sustava. Prirodni ekosustavi karakteriziraju nelinearne interakcije između brojnih komponenti, višestruke povratne veze, prostorna i vremenska varijabilnost, te inherentna nesigurnost. Tradicionalni eksperimentalni pristup, premda neprocjenjiv, često je ograničen u svojoj sposobnosti obuhvaćanja pune složenosti ekoloških procesa. Modeliranje pruža okvir za integraciju znanja iz različitih izvora, omogućava testiranje hipoteza na načine koji nisu praktični ili etični u prirodi, te pomaže u razumijevanju emergentnih svojstava koja proizlaze iz složenih interakcija.

Ekološki modeli služe kao konceptualni mostovi između teorije i empirijskih opažanja. Oni omogućavaju istraživačima da formaliziraju svoje razumijevanje ekoloških procesa, testiraju konzistentnost svojih hipoteza i identificiraju ključne znanja koja nedostaju. Kroz proces modeliranja, često otkrivamo da naše intuitivno razumijevanje složenih sustava može biti nepotpuno ili čak pogrešno, što pokreće daljnja istraživanja i poboljšava naše znanje.

Primjene ekološkog modeliranja protežu se kroz spektar prostornih i vremenskih skala, od molekularnih interakcija unutar organizama do globalnih biogeokemijskih ciklusa koji se odvijaju kroz tisućljeća. Na individualnoj razini, modeli mogu opisivati fiziološke procese poput energetskog metabolizma ili reproduktivnih ciklusa. Na populacijskoj razini, modeliranje pomaže u razumijevanju dinamike brojnosti, strukture starosti i prostornog rasprostiranja vrsta. Modeli zajednica i ekosustava fokusiraju se na međuspecijske interakcije, tokove energije i hranjive tvari, te održavanje bioraznolikosti.

Jedan od ključnih aspekata ekološkog modeliranja je njegova sposobnost integracije znanja iz različitih disciplina. Moderna ekološka istraživanja sve više postaju interdisciplinarna, kombinirajući principe iz biologije, kemije, fizike, geografije, matematike i informatike. Modeliranje omogućava sinteze znanja iz ovih različitih područja u koherentne okvire koji mogu pružiti nova uvida u funkcioniranje prirodnih sustava.

Tehnološki napredak, posebice u područjima računalnih znanosti i umjetne inteligencije, značajno je proširio mogućnosti ekološkog modeliranja. Dostupnost velikih baza podataka, poboljšani algoritmi strojnog učenja i povećana računalna moć omogućili su razvoj sofisticiranih modela koji mogu obraditi kompleksne skupove podataka i uhvatiti suptilne obrasce koji prije nisu bili uočljivi.

Međutim, ekološko modeliranje nije bez izazova. Prirodna varijabilnost, nelinearnost sustava, ograničenja podataka i inherentne nesigurnosti predstavljaju stalne izazove modelarima. Uspješno modeliranje zahtijeva ne samo tehničku ekspertizu, već i duboko razumijevanje ekoloških principa, kritičko mišljenje o ograničenjima modela i sposobnost komunikacije rezultata različitim auditorijima.

\section{Definicije, opseg i uloga}

Ekološko modeliranje može se klasificirati prema svojim primarnim ciljevima i funkcijama. Ova klasifikacija pomaže u razumijevanju različitih pristupa modeliranju i njihovih specifičnih primjena u ekološkoj znanosti i upravljanju okolišem.

\textbf{Deskriptivno modeliranje} predstavlja temeljnu razinu ekološkog modeliranja čiji je primarni cilj opisivanje i kvantifikacija opaženih obrazaca u prirodi. Ovi modeli nastoje odgovoriti na pitanje ``što se događa?'' kroz sistemsku analizu i sažimanje empirijskih podataka. Deskriptivni modeli često služe kao prvi korak u razumijevanju složenih ekoloških fenomena, omogućavajući identificiranje ključnih varijabli, otkrivanje obrazaca i trendova te karakterizaciju varijabilnosti sustava.

Primjeri deskriptivnog modeliranja uključuju analizu vremenskih serija populacijskih abundance, karakterizaciju prostornih obrazaca rasprostiranja vrsta, kvantifikaciju sezonskih promjena u produktivnosti ekosustava ili opisivanje strukture ekoloških mreža. Regresijski modeli koji opisuju odnose između abundancije vrsta i čimbenika okoliša također spadaju u ovu kategoriju. Premda mogu izgledati jednostavno, deskriptivni modeli često zahtijevaju sofisticirane statističke tehnike, posebice kada se suočavaju s kompleksnim, višedimenzionalnim skupovima podataka.

Važnost deskriptivnog modeliranja ne smije se podcjenjivati. Precizno opisivanje prirodnih obrazaca predstavlja temelj za sve daljnje analize i interpretacije. Bez dobrog deskriptivnog razumijevanja, pokušaji objašnjavanja ili predviđanja mogu biti usmjereni pogrešno ili temeljeni na netočnim pretpostavkama.

\textbf{Eksplanatorno modeliranje} ide korak dalje od pukog opisivanja te nastoji objasniti zašto se opaženi obrasci pojavljuju. Ovi modeli fokusiraju se na identificiranje i kvantifikaciju uzročno-posljedičnih veza između različitih komponenti ekoloških sustava. Eksplanatorno modeliranje odgovara na pitanje ``zašto se to događa?'' kroz testiranje hipoteza o mehanizmima koji pokreću opažene procese.

Mechanistički pristup eksplanatornog modeliranja temelji se na razumijevanju osnovnih biologijskih, kemijskih i fizičkih procesa koji djeluju u ekološkim sustavima. Na primjer, modeli populacijske dinamike koji uključuju specifične parametre rođenja, smrti, imigracije i emigracije nastoje objasniti promjene u brojnosti populacije kroz identificiranje ključnih demografskih procesa. Slično tome, modeli međuspecijskih interakcija pokušavaju objasniti koegzistenciju ili kompetitivno isključivanje kroz kvantifikaciju specifičnih mehanizama poput kompeticije za resurse ili grabljivačkih odnosa.

Eksplanatorno modeliranje često uključuje usporedbu alternativnih hipoteza ili mehanizama kroz formalne tehnike poput usporedbe modela ili Bayesovske inferencije. Ovaj pristup omogućava ne samo identificiranje najvjerojatnijih objašnjenja, već i kvantifikaciju nesigurnosti oko različitih hipoteza.

\textbf{Prediktivno modeliranje} usmjereno je na predviđanje budućih stanja ili ponašanja ekoloških sustava. Ova kategorija modeliranja odgovara na pitanje ``što će se dogoditi?'' i predstavlja kritičnu komponentu mnogih aplikacija u upravljanju okolišem i politici zaštite prirode. Prediktivni modeli mogu se fokusirati na kratkoročna predviđanja (npr. sezonske promjene u abundanciji vrsta) ili dugoročne projekcije (npr. utjecaji klimatskih promjena na bioraznolikost).

Uspješno prediktivno modeliranje zahtijeva kombinaciju dobrog razumijevanja sustava (često proizašlog iz deskriptivnih i eksplanatornih analiza) i robusnih statističkih ili matematičkih tehnika. Modeli rasprostiranja vrsta koji predviđaju buduća staništa pod različitim klimatskim scenarijima predstavljaju primjer prediktivnog modeliranja. Slično tome, populacijski modeli koji projektiraju buduće veličine populacija pod različitim upravljačkim scenarijima spadaju u ovu kategoriju.

Važan aspekt prediktivnog modeliranja je procjena i komunikacija nesigurnosti. Prirodni sustavi su inherentno varijabilni i složeni, što znači da predviđanja uvijek nose određenu razinu nesigurnosti. Kvalitetno prediktivno modeliranje mora ovu nesigurnost kvantificirati i prenijeti je krajnjim korisnicima na razumljiv način.

\textbf{Preskriptivno modeliranje} predstavlja najnapredniju razinu ekološkog modeliranja te nastoji pružiti preporuke za upravljanje ili intervencije. Ovi modeli odgovaraju na pitanje ``što trebamo učiniti?'' kombinirajući znanje o tome kako sustavi funkcioniraju s ciljevima upravljanja ili očuvanja. Preskriptivni modeli često integriraju ekološke, ekonomske i socijalne faktore te nastoje identificirati optimalne strategije upravljanja.

Optimizacijski pristupi u preskriptivnom modeliranju mogu uključivati linearno ili nelinearno programiranje, dinamičko programiranje ili metaheurističke algoritme. Na primjer, modeli za planiranje zaštićenih područja nastoje identificirati kombinacije lokacija koje maksimiziraju očuvanje bioraznolikosti uz minimiziranje troškova ili konflikata s drugim korisnicima zemljišta. Slično tome, modeli upravljanja ribljim zajednicama mogu preporučiti kvote izlova koje maksimiziraju dugoročnu održivost uz ekonomsku isplativost.

Preskriptivno modeliranje često zahtijeva uključivanje različitih interesnih skupina u proces definiranja ciljeva i ograničenja. Ovi modeli moraju balansirati različite, često konfliktne ciljeve te uzeti u obzir praktična ograničenja implementacije.

Važno je napomenuti da ove četiri kategorije nisu međusobno isključive. U praksi, kompleksni projekti modeliranja često kombiniraju elemente iz svih četirih pristupa. Deskriptivna analiza može identificirati ključne obrasce, eksplanatorno modeliranje može objasniti uzroke tih obrazaca, prediktivno modeliranje može projektirati buduće scenarije, a preskriptivno modeliranje može preporučiti odgovarajuće upravljačke akcije.

Kombinacija različitih pristupa modeliranja stvara sinergije koje omogućavaju dublje razumijevanje i efikasniju primjenu ekoloških znanja. Ova integracija predstavlja suštinu moderne primijenjene ekologije te ilustrira kako matematički i računalni alati mogu služiti kao mostovi između teorijskog razumijevanja i praktičnih rješenja za očuvanje i upravljanje prirodnim sustavima.
\section{Povijesni pregled}

Razvoj ekološkog modeliranja može se pratiti kroz evoluciju matematičkih pristupa opisivanju prirodnih procesa, pri čemu svaki značajan doprinos gradi na prethodnim spoznajama i odgovara na nova pitanja koja se pojavljuju s napretkom znanosti. Ovaj povijesni pregled prati ključne prekretnice koje su oblikovale modernu disciplinu ekološkog modeliranja, od ranih demografskih teorija do sofisticiranih matričnih pristupa koji i danas čine temelj populacijske ekologije.

\textbf{Malthusovi temelji demografskog modeliranja}

Počeci formalnog pristupa modeliranju populacijskih procesa mogu se pratiti do rada Thomasa Roberta Malthusa (1766--1834), engleskog ekonomista i demografa čiji je utjecaj daleko nadišao granice ekonomske znanosti. U svom revolucionarnom djelu ``An Essay on the Principle of Population'' (1798), Malthus je postavio temelj za kvantitativno razumijevanje populacijske dinamike kroz formulaciju jednostavnog, ali moćnog koncepta eksponencijalnog rasta.

Malthusova osnovna pretpostavka bila je da se populacije, u odsutnosti ograničavajućih faktora, povećavaju u geometrijskoj progresiji, dok se resursi potrebni za održavanje tih populacija povećavaju samo u aritmetičkoj progresiji. Ova asimetrija između potencijala rasta populacije i dostupnosti resursa predstavlja temeljni paradoks koji je Malthus identificirao kao uzrok neizbježnih kriza u ljudskim društvima.

Matematički, Malthusov model može se izraziti diferencijalnom jednadžbom:
$$\frac{dN}{dt} = rN$$

gdje je $N(t)$ veličina populacije u vremenu $t$, a $r$ predstavlja intrinsičnu stopu rasta populacije. Rješenje ove jednadžbe daje eksponencijalnu funkciju $N(t) = N_0 e^{rt}$, gdje je $N_0$ početna veličina populacije. Premda Malthus nije koristio ovu formalnu matematičku notaciju, njegova konceptualna formulacija bila je ekvivalentna ovom modelu.

Značaj Malthusova doprinosa leži ne samo u prepoznavanju eksponencijalnog karaktera neograničenog rasta, već i u razumijevanju da takav rast nije održiv u konačnom svijetu. Ova spoznaja postavila je temelje za kasnije modele koji su nastojali uključiti ograničavajuće faktore i opisati realnije scenarije populacijske dinamike.

Malthusove ideje imale su dalekosežan utjecaj na razvoj demografije, ekonomije i, što je posebno važno za našu raspravu, na Charles Darwinovo razumijevanje prirodne selekcije. Darwin je eksplicitno priznao da ga je čitanje Malthusova eseja inspiriralo na formulaciju principa opstanka najsposobnijih, što ilustrira povezanost između kvantitativnih modela i temeljnih biologijskih teorija.

\textbf{Verhulstova inovacija: uvođenje nosivosti staništa}

Pierre-François Verhulst (1804--1849), belgijski matematičar i demograf, prepoznao je fundamentalno ograničenje Malthusova modela: u stvarnosti, nijedna populacija ne može rasti eksponencijalno neograničeno. Verhulstov ključni uvid bio je da se stopa rasta populacije mora smanjivati kako se populacija približava granicama koje joj nameću dostupni resursi ili drugi čimbenici okoliša.

Godine 1838. Verhulst je uveo ono što je nazvao ``logističkom jednadžbom'':
$$\frac{dN}{dt} = rN\left(1 - \frac{N}{K}\right)$$

gdje je $K$ nova konstanta koju je Verhulst nazvao ``nosivost staništa'' (carrying capacity). Ovaj termin opisuje maksimalnu veličinu populacije koju određeni okoliš može održati neograničeno, uzimajući u obzir dostupnost hranjivih tvari, prostora, i drugih resursa.

Elegantnost Verhulstova modela leži u njegovoj jednostavnosti i intuitivnosti. Kada je populacija mala ($N \ll K$), izraz u zagradi približava se jedinici, pa se model ponaša kao Malthusov eksponencijalni model. Međutim, kako populacija raste i približava se nosivosti staništa, izraz $(1 - N/K)$ postaje sve manji, što rezultira smanjenjem efektivne stope rasta. Kada populacija dosegne nosivost ($N = K$), stopa rasta postaje nula, što znači da je populacija u ravnoteži.

Rješenje logističke jednadžbe daje sigmoidu:
$$N(t) = \frac{K}{1 + \left(\frac{K-N_0}{N_0}\right)e^{-rt}}$$

Ova krivulja karakterizira početnu fazu sporeg rasta, zatim fazu brzog eksponencijalnog rasta, i konačno fazu usporavanja kako se populacija približava asimptoti definiranoj nosivosti staništa.

Verhulstov model predstavljao je paradigmatski pomak u razumijevanju populacijske dinamike. Prvi put je formalno ukljućio koncept ograničavajućih faktora i predstavio ideju da sama veličina populacije može djelovati kao regulacijski mehanizam kroz intraspecijsku konkurenciju. Ovaj model postavio je temelje za razumijevanje density-dependent regulacije populacija, koncepta koji i danas predstavlja središnji element populacijske ekologije.

\textbf{Lotka-Volterrini modeli: rođenje međuspecijskog modeliranja}

Razvoj modela koji opisuju interakcije između različitih vrsta označava sljedeću važnu fazu u evoluciji ekološkog modeliranja. Alfred James Lotka (1880--1949) i Vito Volterra (1860--1940), radivši neovisno jedan o drugome u ranim desetljećima 20. stoljeća, razvili su sustave diferencijalnih jednadžbi koji opisuju dinamiku dvije vrste u interakciji.

Volterra, talijanski matematičar i fizičar, potaknut je na ovaj rad praktičnim problemom. Njegov kolega, italijanski biolog Umberto D'Ancona, primijetio je neočekivane promjene u ulovoj ribljih zajednica u Jadranskom moru tijekom Prvog svjetskog rata. Konkretno, udio grabljivačkih riba u ukupnom ulovu značajno se povećao tijekom rata, kada je ribolov bio ograničen. D'Ancona je zatražio Volterrin savjet za objašnjenje ovog fenomena.

Volterra je pristup ovom problemu kroz matematičku analizu grabljivac-plijen sustava, razvivši sustav jednadžbi koji opisuje dinamiku dvije vrste:

$$\frac{dN}{dt} = rN - aNP$$
$$\frac{dP}{dt} = eaNP - mP$$

gdje $N$ predstavlja abundanciju plijena, $P$ abundanciju grabljivca, $r$ je intrinsična stopa rasta plijena, $a$ je stopa napada grabljivca, $e$ je efikasnost konverzije plijena u potomstvo grabljivca, a $m$ je stopa smrti grabljivca.

Istovremeno, Lotka je razvijao slične modele u kontekstu kemijskih reakcija i općenite teorije sustava, pri čemu je prepoznao da se isti matematički formalizam može primijeniti na različite prirodne procese, uključujući međuspecijske interakcije u ekologiji.

Analiza Lotka-Volterrinih jednadžbi otkriva fascinantna svojstva. Sustav ima neutralnu stabilnost, što znači da oscilira u zatvorenim orbitama oko ravnotežne točke. Ove oscilacije imaju konstantnu amplitudu i period koji ovisi o početnim uvjetima, ali ne konvergiraju k ravnotežnoj točki niti se udaljavaju od nje. Ovakvo ponašanje predstavlja idealizirani scenarij u kojem grabljivac i plijen osciliraju u savršenoj sinkronizaciji.

Značaj Lotka-Volterrinih modela nadilazi njihovu specifičnu primjenu na grabljivac-plijen sustave. Oni predstavljaju prvi formalni pristup modeliranju međuspecijskih interakcija i postavljaju temelje za razumijevanje dinamike složenih ekoloških zajednica. Dodatno, uvode važne koncepte poput funkcijskih odgovora (kako se stopa predacije mijenja s abundancijom plijena) i vremenskih kašnjenja u ekološkim procesima.

Premda osnovni Lotka-Volterrin model čini značajne pretpostavke (poput nepostojanja nosivosti staništa za plijen i konstantnih parametara), on je stimulirao razvoj brojnih proširenja i modifikacija. Ovi uključuju uvođenje nosivosti staništa za plijen, alternativnih funkcijskih odgovora, prostorne heterogenosti, i vremenskih varijacija u parametrima.

\textbf{Lesliejeve matrice: revolucija u strukturiranom modeliranju}

Patrick Holt Leslie (1900--1972), škotski biomatemematičar, revolucionizirao je populacijsko modeliranje uvođenjem matričnih pristupa koji omogućavaju opis populacija s kompleksnom dobnom ili stadijskom strukturom. Leslie je prepoznao da je jednostavno tretiranje populacije kao homogene cjeline často nerealno, jer se demografski parametri (stope rođenja, smrti, reprodukcije) značajno razlikuju između različitih dobnih skupina.

Leslie matrica, formulirana 1940-ih godina, predstavlja način opisivanja populacijske dinamike kroz diskretne vremenske korake, pri čemu se populacija dijeli u dobne klase ili razvojne stadije. Osnovna forma Leslie matrice je:

$$\mathbf{L} = \begin{pmatrix}
	F_1 & F_2 & F_3 & \cdots & F_k \\
	P_1 & 0 & 0 & \cdots & 0 \\
	0 & P_2 & 0 & \cdots & 0 \\
	\vdots & \vdots & \ddots & \ddots & \vdots \\
	0 & 0 & 0 & P_{k-1} & 0
\end{pmatrix}$$

gdje $F_i$ predstavlja plodnost $i$-te dobne klase (broj potomaka koji će preživjeti do prve dobne klase), a $P_i$ predstavlja vjerojatnost preživljavanja iz $i$-te u $(i+1)$-vu dobnu klasu.

Populacijski vektor u vremenu $t+1$ može se izračunati kao:
$$\mathbf{n}(t+1) = \mathbf{L} \mathbf{n}(t)$$

gdje $\mathbf{n}(t) = [n_1(t), n_2(t), \ldots, n_k(t)]^T$ predstavlja vektor abundancija u različitim dobnim klasama.

Moć Lesliejevih matrica leži u njihovoj sposobnosti uhvaćanja ključnih aspekata populacijske strukture i dinamike. Dominantna vlastita vrijednost matrice ($\lambda_1$) predstavlja asimptotsku stopu rasta populacije, dok odgovarajući vlastiti vektor opisuje stabilnu dobnu strukturu ka kojoj populacija konvergira. Ako je $\lambda_1 > 1$, populacija raste; ako je $\lambda_1 < 1$, populacija opada; a ako je $\lambda_1 = 1$, populacija je u ravnoteži.

Leslie je također razvio koncepte osjetljivosti i elastičnosti, koji omogućavaju kvantifikaciju utjecaja promjena u demografskim parametrima na populacijski rast. Osjetljivost mjeri apsolutnu promjenu u $\lambda$ s obzirom na malu promjenu u elementu matrice, dok elastičnost mjeri proporcionalnu promjenu. Ovi koncepti postali su fundamentalni alati u populacijskoj biologiji i biologiji zaštite prirode.

Značaj Lesliejevih matrica transcendira njihovu izvornu primjenu. One predstavljaju mostove između individualnih demografskih procesa i populacijskih obrazaca, omogućavaju rigoroznu analizu demografskih podataka, i pružaju okvir za razumijevanje evolucijskih strategija životnog ciklusa. Dodatno, matični pristup omogućava lako proširivanje na složenije strukture, poput metapopulacija ili zajednica s međuspecijskim interakcijama.

\textbf{Sinteza i naslijeđe}

Ovi pionirski doprinosi uspostavili su temeljna načela ekološkog modeliranja koja i danas oblikuju disciplinu. Malthusov eksponencijalni model uspostavio je kvantitativni pristup demografiji i prepoznao fundamentalni potencijal rasta organizama. Verhulstov logistički model uveo je koncepte ograničavajućih faktora i density-dependent regulacije. Lotka-Volterrine jednadžbe otvorile su područje međuspecijskog modeliranja i pokazale kako se složeni ekološki obrasci mogu nastati iz jednostavnih interakcija. Lesliejeve matrice omogućile su rigorozno tretiranje struktuiranih populacija i povezale individualnu demografiju s populacijskim obrascima.

Svaki od ovih doprinosa predstavlja ne samo tehnički napredak, već i konceptualnu inovaciju koja je proširila naše razumijevanje prirodnih procesa. Oni ilustriraju evolucijski karakter znanosti, gdje se nova znanja grade na prethodnim temeljima, istovremeno proširujući granice mogućeg. Kombinacija matematičke elegantnosti, biološke relevantnosti i praktične primjenjivosti ovih modela osigurava im trajno mjesto u kanonu ekološke znanosti.

Moderna ekološka modeliranja nastavlja se oslanjati na ove temelje, proširujući ih kroz inkorporaciju stohastičnosti, prostorne heterogenosti, klimatskih promjena, i sve sofisticiranijih statističkih i računalnih tehnika. Međutim, osnovna načela identificirana od strane ovih pionira -- važnost kvantifikacije, potreba za uključivanjem ograničavajućih faktora, značaj međuspecijskih interakcija, i vrijednost strukturiranog pristupa -- ostaju središnji elementi uspješnog ekološkog modeliranja.

	\chapter{Matematički i statistički temelji}
\section{Linearna algebra i matrični modeli}

Matrični pristup populacijskom modeliranju predstavlja jedan od najelegantnijih i najmoćnijih alata u kvantitativnoj ekologiji. Ovaj pristup omogućava rigoroznu analizu populacija s kompleksnom demografskom strukturom, pružajući duboke uvide u populacijsku dinamiku kroz primjenu koncepata linearne algebre. Leslie i Lefkovitch matrice, kao temeljni predstavnici ovog pristupa, revolucionizirali su način na koji razumijemo i predviđamo populacijske promjene u strukturiranim populacijama.

\textbf{Osnove matričnog populacijskog modeliranja}

Matrično populacijsko modeliranje temelji se na fundamentalnoj ideji da se kompleksne populacije mogu podijeliti u diskretne klase na temelju dobi, veličine, razvojnog stadija ili drugih demografski relevantnih karakteristika. Umjesto tretiranja populacije kao homogene cjeline, ovaj pristup prepoznaje da se demografski parametri (stope rođenja, smrti, reprodukcije, tranzicije između stadija) značajno razlikuju između različitih skupina unutar populacije.

Osnovna forma matričnog modela populacijske dinamike može se izraziti kao:
$$\mathbf{n}(t+1) = \mathbf{A} \mathbf{n}(t)$$

gdje je $\mathbf{n}(t) = [n_1(t), n_2(t), \ldots, n_k(t)]^T$ vektor koji opisuje broj individua u svakoj od $k$ klasa u vremenu $t$, a $\mathbf{A}$ je $k \times k$ projekcijska matrica koja sadrži sve demografske parametre koji usmjeravaju tranzicije između klasa i produkciju novih individua.

Iterativnom primjenom ove jednadžbe možemo pratiti evoluciju populacijske strukture kroz vrijeme:
$$\mathbf{n}(t) = \mathbf{A}^t \mathbf{n}(0)$$

Ovaj deceptivno jednostavan izraz skriva bogato matematičko ponašanje koje omogućava duboke uvide u populacijsku dinamiku.

\textbf{Leslie matrice: klasifikacija po dobi}

Leslie matrica, nazvana prema škotskom biomatemematičaru Patricku Holtu Leslieju, predstavlja specijaliziranu formu projekcijske matrice namijenjena modeliranju populacija klasificiranih po dobi. Ova matrica ima specifičnu strukturu koja odražava biološke realnosti demografskih procesa kod organizma s diskretnim reproduktivnim sezonama.

Općenita forma Leslie matrice je:
$$\mathbf{L} = \begin{pmatrix}
	F_1 & F_2 & F_3 & \cdots & F_{k-1} & F_k \\
	P_1 & 0 & 0 & \cdots & 0 & 0 \\
	0 & P_2 & 0 & \cdots & 0 & 0 \\
	0 & 0 & P_3 & \cdots & 0 & 0 \\
	\vdots & \vdots & \vdots & \ddots & \vdots & \vdots \\
	0 & 0 & 0 & \cdots & P_{k-1} & 0
\end{pmatrix}$$

Elementi prvog reda, $F_i$, predstavljaju parametre plodnosti (fecundity) za $i$-tu dobnu klasu. Preciznije, $F_i$ označava broj ženskog potomstva koji će individua iz $i$-te dobne klase proizvesti i koji će preživjeti do prve dobne klase u sljedećem vremenskom koraku. Ovi parametri su produkt nekoliko komponenti:
$$F_i = m_i \cdot s_0 \cdot \frac{1}{2}$$

gdje je $m_i$ broj potomaka koje rađa individua iz $i$-te dobne klase, $s_0$ je vjerojatnost preživljavanja potomstva do prve dobne klase, a faktor $\frac{1}{2}$ pretpostavlja ravnotežu spolova (ako modeliramo samo ženke).

Elementi subdiagonale, $P_i$, predstavljaju vjerojatnosti preživljavanja (survival probabilities) iz $i$-te u $(i+1)$-vu dobnu klasu. Formalno:
$$P_i = \mathbb{P}(\text{individua preživljava iz klase } i \text{ u klasu } i+1)$$

Nulti elementi u ostalim pozicijama matrice odražavaju biološke pretpostavke Leslie modela: individue se mogu reproducirati ali ne mogu "ići unazad" u mlađe dobne klase, niti mogu "preskočiti" dobne klase.

\textbf{Svojstva Leslie matrica i spektralna analiza}

Ključ razumijevanja ponašanja Leslie matrica leži u njihovoj spektralnoj analizi - studiju vlastite vrijednosti i vlastite vektora. Za Leslie matricu $\mathbf{L}$, vlastite vrijednosti $\lambda$ su rješenja karakteristične jednadžbe:
$$\det(\mathbf{L} - \lambda \mathbf{I}) = 0$$

što daje karakteristični polinom:
$$\lambda^k - F_1\lambda^{k-1} - F_2P_1\lambda^{k-2} - F_3P_1P_2\lambda^{k-3} - \cdots - F_kP_1P_2\cdots P_{k-1} = 0$$

Prema Perron-Frobeniusovom teoremu, za Leslie matrice koje zadovoljavaju određene biološke uvjete (primitivity - što je obično slučaj kad bar jedna mlada dobna klasa može se reproducirati i kad postoji pozitivna vjerojatnost preživljavanja između uzastopnih dobnih klasa), postoji jedinstvena dominantna vlastita vrijednost $\lambda_1$ koja je:

1. \textbf{Realna i pozitivna}: $\lambda_1 > 0$  

2. \textbf{Jednostruka}: algebarska i geometrijska kratnost je 1  

3. \textbf{Dominantna}: $|\lambda_i| < \lambda_1$ za sve $i \neq 1$  

4. \textbf{Ima pozitivni vlastiti vektor}: $\mathbf{w}_1 > 0$    


\textbf{Dominantna vlastita vrijednost i asimptotska stopa rasta}

Dominantna vlastita vrijednost $\lambda_1$ ima fundamentalnu biološku interpretaciju kao asimptotska stopa rasta populacije. Ova interpretacija proizlazi iz asimptotskog ponašanja iteriranog matričnog modela.

Kada projiciramo populaciju daleko u budućnost, dominantna vlastita vrijednost određuje ponašanje:
$$\lim_{t \to \infty} \frac{|\mathbf{n}(t+1)|}{|\mathbf{n}(t)|} = \lambda_1$$

gdje $|\mathbf{n}(t)|$ označava ukupnu veličinu populacije u vremenu $t$.

Preciznije, asimptotsko ponašanje populacije može se opisati kao:
$$\mathbf{n}(t) \approx C\lambda_1^t \mathbf{w}_1 \quad \text{za velike } t$$

gdje je $C$ konstanta određena početnim uvjetima, a $\mathbf{w}_1$ je dominantni desni vlastiti vektor (stabilan dobni raspored).

Ova konvergencija ima duboke biološke implikacije:

- \textbf{Ako $\lambda_1 > 1$}: populacija raste eksponencijalno stopom $r = \ln(\lambda_1)$
- \textbf{Ako $\lambda_1 < 1$}: populacija opada eksponencijalno i konvergira prema izumiranju
- \textbf{Ako $\lambda_1 = 1$}: populacija je u demografskoj ravnoteži

\textbf{Stabilna dobna distribucija}

Dominantni desni vlastiti vektor $\mathbf{w}_1$ predstavlja stabilnu dobnu distribuciju (stable age distribution) ka kojoj sve populacije s istom Leslie matricom konvergiraju neovisno o početnim uvjetima. Ovaj vektor zadovoljava:
$$\mathbf{L}\mathbf{w}_1 = \lambda_1 \mathbf{w}_1$$

Normaliziranjem ovog vektora tako da $\sum_{i=1}^k w_{1i} = 1$, dobivamo proporcionalnu dobnu strukturu:
$$\mathbf{c} = \frac{\mathbf{w}_1}{\sum_{i=1}^k w_{1i}}$$

gdje $c_i$ predstavlja proporciju populacije u $i$-toj dobnoj klasi u stabilnom stanju.

Stabilna dobna distribucija može se eksplicitno izračunati. Za Leslie matricu, elementi stabilne dobne distribucije su:
$$c_1 = \frac{1}{1 + \frac{P_1}{\lambda_1} + \frac{P_1P_2}{\lambda_1^2} + \cdots + \frac{P_1P_2\cdots P_{k-1}}{\lambda_1^{k-1}}}$$

$$c_i = c_1 \cdot \frac{P_1P_2\cdots P_{i-1}}{\lambda_1^{i-1}} \quad \text{za } i = 2, 3, \ldots, k$$

\textbf{Reproduktivna vrijednost}

Lijevi vlastiti vektor $\mathbf{v}_1$ odgovarajuće dominantne vlastite vrijednosti ima interpretaciju reproduktivne vrijednosti (reproductive value). Ovaj koncept, uveo ga je R.A. Fisher, opisuje relativni doprinos različitih dobnih klasa budućem reproduktivnom outputu populacije:
$$\mathbf{v}_1^T \mathbf{L} = \lambda_1 \mathbf{v}_1^T$$

Reproduktivna vrijednost $i$-te dobne klase, $v_{1i}$, može se interpretirati kao očekivani broj potomaka koje će individua trenutno u $i$-toj dobnoj klasi proizvesti tijekom ostatka svog života, diskontiran asimptotskom stopom rasta populacije.

\textbf{Lefkovitch matrice: generalizacija na stanja}

Lefkovitch matrice, nazvane prema Leonard P. Lefkovitchu, predstavljaju generalizaciju Leslie matrica na populacije klasificirane po bilo kakvim stanjima, a ne nužno po dobi. Ova generalizacija omogućava modeliranje organizama kod kojih je veličina, razvojni stadij, ili neko drugo obilježje relevantnije za demografiju od kronološke dobi.

Općenita forma Lefkovitch matrice je:
$$\mathbf{A} = \begin{pmatrix}
	P_{1,1} + F_{1,1} & F_{1,2} & F_{1,3} & \cdots & F_{1,k} \\
	P_{2,1} & P_{2,2} & P_{2,3} & \cdots & P_{2,k} \\
	P_{3,1} & P_{3,2} & P_{3,3} & \cdots & P_{3,k} \\
	\vdots & \vdots & \vdots & \ddots & \vdots \\
	P_{k,1} & P_{k,2} & P_{k,3} & \cdots & P_{k,k}
\end{pmatrix}$$

gdje $P_{i,j}$ predstavlja vjerojatnost tranzicije iz stanja $j$ u stanje $i$, a $F_{i,j}$ predstavlja plodnost stanja $j$ koja doprinosi stanju $i$.

Ključne razlike u odnosu na Leslie matrice uključuju:

1. \textbf{Mogućnost zadržavanja u istom stanju}: dijagonalni elementi $P_{i,i}$ mogu biti pozitivni

2. \textbf{Mogućnost kretanja "unazad"}: $P_{i,j}$ može biti pozitivan za $i < j$ 
 
3. \textbf{Mogućnost "preskakanja" stanja}: $P_{i,j}$ može biti pozitivan za $|i-j| > 1$

4. \textbf{Fleksibilniji reproduktivni obrazac}: $F_{i,j}$ može biti pozitivan za bilo koje $i,j$


\textbf{Analiza osjetljivosti i elastičnosti}

Jedne od najvažnijih primjena matričnih populacijskih modela su analize osjetljivosti i elastičnosti, koje kvantificiraju kako promjene u demografskim parametrima utječu na populacijski rast.

Osjetljivost dominantne vlastite vrijednosti na promjene u elementu matrice $a_{i,j}$ definira se kao:
$$s_{i,j} = \frac{\partial \lambda_1}{\partial a_{i,j}}$$

Koristeći teoriju perturbacije vlastite vrijednosti, osjetljivost se može izračunati kao:
$$s_{i,j} = \frac{v_{1i} w_{1j}}{\langle \mathbf{v}_1, \mathbf{w}_1 \rangle}$$

gdje $\langle \mathbf{v}_1, \mathbf{w}_1 \rangle = \mathbf{v}_1^T \mathbf{w}_1$ je skalarni produkt lijevog i desnog vlastitog vektora.

Elastičnost predstavlja proporcionalnu osjetljivost:
$$e_{i,j} = \frac{a_{i,j}}{\lambda_1} \frac{\partial \lambda_1}{\partial a_{i,j}} = \frac{a_{i,j}}{\lambda_1} s_{i,j}$$

Elastičnosti zadovoljavaju važno svojstvo:
$$\sum_{i,j} e_{i,j} = 1$$

što omogućava izravnu usporedbu relativne važnosti različitih demografskih parametara.

\textbf{Dekomozicija demografskih doprinosa}

Dominantna vlastita vrijednost može se dekomponirati u doprinose različitih demografskih procesa kroz analizu petlje (loop analysis). Za Leslie matricu, osnovne demografske petlje su:

1. \textbf{Reproduktivne petlje}: $F_i \cdot \frac{P_1 P_2 \cdots P_{i-1}}{\lambda_1^i}$ i 

2. \textbf{Petlje preživljavanja}: $\frac{P_1 P_2 \cdots P_{k-1}}{\lambda_1^k}$

Suma svih petlji mora biti jednaka 1:
$$1 = \sum_{i=1}^k F_i \cdot \frac{P_1 P_2 \cdots P_{i-1}}{\lambda_1^i} + \frac{P_1 P_2 \cdots P_{k-1}}{\lambda_1^k}$$

Ova dekomozicija, poznata kao Eulerova jednadžba, povezuje sve demografske parametre s asimptotskom stopom rasta i omogućava razumijevanje relativnih doprinosa različitih putova kroz životni ciklus.

\textbf{Proširenja i primjene}

Matrični pristupi mogu se proširiti na različite načine:

1. \textbf{Stohastičnost}: Uvođenje vremenski varijabilnih matrica $\mathbf{A}(t)$ za modeliranje čimbenika okoliša

2. \textbf{Density-dependence}: Parametri matrice postaju funkcije veličine populacije

3. \textbf{Prostorna struktura}: Mega-matrice koje kombiniraju demografske i dispersalne procese

4. \textbf{Međuspecijske interakcije}: Povezani sustavi matrica za različite vrste


Matrični modeli našli su široku primjenu u biologiji zaštite prirode, upravljanju prirodnim resursima, ribolovu, šumarstvu i evolucijskoj biologiji. Njihova kombinacija matematičke rigoroznosti i biološke interpretabilnosti čini ih nezamjenjivim alatima za razumijevanje i upravljanje populacijskim procesima.

Dominantna vlastita vrijednost $\lambda_1$, kao ključni parametar koji određuje asimptotsku stopu rasta, ostaje centralna veličina u populacijskoj biologiji, omogućavajući preciznu kvantifikaciju demografskog utjecaja različitih čimbenika na dugoročnu vijabilnost populacija.


\section{Diferencijalne i diferentne jednadžbe}

\textbf{Kontinuirani vs. diskretni vremenski pristup}

Matematički opis populacijske dinamike temelji se na dva komplementarna pristupa koji se razlikuju u tretiranju vremenske varijable: kontinuiranim modelima izraženim diferencijalnim jednadžbama i diskretnim modelima izraženim diferentnim jednadžbama.

\textbf{Diferencijalne jednadžbe za kontinuirane procese}

Diferencijalne jednadžbe koriste se kada promatramo procese koji se odvijaju kontinuirano u vremenu. U ovom pristupu, vrijeme $t$ je kontinuirana varijabla koja može poprimiti bilo koju realnu vrijednost iz određenog intervala. Populacijska veličina $N(t)$ također je kontinuirana funkcija vremena.

Osnovna forma diferencijalne jednadžbe za populacijsku dinamiku je:
$$\frac{dN}{dt} = f(N,t)$$

gdje $\frac{dN}{dt}$ predstavlja trenutačnu stopu promjene populacije u vremenu $t$, a $f(N,t)$ je funkcija koja opisuje kako ta stopa ovisi o trenutačnoj veličini populacije i vremenu.

Ovaj pristup je prikladan za:
\begin{itemize}
	\item Organizme s preklapajućim generacijama
	\item Procese koji se odvijaju kontinuirano (npr. rođenja i smrti mogu nastupiti u bilo kojem trenutku)
	\item Situacije gdje su demografski procesi relativno brzi u odnosu na vremensku skalu promatranja
	\item Mikroorganizme, biljke, ili populacije s asinkronim reproduktivnim ciklusima
\end{itemize}

\textbf{Diferentne jednadžbe za diskretne procese}

Diferentne jednadžbe (također nazivane diskretnim jednadžbama ili jednadžbama razlika) koriste se kada promatramo procese koji se odvijaju u diskretnim vremenskim koracima. Vrijeme se tretira kao niz diskretnih točaka $t = 0, 1, 2, 3, ...$ koji obično predstavljaju sezone, godine, ili generacije.

Osnovna forma diferentne jednadžbe za populacijsku dinamiku je:
$$N_{t+1} = g(N_t, t)$$

gdje $N_t$ predstavlja veličinu populacije u vremenskom koraku $t$, a $N_{t+1}$ je veličina populacije u sljedećem vremenskom koraku.

Ovaj pristup je prikladan za:
\begin{itemize}
	\item Organizme s diskretnim, sinkroniziranim reproduktivnim sezonama
	\item Situacije gdje su generacije jasno odvojene
	\item Godišnje ili sezonske životne cikluse
	\item Mnoge insekte, godišnje biljke, ili populacije s izraženom sezonalnosti
	\item Slučajeve gdje su podaci dostupni samo u diskretnim vremenskim intervalima
\end{itemize}

\textbf{Matematička veza između pristupa}

Povezanost između kontinuiranih i diskretnih modela može se uspostaviti kroz koncepte kao što su:

Za mala vremenska kašnjenja $\Delta t$, diferencijalna jednadžba može se aproksimirati diferentnom jednadžbom:
$$\frac{dN}{dt} \approx \frac{N_{t+\Delta t} - N_t}{\Delta t}$$

što daje:
$$N_{t+\Delta t} \approx N_t + \Delta t \cdot f(N_t, t)$$

Obrnuto, diferentna jednadžba može se povezati s diferencijalnom kroz:
$$\ln\left(\frac{N_{t+1}}{N_t}\right) = r \quad \Rightarrow \quad \frac{N_{t+1}}{N_t} = e^r$$

gdje je $r$ kontinuirana stopa rasta povezana s diskretnom stopom rasta $\lambda = \frac{N_{t+1}}{N_t}$ preko:
$$\lambda = e^r \quad \text{ili} \quad r = \ln(\lambda)$$

Matematički opis populacijske dinamike temelji se na ovim komplementarnim pristupima, pri čemu izbor ovisi o biologiji organizma, vremenskim skalama procesa i dostupnosti podataka. Eksponencijalni i logistički modeli rasta predstavljaju temeljna rješenja koja ilustriraju ključne principe populacijske dinamike u oba okvira.

\textbf{Kontinuirani eksponencijalni rast}

Najjednostavniji model populacijske dinamike pretpostavlja da je trenutačna stopa promjene populacije proporcionalna trenutačnoj veličini populacije. Ovaj model, ursprno formuliran od strane Malthusa, može se izraziti diferencijalnom jednadžbom:

$$\frac{dN}{dt} = rN$$

gdje je $N(t)$ veličina populacije u vremenu $t$, a $r$ je intrinsična stopa rasta populacije (često nazivana Malthusovim parametrom).

Parametar $r$ ima fundamentalnu biološku interpretaciju. On predstavlja razliku između per capita stope rođenja ($b$) i per capita stope smrti ($d$):
$$r = b - d$$

Kada je $r > 0$, populacija rasta; kada je $r < 0$, populacija opada; a kada je $r = 0$, populacija je u ravnoteži.

Za rješavanje ove separabilne diferencijalne jednadžbe, možemo odvojiti varijable:
$$\frac{dN}{N} = r \, dt$$

Integriranjem obje strane:
$$\int_{N_0}^{N(t)} \frac{dN'}{N'} = \int_0^t r \, dt'$$

$$\ln N(t) - \ln N_0 = rt$$

$$\ln\left(\frac{N(t)}{N_0}\right) = rt$$

Exponenciranjem dobivamo rješenje:
$$N(t) = N_0 e^{rt}$$

gdje je $N_0 = N(0)$ početna veličina populacije.

Ovo rješenje opisuje eksponencijalni rast ($r > 0$) ili eksponencijalni pad ($r < 0$). Vrijeme udvostručivanja populacije (doubling time) može se izračunati kao:
$$t_d = \frac{\ln 2}{r}$$

dok je poluživot populacije (half-life) u slučaju opadanja:
$$t_{1/2} = \frac{\ln 2}{|r|}$$

\textbf{Kontinuirani logistički rast}

Eksponencijalni model nerealno pretpostavlja neograničene resurse. Verhulstov logistički model modificira eksponencijalni model uključivanjem density-dependent čimbenika kroz uvođenje nosivosti staništa $K$:

$$\frac{dN}{dt} = rN\left(1-\frac{N}{K}\right)$$

Ovdje izraz u zagradi $\left(1-\frac{N}{K}\right)$ predstavlja ``otpor okoliša'' - kada je $N \ll K$, otpor je minimalan i populacija raste gotovo eksponencijalno; kada se $N$ približava $K$, otpor postaje značajan i usporava rast.

Logistička jednadžba može se prepisati kao:
$$\frac{dN}{dt} = rN - \frac{r}{K}N^2$$

što jasno pokazuje da je ovo nelinearna diferencijalna jednadžba zbog kvadratnog člana.

Za rješavanje logističke jednadžbe, koristimo separaciju varijabli:
$$\frac{dN}{N(1-\frac{N}{K})} = r \, dt$$

Lijeva strana zahtijeva parcijalnu dekompoziciju:
$$\frac{1}{N(1-\frac{N}{K})} = \frac{1}{N\left(\frac{K-N}{K}\right)} = \frac{K}{N(K-N)}$$

Koristeći parcijalne frakcije:
$$\frac{K}{N(K-N)} = \frac{A}{N} + \frac{B}{K-N}$$

Rješavanjem za $A$ i $B$ dobivamo $A = 1$ i $B = 1$, pa:
$$\frac{K}{N(K-N)} = \frac{1}{N} + \frac{1}{K-N}$$

Sada možemo integrirati:
$$\int_{N_0}^{N(t)} \left(\frac{1}{N'} + \frac{1}{K-N'}\right) dN' = \int_0^t r \, dt'$$

$$\ln N(t) - \ln N_0 - \ln(K-N(t)) + \ln(K-N_0) = rt$$

$$\ln\left(\frac{N(t)}{N_0} \cdot \frac{K-N_0}{K-N(t)}\right) = rt$$

$$\ln\left(\frac{N(t)(K-N_0)}{N_0(K-N(t))}\right) = rt$$

Exponenciranjem i rješavanjem za $N(t)$:
$$\frac{N(t)(K-N_0)}{N_0(K-N(t))} = e^{rt}$$

$$N(t)(K-N_0) = N_0(K-N(t))e^{rt}$$

$$N(t)(K-N_0) = N_0 K e^{rt} - N_0 N(t) e^{rt}$$

$$N(t)[(K-N_0) + N_0 e^{rt}] = N_0 K e^{rt}$$

$$N(t) = \frac{N_0 K e^{rt}}{(K-N_0) + N_0 e^{rt}}$$

Dijeljenjem brojnika i nazivnika sa $N_0 e^{rt}$:
$$N(t) = \frac{K}{1 + \frac{K-N_0}{N_0}e^{-rt}}$$

što je standardni oblik logističke funkcije.

\textbf{Analiza logističke krivulje}

Logistička krivulja $N(t)$ ima karakteristična svojstva:

\begin{enumerate}
	\item \textbf{Početno ponašanje}: Za mala $t$, kada je $e^{-rt} \gg 1$:
	$$N(t) \approx \frac{K N_0}{K-N_0} e^{rt} = \frac{N_0}{1-\frac{N_0}{K}} e^{rt}$$
	
	što se aproximira eksponencijalnim rastom kada je $N_0 \ll K$.
	
	\item \textbf{Asimptotsko ponašanje}: Kada $t \to \infty$, $e^{-rt} \to 0$:
	$$\lim_{t \to \infty} N(t) = K$$
	
	\item \textbf{Infleksijska točka}: Maksimalna stopa rasta postiže se kada je $\frac{d^2N}{dt^2} = 0$.
	
	Izračunavanjem druge derivacije:
	$$\frac{d^2N}{dt^2} = \frac{d}{dt}\left[rN\left(1-\frac{N}{K}\right)\right] = r\frac{dN}{dt}\left(1-\frac{2N}{K}\right)$$
	
	Postavivši $\frac{d^2N}{dt^2} = 0$ i koristeći $\frac{dN}{dt} \neq 0$:
	$$1-\frac{2N}{K} = 0 \Rightarrow N = \frac{K}{2}$$
	
	Infleksijska točka nastupa kada populacija dosegne polovicu nosivosti staništa.
	
	\item \textbf{Maksimalna stopa rasta}: U infleksijskoj točki:
	$$\left.\frac{dN}{dt}\right|_{N=K/2} = r \cdot \frac{K}{2} \cdot \left(1-\frac{1}{2}\right) = \frac{rK}{4}$$
\end{enumerate}

\textbf{Diskretni eksponencijalni rast}

Za organizme s diskretnim reproduktivnim sezonama, kontinuirani modeli mogu biti neprikladni. Diskretni model eksponencijalnog rasta može se izraziti kao:
$$N_{t+1} = \lambda N_t$$

gdje je $\lambda$ konačna stopa rasta (finite rate of increase).

Povezanost između kontinuiranog i diskretnog parametra rasta je:
$$\lambda = e^r$$

odnosno:
$$r = \ln \lambda$$

Iterativno rješenje diskretnog modela daje:
$$N_t = \lambda^t N_0$$

što je diskretni ekvivalent kontinuiranog eksponencijalnog rasta.

\textbf{Diskretni logistički rast}

Diskretna verzija logističkog modela može se formulirati različito, ovisno o pretpostavkama o timing efektima density dependence. Najčešći pristupi su:

\begin{enumerate}
	\item \textbf{Ricker model}:
	$$N_{t+1} = N_t e^{r(1-\frac{N_t}{K})}$$
	
	Ovaj model pretpostavlja da density dependence djeluje eksponencijalno na stopu rasta.
	
	\item \textbf{Beverton-Holt model}:
	$$N_{t+1} = \frac{\lambda N_t}{1 + \frac{(\lambda-1)N_t}{K}}$$
	
	gdje je $\lambda > 1$ maksimalna stopa rasta pri malim abundancijama.
\end{enumerate}

\textbf{Analiza stabilnosti kontinuiranih sustava}

Za analizu stabilnosti logističkog modela, lineariziramo oko ravnotežne točke $N^* = K$. Neka je $n(t) = N(t) - K$ mala perturbacija oko ravnoteže:
$$\frac{dn}{dt} = \frac{d(N-K)}{dt} = \frac{dN}{dt} = r(n+K)\left(1-\frac{n+K}{K}\right)$$

$$= r(n+K)\left(-\frac{n}{K}\right) = -\frac{r}{K}(n+K)n = -\frac{r}{K}n^2 - rn$$

Za male perturbacije, zanemarujemo kvadratni član:
$$\frac{dn}{dt} \approx -rn$$

što ima rješenje $n(t) = n_0 e^{-rt}$. Budući da je $r > 0$, perturbacije eksponencijalno opadaju, što potvrđuje stabilnost ravnotežne točke $N^* = K$.

\textbf{Analiza stabilnosti diskretnih sustava}

Za Ricker model, ravnotežne točke nalazimo rješavanjem:
$$N^* = N^* e^{r(1-\frac{N^*}{K})}$$

Netrivijalna ravnotežna točka je $N^* = K$ (dobiva se iz $r(1-\frac{N^*}{K}) = 0$).

Za analizu lokalne stabilnosti, lineariziramo oko $N^* = K$:
$$f(N) = N e^{r(1-\frac{N}{K})}$$

$$f'(N) = e^{r(1-\frac{N}{K})} + N \cdot e^{r(1-\frac{N}{K})} \cdot \left(-\frac{r}{K}\right)$$

$$f'(K) = e^0 + K \cdot e^0 \cdot \left(-\frac{r}{K}\right) = 1 - r$$

Ravnotežna točka je:

\begin{itemize}
	\item \textbf{Stabilna} ako $|f'(K)| < 1$, tj. $|1-r| < 1$, što daje $0 < r < 2$
	\item \textbf{Nestabilna} ako $|f'(K)| > 1$, tj. $r > 2$ ili $r < 0$
\end{itemize}

\textbf{Bifurkacije i kaotično ponašanje}

Ricker model pokazuje bogato dinamičko ponašanje ovisno o vrijednosti parametra $r$:

\begin{enumerate}
	\item \textbf{Za $0 < r < 2$}: Monotonska konvergencija prema $K$
	\item \textbf{Za $r = 2$}: Granična stabilnost
	\item \textbf{Za $2 < r < 2.526...$}: Oscilirajuća konvergencija prema $K$
	\item \textbf{Za $r = 2.526...$}: Prvo udvostručenje periode (period-doubling bifurcation)
	\item \textbf{Za $2.526... < r < 2.692...$}: Ciklus periode 2
	\item \textbf{Za $r > 2.692...$}: Kaskada udvostručivanja perioda vodeći u kaos
\end{enumerate}

Kaotično ponašanje karakterizira pozitivni Lyapunovov eksponent:
$$\lambda_L = \lim_{T \to \infty} \frac{1}{T} \sum_{t=0}^{T-1} \ln|f'(N_t)|$$

Za Ricker model:
$$\lambda_L = \lim_{T \to \infty} \frac{1}{T} \sum_{t=0}^{T-1} \ln|1 - r(1-\frac{N_t}{K})|$$

Pozitivna vrijednost $\lambda_L$ indicira kaotično ponašanje s eksponencijalnim razilaženjima bliskih trajektorija.

\textbf{Beverton-Holt model i kontrastno ponašanje}

Za razliku od Ricker modela, Beverton-Holt model ne pokazuje kaotično ponašanje. Ravnotežna točka $N^* = K$ je globalno stabilna za sve $\lambda > 1$.

Linearizacija oko ravnoteže:
$$g(N) = \frac{\lambda N}{1 + \frac{(\lambda-1)N}{K}}$$

$$g'(N) = \frac{\lambda}{1 + \frac{(\lambda-1)N}{K}} - \frac{\lambda N \cdot \frac{\lambda-1}{K}}{(1 + \frac{(\lambda-1)N}{K})^2}$$

$$g'(K) = \frac{\lambda}{1 + (\lambda-1)} - \frac{\lambda K \cdot \frac{\lambda-1}{K}}{(1 + (\lambda-1))^2} = \frac{\lambda}{\lambda} - \frac{\lambda(\lambda-1)}{\lambda^2} = 1 - \frac{\lambda-1}{\lambda} = \frac{1}{\lambda}$$

Budući da je $\lambda > 1$, imamo $|g'(K)| = \frac{1}{\lambda} < 1$, što garantira lokalnu stabilnost.

\textbf{Usporedba kontinuiranih i diskretnih modela}

Kontinuirani i diskretni pristupi mogu davati kvalitativno različite rezultate:

\begin{enumerate}
	\item \textbf{Kontinuirani logistički model}: Uvijek monotonski konvergira prema nosivosti staništa
	\item \textbf{Diskretni modeli}: Mogu pokazivati oscilacije, periodičko ponašanje, ili kaos
\end{enumerate}

Ova razlika nastaje zbog različitog tretiranja vremenske skale demografskih procesa. Kontinuirani modeli pretpostavljaju trenutačne demografske odgovore, dok diskretni modeli mogu uhvatiti kašnjenja i prekoračenja (overshoots) karakteristične za organizme s diskretnim generacijama.

Izbor između kontinuiranih i diskretnih modela ovisi o:

\begin{itemize}
	\item \textbf{Biologiji organizma}: Preklapajuće vs. diskretne generacije
	\item \textbf{Vremenskoj skali procesa}: Brzi vs. spori demografski procesi  
	\item \textbf{Dostupnosti podataka}: Kontinuirani monitoring vs. godišnji surveyji
	\item \textbf{Ciljevima modeliranja}: Kvalitativno razumijevanje vs. kvantitativno predviđanje
\end{itemize}

Kombinacija teorijskih uvida iz oba pristupa omogućava dublje razumijevanje kompleksnosti populacijske dinamike i pomoć u donošenju informiranih odluka u ekološkom upravljanju i biologiji zaštite prirode.

\section{Vjerojatnost, statistika i stohastičnost}

\subsection{Uvod u stohastičnost u ekološkim sustavima}

Ekološki sustavi su inherentno nepredvidljivi zbog brojnih čimbenika koji utječu na populacije i zajednice. \textbf{Stohastičnost} se odnosi na slučajnu varijabilnost koja se javlja u ekološkim procesima i može se klasificirati u tri glavne kategorije:

\begin{enumerate}
	\item \textbf{Demografska stohastičnost} -- slučajne varijacije u rađanju, smrti i reprodukciji individualnih organizama
	\item \textbf{Okolišna stohastičnost} -- slučajne varijacije u uvjetima okoliša
	\item \textbf{Katastrofična stohastičnost} -- rijetki, ali ekstremni događaji
\end{enumerate}

\subsection{Osnovni koncepti vjerojatnosti}

\subsubsection{Vjerojatnosne distribucije}

Neka je $X$ slučajna varijabla koja opisuje neki ekološki parametar. Funkcija vjerojatnosne gustoće $f(x)$ mora zadovoljiti:

\begin{equation}
	\int_{-\infty}^{\infty} f(x) \, dx = 1
\end{equation}

Za diskretne slučajne varijable, funkcija vjerojatnosne mase $P(X = x_i)$ mora zadovoljiti:

\begin{equation}
	\sum_{i} P(X = x_i) = 1
\end{equation}

\subsubsection{Očekivana vrijednost i varijabilnost}

\textbf{Očekivana vrijednost} (matematičko očekivanje):
\begin{equation}
	E[X] = \mu = \int_{-\infty}^{\infty} x f(x) \, dx
\end{equation}

\textbf{Varijanca}:
\begin{equation}
	\text{Var}(X) = \sigma^2 = E[(X - \mu)^2] = E[X^2] - (E[X])^2
\end{equation}

\textbf{Standardna devijacija}:
\begin{equation}
	\sigma = \sqrt{\text{Var}(X)}
\end{equation}

\subsubsection{Koeficijent varijacije}

U ekologiji često koristimo \textbf{koeficijent varijacije} (CV) za mjerenje relativne varijabilnosti:

\begin{equation}
	CV = \frac{\sigma}{\mu}
\end{equation}

\subsection{Demografska stohastičnost}

\subsubsection{Binomni model preživljavanja}

Pretpostavimo da imamo populaciju od $N$ jedinki, gdje svaka jedinka ima vjerojatnost preživljavanja $s$. Broj preživjelih jedinki $S$ slijedi binomnu distribuciju:

\begin{equation}
	P(S = k) = \binom{N}{k} s^k (1-s)^{N-k}
\end{equation}

\textbf{Očekivana vrijednost}: $E[S] = Ns$\\
\textbf{Varijanca}: $\text{Var}(S) = Ns(1-s)$

\subsubsection{Poissonov model rođenja}

Ako je stopa rođenja $\lambda$ konstantna kroz vrijeme, broj rođenih jedinki u vremenu $t$ slijedi Poissonovu distribuciju:

\begin{equation}
	P(X = k) = \frac{(\lambda t)^k e^{-\lambda t}}{k!}
\end{equation}

\textbf{Očekivana vrijednost}: $E[X] = \lambda t$\\
\textbf{Varijanca}: $\text{Var}(X) = \lambda t$

\subsubsection{Utjecaj demografske stohastičnosti na male populacije}

Za malu populaciju veličine $N$, demografska stohastičnost može se aproksimirati normalnom distribucijom:

\begin{equation}
	N(t+1) \sim \mathcal{N}(\mu N(t), \sigma^2 N(t))
\end{equation}

gdje je $\mu$ očekivana stopa rasta po jedinki, a $\sigma^2$ varijanca demografskih procesa.

\textbf{Vjerojatnost izumiranja} za populaciju s početnom veličinom $N_0$ i negativnom stopom rasta $r < 0$:

\begin{equation}
	P(\text{izumiranje}) = \left(\frac{\sigma^2}{2|r|N_0}\right)^{2|r|/\sigma^2}
\end{equation}

\subsection{Okolišna stohastičnost}

\subsubsection{Model s okolišnom stohastičnošću}

Stopa rasta populacije varira stohastički ovisno o uvjetima okoliša:

\begin{equation}
	\frac{dN}{dt} = r(t) N(t)
\end{equation}

gdje je $r(t)$ stohastički proces. Često pretpostavljamo da je $r(t)$ bijeli šum:

\begin{equation}
	r(t) = \mu + \sigma \xi(t)
\end{equation}

gdje je $\xi(t)$ standardni bijeli šum s $E[\xi(t)] = 0$ i $E[\xi(t)\xi(s)] = \delta(t-s)$.

\subsubsection{Geometrijski Brownov pokret}

Rješenje stohastičke diferencijalne jednadžbe:

\begin{equation}
	dN = \mu N \, dt + \sigma N \, dW
\end{equation}

gdje je $dW$ Wienerov proces, daje:

\begin{equation}
	N(t) = N_0 \exp\left[\left(\mu - \frac{\sigma^2}{2}\right)t + \sigma W(t)\right]
\end{equation}

\textbf{Očekivana vrijednost}: $E[N(t)] = N_0 e^{\mu t}$\\
\textbf{Varijanca}: $\text{Var}(N(t)) = N_0^2 e^{2\mu t}(e^{\sigma^2 t} - 1)$

\subsubsection{Logistički model s okolišnom stohastičnošću}

\begin{equation}
	dN = rN\left(1 - \frac{N}{K}\right)dt + \sigma N \, dW
\end{equation}

\textbf{Stacionarna distribucija} (kada postoji) ima oblik:

\begin{equation}
	\pi(N) \propto N^{2r/\sigma^2 - 1} \left(1 - \frac{N}{K}\right)^{2r/\sigma^2 - 1} \exp\left(-\frac{2rN}{\sigma^2 K}\right)
\end{equation}

\subsection{Katastrofična stohastičnost}

\subsubsection{Poissonov model katastrofa}

Katastrofe se javljaju prema Poissonovom procesu s intenzitetom $\lambda$. Vjerojatnost da se dogodi $k$ katastrofa u vremenu $t$:

\begin{equation}
	P(N_t = k) = \frac{(\lambda t)^k e^{-\lambda t}}{k!}
\end{equation}

\subsubsection{Model s katastrofičnim smanjenjem}

Kombinacija eksponencijalnog rasta s povremenim katastrofičnim smanjenjem:

\begin{equation}
	N(t^+) = \alpha N(t^-)
\end{equation}

gdje je $\alpha \in (0,1)$ faktor preživljavanja nakon katastrofe.

\textbf{Srednja stopa rasta} u prisutnosti katastrofa:

\begin{equation}
	r_{\text{eff}} = r - \lambda(1 - E[\alpha])
\end{equation}

gdje je $E[\alpha]$ očekivana vrijednost faktora preživljavanja.

\subsection{Maksimalna vjerojatnost (Maximum Likelihood)}

\subsubsection{Likelihood funkcija}

Za skup nezavisnih opažanja $\mathbf{x} = (x_1, x_2, \ldots, x_n)$ iz distribucije s parametrima $\boldsymbol{\theta}$, likelihood funkcija je:

\begin{equation}
	L(\boldsymbol{\theta}) = \prod_{i=1}^n f(x_i | \boldsymbol{\theta})
\end{equation}

\subsubsection{Log-likelihood}

Često je lakše raditi s logaritmom likelihood funkcije:

\begin{equation}
	\ell(\boldsymbol{\theta}) = \log L(\boldsymbol{\theta}) = \sum_{i=1}^n \log f(x_i | \boldsymbol{\theta})
\end{equation}

\subsubsection{Maksimalna vjerojatnost procjena (MLE)}

Procjenjeni parametri su oni koji maksimiziraju likelihood funkciju:

\begin{equation}
	\hat{\boldsymbol{\theta}} = \arg\max_{\boldsymbol{\theta}} L(\boldsymbol{\theta})
\end{equation}

\textbf{Uvjeti prvog reda}:
\begin{equation}
	\frac{\partial \ell(\boldsymbol{\theta})}{\partial \theta_j} = 0, \quad j = 1, 2, \ldots, p
\end{equation}

\subsubsection{Primjer: Procjena stope rasta}

Za eksponencijalni model populacije $N(t) = N_0 e^{rt}$ s Gaussovim šumom:

\begin{equation}
	N_{\text{obs}}(t_i) = N_0 e^{rt_i} + \epsilon_i
\end{equation}

gdje je $\epsilon_i \sim \mathcal{N}(0, \sigma^2)$.

\textbf{Log-likelihood funkcija}:
\begin{equation}
	\ell(r, N_0, \sigma^2) = -\frac{n}{2}\log(2\pi\sigma^2) - \frac{1}{2\sigma^2}\sum_{i=1}^n (N_{\text{obs}}(t_i) - N_0 e^{rt_i})^2
\end{equation}

\subsubsection{Asimptotska svojstva MLE}

Za velike uzorke, MLE je:
\begin{itemize}
	\item \textbf{Konzistentan}: $\hat{\boldsymbol{\theta}} \to \boldsymbol{\theta}_0$ kada $n \to \infty$
	\item \textbf{Asimptotski normalan}: $\sqrt{n}(\hat{\boldsymbol{\theta}} - \boldsymbol{\theta}_0) \to \mathcal{N}(0, \mathbf{I}^{-1})$
	\item \textbf{Asimptotski efikasan}: postiže Cramér-Rao donju granicu
\end{itemize}

\subsubsection{Fisherova informacijska matrica}

\begin{equation}
	\mathbf{I}_{jk} = -E\left[\frac{\partial^2 \ell(\boldsymbol{\theta})}{\partial \theta_j \partial \theta_k}\right]
\end{equation}

\textbf{Standardne greške} procjena: $SE(\hat{\theta}_j) = \sqrt{[\mathbf{I}^{-1}]_{jj}}$

\subsection{Bayesovska statistika}

\subsubsection{Bayesov teorem}

\begin{equation}
	P(\boldsymbol{\theta} | \mathbf{x}) = \frac{P(\mathbf{x} | \boldsymbol{\theta}) P(\boldsymbol{\theta})}{P(\mathbf{x})}
\end{equation}

gdje je:
\begin{itemize}
	\item $P(\boldsymbol{\theta} | \mathbf{x})$ -- \textbf{posteriorni} razočetak
	\item $P(\mathbf{x} | \boldsymbol{\theta})$ -- \textbf{likelihood}
	\item $P(\boldsymbol{\theta})$ -- \textbf{priorni} razočetak
	\item $P(\mathbf{x})$ -- \textbf{marginalna vjerojatnost} (normalizacijska konstanta)
\end{itemize}

\subsubsection{Konjugirani priorovi}

Za Poissonovu distribuciju s Gamma priorom:

\textbf{Prior}: $\theta \sim \text{Gamma}(\alpha, \beta)$\\
\textbf{Likelihood}: $x_i | \theta \sim \text{Poisson}(\theta)$\\
\textbf{Posterior}: $\theta | \mathbf{x} \sim \text{Gamma}\left(\alpha + \sum x_i, \beta + n\right)$

\subsubsection{Kredibilni intervali}

95\% kredibilni interval za parametar $\theta$:

\begin{equation}
	P(\theta_L \leq \theta \leq \theta_U | \mathbf{x}) = 0.95
\end{equation}

\subsubsection{Markov Chain Monte Carlo (MCMC)}

\paragraph{Metropolis-Hastings algoritam}

\begin{enumerate}
	\item Započni s početnom vrijednošću $\theta^{(0)}$
	\item Za $t = 1, 2, \ldots$:
	\begin{itemize}
		\item Predloži novu vrijednost: $\theta^* \sim q(\theta^* | \theta^{(t-1)})$
		\item Računaj omjer: 
		\begin{equation}
			\alpha = \min\left(1, \frac{P(\theta^* | \mathbf{x}) q(\theta^{(t-1)} | \theta^*)}{P(\theta^{(t-1)} | \mathbf{x}) q(\theta^* | \theta^{(t-1)})}\right)
		\end{equation}
		\item Postavi: 
		\begin{equation}
			\theta^{(t)} = \begin{cases} 
				\theta^* & \text{s vjerojatnosti } \alpha \\ 
				\theta^{(t-1)} & \text{inače} 
			\end{cases}
		\end{equation}
	\end{itemize}
\end{enumerate}

\paragraph{Gibbs sampling}

Za multidimenzionalne probleme, uzorkuj svaku komponentu uvjetno na ostale:

\begin{equation}
	\theta_j^{(t)} \sim P(\theta_j | \theta_1^{(t)}, \ldots, \theta_{j-1}^{(t)}, \theta_{j+1}^{(t-1)}, \ldots, \theta_p^{(t-1)}, \mathbf{x})
\end{equation}

\subsection{Primjeri u ekološkom modeliranju}

\subsubsection{Procjena parametara logističkog modela}

Za logistički model:
\begin{equation}
	\frac{dN}{dt} = rN\left(1 - \frac{N}{K}\right)
\end{equation}

s opažanjima $N_1, N_2, \ldots, N_n$ u vremenima $t_1, t_2, \ldots, t_n$.

\textbf{Likelihood za kontinuirani model s Gaussovim šumom}:
\begin{equation}
	L(r, K, \sigma^2) = \prod_{i=1}^n \frac{1}{\sqrt{2\pi\sigma^2}} \exp\left(-\frac{(N_i - N(t_i; r, K))^2}{2\sigma^2}\right)
\end{equation}

\subsubsection{Analiza preživljavanja u ekologiji}

Za Kaplan-Meier procjenitelj funkcije preživljavanja:

\begin{equation}
	\hat{S}(t) = \prod_{t_i \leq t} \left(1 - \frac{d_i}{n_i}\right)
\end{equation}

gdje je $d_i$ broj smrti u vremenu $t_i$, a $n_i$ broj jedinki pod rizikom.

\subsubsection{Bayesova analiza metapopulacije}

Za Levinsov metapopulacijski model:
\begin{equation}
	\frac{dp}{dt} = cp(1-p) - ep
\end{equation}

s priorom na parametrima:
\begin{itemize}
	\item $c \sim \text{Gamma}(\alpha_c, \beta_c)$ (stopa kolonizacije)
	\item $e \sim \text{Gamma}(\alpha_e, \beta_e)$ (stopa izumiranja)
\end{itemize}

\textbf{Posteriorni razočetak} se računa numerički pomoću MCMC metoda.

\subsection{Model selekcija i usporedba}

\subsubsection{Akaike Information Criterion (AIC)}

\begin{equation}
	AIC = 2k - 2\ell(\hat{\boldsymbol{\theta}})
\end{equation}

gdje je $k$ broj parametara, a $\ell(\hat{\boldsymbol{\theta}})$ maksimalna log-likelihood vrijednost.

\subsubsection{Bayesian Information Criterion (BIC)}

\begin{equation}
	BIC = k\log(n) - 2\ell(\hat{\boldsymbol{\theta}})
\end{equation}

\subsubsection{Bayes faktori}

Za usporedbu modela $M_1$ i $M_2$:

\begin{equation}
	BF_{12} = \frac{P(\mathbf{x} | M_1)}{P(\mathbf{x} | M_2)}
\end{equation}

\textbf{Interpretacija}:
\begin{itemize}
	\item $BF_{12} > 10$: snažna podrška za $M_1$
	\item $BF_{12} > 100$: odlučujuća podrška za $M_1$
\end{itemize}

\subsection{Propagacija nesigurnosti}

\subsubsection{Monte Carlo simulacija}

Za model $y = f(\mathbf{x})$ gdje je $\mathbf{x}$ vektor parametara s poznatim distribucijama:

\begin{enumerate}
	\item Generiraj $N$ uzoraka $\mathbf{x}^{(i)}$ iz zajedničke distribucije
	\item Računaj $y^{(i)} = f(\mathbf{x}^{(i)})$ za $i = 1, \ldots, N$
	\item Analiziraj empirijsku distribuciju $\{y^{(i)}\}$
\end{enumerate}

\subsubsection{Delta metoda}

Za funkciju $g(\boldsymbol{\theta})$ i MLE $\hat{\boldsymbol{\theta}}$:

\begin{equation}
	\sqrt{n}(g(\hat{\boldsymbol{\theta}}) - g(\boldsymbol{\theta}_0)) \to \mathcal{N}(0, \nabla g^T \mathbf{I}^{-1} \nabla g)
\end{equation}

gdje je $\nabla g$ gradijent funkcije $g$.

\subsubsection{Bootstrapping}

\begin{enumerate}
	\item Uzorkuj s vraćanjem iz originalnih podataka
	\item Računaj statistiku od interesa za svaki bootstrap uzorak
	\item Konstruiraj empirijsku distribuciju bootstrap statistika
\end{enumerate}

\textbf{Bias-corrected and accelerated (BCa) bootstrap interval}:
\begin{equation}
	\left[\hat{F}^{-1}(\alpha_1), \hat{F}^{-1}(\alpha_2)\right]
\end{equation}

gdje su $\alpha_1$ i $\alpha_2$ korigirane vjerojatnosti.

\subsection{Ključne točke za praktičnu primjenu}

\begin{enumerate}
	\item Uvijek testiraj pretpostavke o distribucijama
	\item Koristi dijagnostičke grafove za provjeru modela
	\item Kvantificiraj i prenesi nesigurnost u konačne rezultate
	\item Kombiniraj različite pristupe za robusne zaključke
	\item Dokumentiraj sve pretpostavke i ograničenja analíze
\end{enumerate}


% Potrebni paketi za ovu sekciju:
% \usepackage{listings}
% \usepackage{color}
% \usepackage{xcolor}

% Potrebni paketi za ovu sekciju:
% \usepackage{listings}
% \usepackage{color}
% \usepackage{xcolor}

% Potrebni paketi za ovu sekciju:
% \usepackage{listings}
% \usepackage{color}
% \usepackage{xcolor}

\section{Računalni alati}

\subsection{Uvod u računalne alate za ekološko modeliranje}

Moderne metode ekološkog modeliranja uvelike ovise o računalnim alatima koji omogućavaju implementaciju složenih matematičkih modela, analizu velikih skupova podataka i vizualizaciju rezultata. Četiri glavna alata koji dominiraju u ekološkom modeliranju su R, Python, MATLAB i NetLogo. Svaki od njih ima specifične prednosti ovisno o vrsti modeliranja i analitičkim potrebama.

\subsection{R programski jezik i okruženje}

\subsubsection{Uvod u R}

R je \textit{open-source} programski jezik i okruženje posebno dizajnirano za statističku analizu i grafiku. Razvila ga je R Core Team, a temelji se na S programskom jeziku. R je postao dominantan alat u ekološkoj statistici zbog svoje fleksibilnosti, opsežne biblioteke paketa i snažne zajednice korisnika.

\subsubsection{Prednosti R-a za ekološko modeliranje}

\begin{itemize}
	\item \textbf{Specijalizirani paketi}: Preko 18,000 paketa na CRAN repozitoriju
	\item \textbf{Statistička snaga}: Ugrađene funkcije za naprednu statistiku
	\item \textbf{Grafike}: Iznimno fleksibilna grafička mogućnosti
	\item \textbf{Reproducibilnost}: R Markdown i integracija s Git
	\item \textbf{Zajednica}: Aktivna zajednica ekologa i statističara
\end{itemize}

\subsubsection{Ključni paketi za ekološko modeliranje}

\paragraph{Osnovni statistički paketi}

\begin{itemize}
	\item \texttt{stats}: Osnovne statistike i linearna regresija
	\item \texttt{nlme}: Nelinearna mješovita modeli
	\item \texttt{lme4}: Linearna mješoviti modeli s random efektima
	\item \texttt{mgcv}: Generalizirani aditivni modeli (GAM)
	\item \texttt{MASS}: Moderna primjenjena statistika
\end{itemize}

\paragraph{Populacijska dinamika i demografija}

\begin{itemize}
	\item \texttt{popbio}: Analiza populacijskih matrica
	\item \texttt{demogR}: Demografska analiza
	\item \texttt{Rcompadre}: Baza demografskih podataka
	\item \texttt{lefko3}: Analize životnog ciklusa
	\item \texttt{IPMpack}: Integral Projection Models
\end{itemize}

\paragraph{Prostorno ekološko modeliranje}

\begin{itemize}
	\item \texttt{dismo}: Modeliranje distribucije vrsta
	\item \texttt{raster}: Manipulacija raster podataka
	\item \texttt{sp}: Prostorni podaci
	\item \texttt{sf}: Simple Features za prostorne podatke
	\item \texttt{rgdal}: Geospatial Data Abstraction Library
	\item \texttt{biomod2}: Ansambl modeliranje bioraznolikosti
\end{itemize}

\paragraph{Ekološke mreže i zajednice}

\begin{itemize}
	\item \texttt{vegan}: Analiza ekoloških zajednica
	\item \texttt{igraph}: Analiza mreža
	\item \texttt{bipartite}: Bipartitne ekološke mreže
	\item \texttt{foodweb}: Hranidbene mreže
	\item \texttt{NetIndices}: Mrežni indeksi
\end{itemize}

\paragraph{Bayesovska analiza}

\begin{itemize}
	\item \texttt{MCMCglmm}: MCMC za generalizirane linearne modele
	\item \texttt{rstanarm}: Bayesova regresija putem Stan-a
	\item \texttt{brms}: Bayesova regresija s Stan backend
	\item \texttt{R2jags}: Interface za JAGS
	\item \texttt{nimble}: MCMC algoritmi
\end{itemize}

\subsubsection{Primjer koda: Leslie matrica u R-u}

\begin{lstlisting}[language=R, caption=Implementacija Leslie matrice]
	# Definiranje Leslie matrice
	# F - fertility rates, S - survival rates
	create_leslie <- function(F, S) {
		n <- length(F)
		L <- matrix(0, nrow = n, ncol = n)
		
		# Prvi red: fertility rates
		L[1, ] <- F
		
		# Subdiagonala: survival rates
		for(i in 2:n) {
			L[i, i-1] <- S[i-1]
		}
		
		return(L)
	}
	
	# Parametri za primjer
	fertility <- c(0, 0.3, 1.2, 1.5, 0.8)
	survival <- c(0.6, 0.7, 0.8, 0.3)
	
	# Kreiranje matrice
	L <- create_leslie(fertility, survival)
	
	# Početna populacija
	n0 <- c(100, 80, 60, 40, 20)
	
	# Simulacija kroz vrijeme
	simulate_population <- function(L, n0, t_max = 50) {
		n_age <- length(n0)
		N <- matrix(NA, nrow = n_age, ncol = t_max + 1)
		N[, 1] <- n0
		
		for(t in 1:t_max) {
			N[, t + 1] <- L %*% N[, t]
		}
		
		return(N)
	}
	
	# Pokretanje simulacije
	results <- simulate_population(L, n0, 50)
	
	# Računanje ukupne populacije
	total_pop <- colSums(results)
	
	# Analiza asimptotskog ponašanja
	eigen_analysis <- eigen(L)
	lambda <- Re(eigen_analysis$values[1])
	stable_age_dist <- Re(eigen_analysis$vectors[, 1])
	stable_age_dist <- stable_age_dist / sum(stable_age_dist)
	
	cat("Asimptotska stopa rasta (lambda):", lambda, "\n")
	cat("Stabilna dobna distribucija:", stable_age_dist, "\n")
\end{lstlisting}

\subsubsection{Upravljanje paketima u R-u}

\paragraph{CRAN repozitorij}

\begin{lstlisting}[language=R, caption=Instalacija paketa iz CRAN-a]
	# Instalacija pojedinačnog paketa
	install.packages("dismo")
	
	# Instalacija više paketa odjednom
	packages <- c("vegan", "mgcv", "lme4", "ggplot2")
	install.packages(packages)
	
	# Provjera dostupnih verzija
	available.packages()[c("dismo", "vegan"), ]
\end{lstlisting}

\paragraph{GitHub repozitoriji}

\begin{lstlisting}[language=R, caption=Instalacija iz GitHub-a]
	# Instalacija devtools
	install.packages("devtools")
	
	# Instalacija paketa iz GitHub-a
	devtools::install_github("ropensci/spocc")
	devtools::install_github("user/package", ref = "branch_name")
\end{lstlisting}

\paragraph{Upravljanje verzijama s renv}

\begin{lstlisting}[language=R, caption=Reproducibilno okruženje s renv]
	# Inicijalizacija projekta
	renv::init()
	
	# Snapshot trenutnog stanja
	renv::snapshot()
	
	# Vraćanje na prethodno stanje
	renv::restore()
	
	# Ažuriranje paketa
	renv::update()
\end{lstlisting}

\subsection{Python programski jezik}

\subsubsection{Uvod u Python za ekologiju}

Python je \textit{general-purpose} programski jezik koji je postao iznimno popularan u znanstvenoj zajednici zbog svoje čitljivosti, fleksibilnosti i opsežnog ekosustava znanstvenih biblioteka. Za ekološko modeliranje, Python nudi prednosti u integraciji s drugim sustavima, obradi velikih podataka i strojnom učenju.

\subsubsection{Ključne biblioteke za ekološko modeliranje}

\paragraph{Znanstvene računalne biblioteke}

\begin{itemize}
	\item \texttt{numpy}: Numeričko računanje i rad s nizovima
	\item \texttt{scipy}: Znanstveni algoritmi i optimizacija
	\item \texttt{pandas}: Manipulacija i analiza podataka
	\item \texttt{matplotlib}: Osnovna vizualizacija
	\item \texttt{seaborn}: Statistička vizualizacija
	\item \texttt{plotly}: Interaktivna vizualizacija
\end{itemize}

\paragraph{Strojno učenje i statistika}

\begin{itemize}
	\item \texttt{scikit-learn}: Opće potrebe strojnog učenja
	\item \texttt{statsmodels}: Statistički modeli
	\item \texttt{pymc}: Bayesovska statistika
	\item \texttt{tensorflow}/\texttt{pytorch}: Duboko učenje
	\item \texttt{xgboost}: Gradient boosting
\end{itemize}

\paragraph{Prostorni podaci i GIS}

\begin{itemize}
	\item \texttt{geopandas}: Prostorni podaci
	\item \texttt{rasterio}: Raster podaci
	\item \texttt{shapely}: Geometrijske operacije
	\item \texttt{fiona}: Čitanje/pisanje prostornih podataka
	\item \texttt{folium}: Interaktivne karte
\end{itemize}

\paragraph{Specijalizirane ekološke biblioteke}

\begin{itemize}
	\item \texttt{ecohydro}: Ekohidroški modeli
	\item \texttt{scikit-bio}: Bioinformatika
	\item \texttt{biopython}: Molekularna biologija
	\item \texttt{ete3}: Filogenetska analiza
\end{itemize}

\subsubsection{Primjer koda: Lotka-Volterra model u Pythonu}

\begin{lstlisting}[language=Python, caption=Lotka-Volterra predator-prey model (dio 1)]
	import numpy as np
	import matplotlib.pyplot as plt
	from scipy.integrate import solve_ivp
	
	class LotkaVolterra:
	"""
	Klasa za Lotka-Volterra predator-prey model
	"""
	
	def __init__(self, r, a, e, m):
	"""
	Parametri:
	r - stopa rasta plijena
	a - stopa napada
	e - efikasnost konverzije
	m - stopa smrti predatora
	"""
	self.r = r  # intrinsic growth rate of prey
	self.a = a  # attack rate
	self.e = e  # conversion efficiency
	self.m = m  # predator death rate
	
	def equations(self, t, z):
	"""
	Lotka-Volterra diferencijalne jednadžbe
	"""
	N, P = z  # prey, predator
	
	dN_dt = self.r * N - self.a * N * P
	dP_dt = self.e * self.a * N * P - self.m * P
	
	return [dN_dt, dP_dt]
	
	def simulate(self, initial_conditions, t_span, t_eval=None):
	"""
	Simulacija modela
	"""
	if t_eval is None:
	t_eval = np.linspace(t_span[0], t_span[1], 1000)
	
	solution = solve_ivp(
	self.equations, 
	t_span, 
	initial_conditions,
	t_eval=t_eval,
	dense_output=True
	)
	
	return solution
\end{lstlisting}

\subsubsection{Upravljanje okruženjima u Pythonu}

\paragraph{Conda upravljanje paketima}

\begin{lstlisting}[language=bash, caption=Conda environment management]
	# Kreiranje novog okruženja
	conda create -n ecology python=3.9
	
	# Aktivacija okruženja
	conda activate ecology
	
	# Instalacija paketa
	conda install numpy scipy pandas matplotlib
	conda install -c conda-forge geopandas
	
	# Export okruženja
	conda env export > environment.yml
	
	# Kreiranje okruženja iz filea
	conda env create -f environment.yml
	
	# Lista svih okruženja
	conda env list
\end{lstlisting}

\paragraph{pip i virtualenv}

\begin{lstlisting}[language=bash, caption=pip package management]
	# Kreiranje virtualnog okruženja
	python -m venv ecology_env
	
	# Aktivacija (Linux/Mac)
	source ecology_env/bin/activate
	
	# Aktivacija (Windows)
	ecology_env\Scripts\activate
	
	# Instalacija paketa
	pip install numpy scipy pandas matplotlib
	
	# Generiranje requirements.txt
	pip freeze > requirements.txt
	
	# Instalacija iz requirements.txt
	pip install -r requirements.txt
\end{lstlisting}

\subsection{MATLAB}

\subsubsection{Uvod u MATLAB za ekološko modeliranje}

MATLAB je komercijalni programski jezik i okruženje optimizirano za numeričko računanje i matrične operacije. Iako je komercijalan, MATLAB nudi iznimno snažne alate za simulaciju dinamičkih sustava, numeričku optimizaciju i vizualizaciju, što ga čini popularnim u inženjerskim pristupima ekološkom modeliranju.

\subsubsection{Prednosti MATLAB-a}

\begin{itemize}
	\item \textbf{Numerička stabilnost}: Visoko optimizirani algoritmi
	\item \textbf{Simulink}: Grafičko modeliranje dinamičkih sustava
	\item \textbf{Toolboxovi}: Specijalizirani alati za različite domene
	\item \textbf{Paralelno računanje}: Ugrađena podrška za GPU i klaster računanje
	\item \textbf{Integracija}: Povezivanje s drugim sustavima i hardware
\end{itemize}

\subsubsection{Relevantni toolboxovi za ekologiju}

\begin{itemize}
	\item \textbf{Statistics and Machine Learning Toolbox}: Statistička analiza
	\item \textbf{Optimization Toolbox}: Numerička optimizacija
	\item \textbf{Parallel Computing Toolbox}: Paralelno izvršavanje
	\item \textbf{Mapping Toolbox}: GIS i kartografija
	\item \textbf{Image Processing Toolbox}: Analiza slika (daljinska istraživanja)
	\item \textbf{Global Optimization Toolbox}: Globalna optimizacija
\end{itemize}

\subsection{GNU Octave}

\subsubsection{Uvod u GNU Octave}

GNU Octave je besplatna, open-source alternativa MATLAB-u s visokom sintaksnom kompatibilnošću. Razvijan od 1988. godine, Octave omogućava izvršavanje većine MATLAB koda bez promjena, što ga čini idealnim za prelazak s komercijalnih na open-source alate ili za institucije s ograničenim budgetom.

\subsubsection{Prednosti Octave-a za ekološko modeliranje}

\begin{itemize}
	\item \textbf{MATLAB kompatibilnost}: 95\%+ kompatibilnost sintakse
	\item \textbf{Besplatnost}: Potpuno besplatan za akademsku i komercijalnu upotrebu
	\item \textbf{Cross-platform}: Dostupan za Windows, Mac, Linux
	\item \textbf{Aktivna zajednica}: Redovita ažuriranja i podrška
	\item \textbf{Proširivost}: Mogućnost dodavanja C++ funkcija
\end{itemize}

\subsubsection{Ključni paketi za ekološko modeliranje}

\paragraph{Osnovni paketi}

\begin{itemize}
	\item \texttt{statistics}: Statistička analiza i distribucije
	\item \texttt{optim}: Optimizacija i fitting algoritmi
	\item \texttt{signal}: Obrada signala i filtriranje
	\item \texttt{image}: Osnovna obrada slika
	\item \texttt{io}: Input/output operacije za različite formate
\end{itemize}

\paragraph{Numeričko rješavanje}

\begin{itemize}
	\item \texttt{ode}: Dodatni rješavači diferencijalnih jednadžbi
	\item \texttt{control}: Teorija upravljanja i analiza sustava
	\item \texttt{symbolic}: Simboličko računanje
	\item \texttt{parallel}: Paralelno izvršavanje
\end{itemize}

\subsubsection{Primjer koda: Populacijski model u Octave}

\begin{lstlisting}[language=Matlab, caption=Stohastički populacijski model u Octave]
	function stochastic_population_model()
	% Stohastički populacijski model s okolišnim varijacijama
	
	% Parametri
	params = struct();
	params.r_mean = 0.1;        % srednja stopa rasta
	params.r_std = 0.05;        % varijabilnost stope rasta
	params.K = 1000;            % nosivost staništa
	params.sigma_demographic = 0.1;  % demografska stohastičnost
	params.dt = 0.1;            % vremenski korak
	params.t_max = 50;          % maksimalno vrijeme
	params.n_simulations = 100; % broj simulacija
	
	% Vremenska mreža
	t = 0:params.dt:params.t_max;
	n_steps = length(t);
	
	% Matrica za pohranu rezultata
	N = zeros(params.n_simulations, n_steps);
	
	% Početna populacija
	N0 = 50;
	
	% Monte Carlo simulacije
	for sim = 1:params.n_simulations
	N(sim, 1) = N0;
	
	for i = 2:n_steps
	% Trenutna populacija
	N_current = N(sim, i-1);
	
	% Okolišna stohastičnost (varijacija u r)
	r_current = params.r_mean + params.r_std * randn();
	
	% Deterministički dio
	dN_det = r_current * N_current * (1 - N_current / params.K);
	
	% Demografska stohastičnost
	if N_current > 0
	dN_stoch = params.sigma_demographic * sqrt(N_current) * randn();
	else
	dN_stoch = 0;
	end
	
	% Ažuriranje populacije
	N(sim, i) = max(0, N_current + (dN_det + dN_stoch) * params.dt);
	
	% Provjera izumiranja
	if N(sim, i) < 1
	N(sim, i:end) = 0;
	break;
	end
	end
	end
	
	% Analiza rezultata
	analyze_results(t, N, params);
	
	% Vizualizacija
	visualize_results(t, N, params);
	end
	
	function analyze_results(t, N, params)
	% Osnovne statistike
	final_populations = N(:, end);
	extinctions = sum(final_populations == 0);
	extinction_probability = extinctions / params.n_simulations;
	
	fprintf('Analiza stohastičkog populacijskog modela:\n');
	fprintf('Broj simulacija: %d\n', params.n_simulations);
	fprintf('Vjerojatnost izumiranja: %.3f\n', extinction_probability);
	
	% Statistike za preživjele populacije
	survivors = final_populations(final_populations > 0);
	if ~isempty(survivors)
	fprintf('Srednja konačna populacija (preživjeli): %.2f ± %.2f\n', ...
	mean(survivors), std(survivors));
	fprintf('Raspon konačnih populacija: %.1f - %.1f\n', ...
	min(survivors), max(survivors));
	end
	
	% Vrijeme do izumiranja
	extinction_times = zeros(extinctions, 1);
	extinct_count = 0;
	
	for sim = 1:params.n_simulations
	extinct_indices = find(N(sim, :) == 0, 1);
	if ~isempty(extinct_indices)
	extinct_count = extinct_count + 1;
	extinction_times(extinct_count) = t(extinct_indices);
	end
	end
	
	if extinct_count > 0
	fprintf('Srednje vrijeme do izumiranja: %.2f ± %.2f\n', ...
	mean(extinction_times), std(extinction_times));
	end
	end
	
	function visualize_results(t, N, params)
	% Kreiranje slika
	figure('Position', [100, 100, 1200, 600]);
	
	% Subplot 1: Vremenske serije
	subplot(2, 3, 1);
	plot(t, N(1:min(20, size(N,1)), :)', 'Color', [0.7, 0.7, 0.7]);
	hold on;
	plot(t, mean(N, 1), 'r-', 'LineWidth', 2);
	plot(t, quantile(N, 0.05, 1), 'b--', 'LineWidth', 1);
	plot(t, quantile(N, 0.95, 1), 'b--', 'LineWidth', 1);
	xlabel('Vrijeme');
	ylabel('Veličina populacije');
	title('Stohastičke trajektorije');
	legend('Simulacije', 'Prosjek', '5-95% kvantili', 'Location', 'best');
	grid on;
	
	% Subplot 2: Distribucija konačnih populacija
	subplot(2, 3, 2);
	final_pops = N(:, end);
	final_pops_nonzero = final_pops(final_pops > 0);
	
	if ~isempty(final_pops_nonzero)
	histogram(final_pops_nonzero, 20);
	xlabel('Konačna veličina populacije');
	ylabel('Frekvencija');
	title('Distribucija konačnih populacija');
	grid on;
	end
	
	% Subplot 3: Vjerojatnost preživljavanja kroz vrijeme
	subplot(2, 3, 3);
	survival_prob = zeros(size(t));
	for i = 1:length(t)
	survival_prob(i) = sum(N(:, i) > 0) / params.n_simulations;
	end
	
	plot(t, survival_prob, 'g-', 'LineWidth', 2);
	xlabel('Vrijeme');
	ylabel('Vjerojatnost preživljavanja');
	title('Krivulja preživljavanja');
	grid on;
	ylim([0, 1]);
	
	% Subplot 4: Varijabilnost kroz vrijeme
	subplot(2, 3, 4);
	cv_through_time = std(N, 1) ./ mean(N, 1);
	cv_through_time(isnan(cv_through_time)) = 0;
	
	plot(t, cv_through_time, 'm-', 'LineWidth', 2);
	xlabel('Vrijeme');
	ylabel('Koeficijent varijacije');
	title('Varijabilnost populacije');
	grid on;
	
	% Subplot 5: Fazni dijagram (N vs dN/dt)
	subplot(2, 3, 5);
	% Računanje aproksimativnog dN/dt
	N_sample = N(1, :);  % Uzmi prvu simulaciju kao primjer
	dN_dt = diff(N_sample) / params.dt;
	N_mid = N_sample(1:end-1);
	
	plot(N_mid, dN_dt, 'ko', 'MarkerSize', 3);
	xlabel('N');
	ylabel('dN/dt');
	title('Fazni dijagram');
	grid on;
	
	% Dodaj teorijsku krivulju
	N_theory = 0:params.K/100:params.K;
	dN_theory = params.r_mean * N_theory .* (1 - N_theory / params.K);
	hold on;
	plot(N_theory, dN_theory, 'r-', 'LineWidth', 2);
	legend('Simulacija', 'Teorijski', 'Location', 'best');
	
	% Subplot 6: Autocorrelation analiza
	subplot(2, 3, 6);
	% Autocorrelation srednje populacije
	mean_pop = mean(N, 1);
	mean_pop_detrended = detrend(mean_pop);
	
	% Izračunaj autokorelaciju
	max_lag = min(50, floor(length(mean_pop_detrended)/4));
	autocorr_values = zeros(max_lag+1, 1);
	
	for lag = 0:max_lag
	if lag == 0
	autocorr_values(lag+1) = 1;
	else
	x1 = mean_pop_detrended(1:end-lag);
	x2 = mean_pop_detrended(lag+1:end);
	autocorr_values(lag+1) = corr(x1', x2');
	end
	end
	
	lags = 0:max_lag;
	plot(lags * params.dt, autocorr_values, 'b-o', 'LineWidth', 1.5);
	xlabel('Lag (vrijeme)');
	ylabel('Autokorelacija');
	title('Vremenska autokorelacija');
	grid on;
	
	% Dodaj liniju na y=0
	hold on;
	plot([0, max(lags) * params.dt], [0, 0], 'k--');
	end
	
	% Pokretanje simulacije
	stochastic_population_model();
\end{lstlisting}

\subsubsection{Instalacija i upravljanje paketima}

\paragraph{Instalacija Octave-a}

\begin{lstlisting}[language=bash, caption=Instalacija GNU Octave]
	# Ubuntu/Debian
	sudo apt-get install octave octave-doc
	
	# MacOS (Homebrew)
	brew install octave
	
	# Windows - download from https://www.gnu.org/software/octave/
	
	# Fedora/CentOS
	sudo dnf install octave
\end{lstlisting}

\paragraph{Upravljanje paketima}

\begin{lstlisting}[language=Matlab, caption=Octave package management]
	% Lista dostupnih paketa
	pkg list
	
	% Instalacija paketa
	pkg install -forge statistics
	pkg install -forge optim
	pkg install -forge signal
	
	% Učitavanje paketa
	pkg load statistics
	
	% Ažuriranje svih paketa
	pkg update
	
	% Uklanjanje paketa
	pkg uninstall statistics
	
	% Provjera statusa paketa
	pkg describe statistics
\end{lstlisting}

\subsection{Python Scientific Computing Stack}

\subsubsection{SciPy ekosustav za ekološko modeliranje}

SciPy ekosustav predstavlja kolekciju Python biblioteka optimiziranih za znanstveno računanje. Ovaj stack čini temelj za većinu znanstvenih Python aplikacija i nudi snažnu alternativu komercijskim sustavima poput MATLAB-a.

\subsubsection{Ključne komponente SciPy stack-a}

\paragraph{NumPy - Numerička osnova}

\begin{itemize}
	\item \textbf{N-dimenzijski nizovi}: Efikasne matrične operacije
	\item \textbf{Broadcasting}: Operacije na nizovima različitih dimenzija
	\item \textbf{Linear algebra}: BLAS/LAPACK optimizacija
	\item \textbf{Random sampling}: Pseudoslučajni broj generatori
	\item \textbf{Fourier transforms}: FFT implementacije
\end{itemize}

\paragraph{SciPy - Znanstveni algoritmi}

\begin{itemize}
	\item \textbf{scipy.integrate}: ODE/PDE rješavači
	\item \textbf{scipy.optimize}: Nelinearna optimizacija
	\item \textbf{scipy.stats}: Statistička distribucije i testovi
	\item \textbf{scipy.interpolate}: Interpolacija i aproksimacija
	\item \textbf{scipy.signal}: Obrada signala
\end{itemize}

\paragraph{Matplotlib - Vizualizacija}

\begin{itemize}
	\item \textbf{2D plotting}: Line plots, scatter plots, heatmaps
	\item \textbf{3D visualization}: Surface plots, volumetric rendering
	\item \textbf{Animation}: Dinamičke vizualizacije
	\item \textbf{Publication quality}: LaTeX podrška, fine-tuning
\end{itemize}

\subsubsection{Napredni primjer: Prostorno-eksplicitni epidemiološki model}

\begin{lstlisting}[language=Python, caption=SIR model s prostornom difuzijom]
	import numpy as np
	import matplotlib.pyplot as plt
	from scipy.integrate import solve_ivp
	from scipy.ndimage import gaussian_filter
	import matplotlib.animation as animation
	from mpl_toolkits.mplot3d import Axes3D
	
	class SpatialSIRModel:
	"""
	Prostorno-eksplicitni SIR epidemiološki model
	s difuzijom i heterogenim kontaktnim uzorcima
	"""
	
	def __init__(self, grid_size, dx, beta, gamma, D_S, D_I, D_R):
	"""
	Parametri:
	grid_size: tuple (nx, ny) - dimenzije prostorne mreže
	dx: float - prostorna rezolucija
	beta: float ili array - stopa prijenosa
	gamma: float - stopa oporavka
	D_S, D_I, D_R: float - difuzijski koeficijenti
	"""
	self.nx, self.ny = grid_size
	self.dx = dx
	self.beta = beta
	self.gamma = gamma
	self.D_S = D_S
	self.D_I = D_I
	self.D_R = D_R
	
	# Prostorna mreža
	self.x = np.linspace(0, (self.nx-1)*dx, self.nx)
	self.y = np.linspace(0, (self.ny-1)*dx, self.ny)
	self.X, self.Y = np.meshgrid(self.x, self.y)
	
	# Laplacian operator za difuziju
	self.laplacian_kernel = np.array([[0, 1, 0],
	[1, -4, 1],
	[0, 1, 0]]) / (dx**2)
	
	def laplacian_2d(self, field):
	"""Compute 2D Laplacian using convolution"""
	from scipy.ndimage import convolve
	return convolve(field, self.laplacian_kernel, mode='constant', cval=0)
	
	def spatial_sir_system(self, t, state_vector):
	"""
	Prostorno-eksplicitni SIR sustav jednadžbi
	"""
	# Reshape 1D state vector to 3D array (3 compartments x nx x ny)
	state = state_vector.reshape(3, self.nx, self.ny)
	S, I, R = state[0], state[1], state[2]
	
	# Ukupna populacija u svakoj ćeliji
	N = S + I + R
	
	# Spriječi dijeljenje s nulom
	N = np.where(N > 0, N, 1)
	
	# SIR dynamics
	if isinstance(self.beta, np.ndarray):
	# Heterogeni kontaktni uzorci
	infection_rate = self.beta * S * I / N
	else:
	# Homogeni kontaktni uzorci
	infection_rate = self.beta * S * I / N
	
	recovery_rate = self.gamma * I
	
	# Difuzijski operatori
	dS_diffusion = self.D_S * self.laplacian_2d(S)
	dI_diffusion = self.D_I * self.laplacian_2d(I)
	dR_diffusion = self.D_R * self.laplacian_2d(R)
	
	# Sustav jednadžbi
	dS_dt = -infection_rate + dS_diffusion
	dI_dt = infection_rate - recovery_rate + dI_diffusion
	dR_dt = recovery_rate + dR_diffusion
	
	# Return flattened derivative
	return np.array([dS_dt, dI_dt, dR_dt]).flatten()
	
	def create_initial_conditions(self, total_pop, initial_infected_centers):
	"""
	Kreiranje početnih uvjeta s lokaliziranim žarištima infekcije
	"""
	S0 = np.ones((self.nx, self.ny)) * total_pop
	I0 = np.zeros((self.nx, self.ny))
	R0 = np.zeros((self.nx, self.ny))
	
	# Dodaj žarišta infekcije
	for center_x, center_y, radius, intensity in initial_infected_centers:
	i_center = int(center_x / self.dx)
	j_center = int(center_y / self.dx)
	
	for i in range(max(0, i_center - radius), 
	min(self.nx, i_center + radius + 1)):
	for j in range(max(0, j_center - radius), 
	min(self.ny, j_center + radius + 1)):
	distance = np.sqrt((i - i_center)**2 + (j - j_center)**2)
	if distance <= radius:
	infection_fraction = intensity * np.exp(-distance**2 / (radius**2))
	infected_count = min(infection_fraction * S0[i, j], S0[i, j])
	I0[i, j] += infected_count
	S0[i, j] -= infected_count
	
	return np.array([S0, I0, R0])
	
	def create_heterogeneous_beta(self, urban_centers, urban_beta, rural_beta):
	"""
	Kreiranje heterogene matrice kontaktnih stopa
	"""
	beta_map = np.ones((self.nx, self.ny)) * rural_beta
	
	for center_x, center_y, radius in urban_centers:
	i_center = int(center_x / self.dx)
	j_center = int(center_y / self.dx)
	
	for i in range(self.nx):
	for j in range(self.ny):
	distance = np.sqrt((i - i_center)**2 + (j - j_center)**2)
	if distance <= radius:
	# Gradijent od urbane do ruralne stope
	urban_fraction = np.exp(-distance**2 / (radius**2))
	beta_map[i, j] = (urban_fraction * urban_beta + 
	(1 - urban_fraction) * rural_beta)
	
	return beta_map
	
	def simulate(self, initial_state, t_span, t_eval=None):
	"""
	Pokretanje simulacije
	"""
	if t_eval is None:
	t_eval = np.linspace(t_span[0], t_span[1], 100)
	
	# Flatten initial state
	y0 = initial_state.flatten()
	
	# Solve ODE system
	solution = solve_ivp(
	self.spatial_sir_system,
	t_span,
	y0,
	t_eval=t_eval,
	method='RK45',
	rtol=1e-6,
	atol=1e-8
	)
	
	# Reshape solution back to 3D
	solution_3d = solution.y.reshape(3, self.nx, self.ny, len(t_eval))
	
	return solution.t, solution_3d
	
	def analyze_epidemic_dynamics(self, t, solution):
	"""
	Analiza dinamike epidemije
	"""
	S, I, R = solution[0], solution[1], solution[2]
	
	# Ukupne populacije kroz vrijeme
	total_S = np.sum(S, axis=(0, 1))
	total_I = np.sum(I, axis=(0, 1))
	total_R = np.sum(R, axis=(0, 1))
	total_N = total_S + total_I + total_R
	
	# Ključne metrike
	peak_infected = np.max(total_I)
	peak_time_idx = np.argmax(total_I)
	peak_time = t[peak_time_idx]
	
	final_attack_rate = total_R[-1] / total_N[0]
	
	# Prostorna analiza
	max_local_infected = np.max(I, axis=2)
	spatial_peak_locations = []
	
	for time_idx in range(len(t)):
	max_val = np.max(max_local_infected[:, :, time_idx])
	if max_val > 0:
	max_locations = np.where(max_local_infected[:, :, time_idx] == max_val)
	if len(max_locations[0]) > 0:
	spatial_peak_locations.append((
	time_idx, 
	max_locations[0][0] * self.dx,
	max_locations[1][0] * self.dx,
	max_val
	))
	
	return {
		'total_timeseries': (total_S, total_I, total_R),
		'peak_infected': peak_infected,
		'peak_time': peak_time,
		'final_attack_rate': final_attack_rate,
		'spatial_peaks': spatial_peak_locations
	}
	
	def run_spatial_sir_example():
	"""
	Pokretanje primjera prostorno-eksplicitnog SIR modela
	"""
	# Parametri modela
	grid_size = (50, 50)
	dx = 1.0  # km
	gamma = 1/14  # oporavak za 14 dana
	D_S = 0.1    # difuzija susceptible
	D_I = 0.05   # difuzija infected (manji zbog karantene)
	D_R = 0.1    # difuzija recovered
	
	# Kreiranje modela
	model = SpatialSIRModel(grid_size, dx, None, gamma, D_S, D_I, D_R)
	
	# Heterogena struktura kontakata
	urban_centers = [(15, 15, 8), (35, 35, 6)]  # (x, y, radius)
	beta_map = model.create_heterogeneous_beta(
	urban_centers, 
	urban_beta=0.5,   # visoka stopa u gradovima
	rural_beta=0.1    # niska stopa u ruralnim područjima
	)
	model.beta = beta_map
	
	# Početni uvjeti
	total_population = 1000
	infection_centers = [(15, 15, 3, 0.05), (35, 35, 2, 0.02)]  # (x, y, radius, intensity)
	
	initial_state = model.create_initial_conditions(
	total_population, 
	infection_centers
	)
	
	# Simulacija
	t_span = (0, 365)  # godina dana
	t_eval = np.linspace(0, 365, 200)
	
	print("Pokretanje prostorno-eksplicitne SIR simulacije...")
	time, solution = model.simulate(initial_state, t_span, t_eval)
	
	# Analiza
	results = model.analyze_epidemic_dynamics(time, solution)
	
	print(f"Vrhunac infekcije: {results['peak_infected']:.0f} na dan {results['peak_time']:.1f}")
	print(f"Konačna stopa napada: {results['final_attack_rate']:.3f}")
	
	# Vizualizacija
	visualize_spatial_sir_results(model, time, solution, results)
	
	def visualize_spatial_sir_results(model, time, solution, results):
	"""
	Vizualizacija rezultata prostorno-eksplicitnog SIR modela
	"""
	S, I, R = solution[0], solution[1], solution[2]
	
	# Kreiranje figure
	fig = plt.figure(figsize=(15, 10))
	
	# Subplot 1: Vremenske serije ukupnih populacija
	ax1 = plt.subplot(2, 3, 1)
	total_S, total_I, total_R = results['total_timeseries']
	
	plt.plot(time, total_S, 'b-', label='Susceptible', linewidth=2)
	plt.plot(time, total_I, 'r-', label='Infected', linewidth=2)
	plt.plot(time, total_R, 'g-', label='Recovered', linewidth=2)
	plt.xlabel('Vrijeme (dani)')
	plt.ylabel('Broj jedinki')
	plt.title('Ukupna dinamika populacije')
	plt.legend()
	plt.grid(True, alpha=0.3)
	
	# Subplot 2: Prostorna distribucija na vrhuncu
	peak_idx = np.argmax(total_I)
	
	ax2 = plt.subplot(2, 3, 2)
	im2 = plt.imshow(I[:, :, peak_idx].T, origin='lower', 
	extent=[0, model.nx*model.dx, 0, model.ny*model.dx],
	cmap='Reds', interpolation='bilinear')
	plt.colorbar(im2, ax=ax2, label='Infected')
	plt.xlabel('X (km)')
	plt.ylabel('Y (km)')
	plt.title(f'Infected na vrhuncu (dan {time[peak_idx]:.1f})')
	
	# Subplot 3: Konačna prostorna distribucija oporavljenih
	ax3 = plt.subplot(2, 3, 3)
	im3 = plt.imshow(R[:, :, -1].T, origin='lower',
	extent=[0, model.nx*model.dx, 0, model.ny*model.dx],
	cmap='Greens', interpolation='bilinear')
	plt.colorbar(im3, ax=ax3, label='Recovered')
	plt.xlabel('X (km)')
	plt.ylabel('Y (km)')
	plt.title('Konačna distribucija oporavljenih')
	
	# Subplot 4: Evolucija prostornog centra infekcije
	ax4 = plt.subplot(2, 3, 4)
	
	# Računanje centra mase infekcije kroz vrijeme
	centers_x = []
	centers_y = []
	
	for t_idx in range(len(time)):
	I_current = I[:, :, t_idx]
	total_infected = np.sum(I_current)
	
	if total_infected > 0:
	center_x = np.sum(model.X * I_current) / total_infected
	center_y = np.sum(model.Y * I_current) / total_infected
	centers_x.append(center_x)
	centers_y.append(center_y)
	else:
	centers_x.append(np.nan)
	centers_y.append(np.nan)
	
	# Plot trajectory
	valid_indices = ~np.isnan(centers_x)
	if np.any(valid_indices):
	plt.plot(np.array(centers_x)[valid_indices], 
	np.array(centers_y)[valid_indices], 
	'ro-', markersize=3, linewidth=1)
	plt.plot(centers_x[0], centers_y[0], 'go', markersize=8, label='Početak')
	plt.plot(centers_x[peak_idx], centers_y[peak_idx], 'ro', markersize=8, label='Vrhunac')
	
	plt.xlabel('X (km)')
	plt.ylabel('Y (km)')
	plt.title('Kretanje centra infekcije')
	plt.legend()
	plt.grid(True, alpha=0.3)
	
	# Subplot 5: Heatmap ukupne incidencije
	ax5 = plt.subplot(2, 3, 5)
	
	# Ukupna incidencija = početni S - konačni S
	total_incidence = S[:, :, 0] - S[:, :, -1]
	
	im5 = plt.imshow(total_incidence.T, origin='lower',
	extent=[0, model.nx*model.dx, 0, model.ny*model.dx],
	cmap='YlOrRd', interpolation='bilinear')
	plt.colorbar(im5, ax=ax5, label='Ukupna incidencija')
	plt.xlabel('X (km)')
	plt.ylabel('Y (km)')
	plt.title('Ukupna incidencija')
	
	# Subplot 6: Vremenska evolucija maksimalne lokalne incidencije
	ax6 = plt.subplot(2, 3, 6)
	
	max_local_I = np.max(I, axis=(0, 1))
	mean_I = np.mean(I, axis=(0, 1))
	std_I = np.std(I, axis=(0, 1))
	
	plt.plot(time, max_local_I, 'r-', linewidth=2, label='Maksimalna lokalna')
	plt.plot(time, mean_I, 'b-', linewidth=2, label='Srednja')
	plt.fill_between(time, mean_I - std_I, mean_I + std_I, 
	alpha=0.3, color='blue', label='±1 SD')
	
	plt.xlabel('Vrijeme (dani)')
	plt.ylabel('Broj zaraženih')
	plt.title('Prostorna varijabilnost infekcije')
	plt.legend()
	plt.grid(True, alpha=0.3)
	
	plt.tight_layout()
	plt.show()
	
	# Pokretanje primjera
	if __name__ == "__main__":
	run_spatial_sir_example()
\end{lstlisting}

\subsection{Julia programski jezik}

\subsubsection{Uvod u Julia za ekološko modeliranje}

Julia je moderan programski jezik dizajniran za visokoperformantno znanstveno računanje. Kombinira jednostavnost Python-a s brzinom C-a, što ga čini idealnim za računalno zahtjevne ekološke simulacije. Razvijen na MIT-u, Julia posebno je prikladna za numeričke simulacije, optimizaciju i paralelno računanje.

\subsubsection{Prednosti Julia za ekologiju}

\begin{itemize}
	\item \textbf{Performanse}: Brzina bliska C/Fortran kodu
	\item \textbf{Jednostavnost}: Čitljiva sintaksa slična Python-u i R-u
	\item \textbf{Multiple dispatch}: Fleksibilan sustav funkcija
	\item \textbf{Paralelizam}: Ugrađena podrška za distribuirano računanje
	\item \textbf{Interoperabilnost}: Pozivanje C, Python, R koda
	\item \textbf{Dinamičnost}: REPL okruženje za interaktivno programiranje
\end{itemize}

\subsubsection{Ključni paketi za ekološko modeliranje}

\paragraph{Diferencijalne jednadžbe i simulacije}

\begin{itemize}
	\item \texttt{DifferentialEquations.jl}: Napredni ODE/SDE/PDE rješavači
	\item \texttt{ModelingToolkit.jl}: Simboličko modeliranje sustava
	\item \texttt{Catalyst.jl}: Reakcijske mreže i biokemijski modeli
	\item \texttt{StochasticDiffEq.jl}: Stohastičke diferencijalne jednadžbe
\end{itemize}

\paragraph{Optimizacija i fitting}

\begin{itemize}
	\item \texttt{Optim.jl}: Optimizacijski algoritmi
	\item \texttt{BlackBoxOptim.jl}: Global optimization
	\item \texttt{DiffEqParamEstim.jl}: Parameter estimation za DE
	\item \texttt{Turing.jl}: Bayesovska analiza
\end{itemize}

\paragraph{Podatkovni rad i statistika}

\begin{itemize}
	\item \texttt{DataFrames.jl}: Manipulacija tabličnih podataka
	\item \texttt{CSV.jl}: Čitanje/pisanje CSV datoteka
	\item \texttt{Statistics.jl}: Osnovne statistike
	\item \texttt{StatsBase.jl}: Napredne statistike
	\item \texttt{Distributions.jl}: Vjerojatnosne distribucije
\end{itemize}

\paragraph{Vizualizacija}

\begin{itemize}
	\item \texttt{Plots.jl}: Opća plotting biblioteka
	\item \texttt{PlotlyJS.jl}: Interaktivne web vizualizacije
	\item \texttt{Makie.jl}: Visokokvalitetna grafika
	\item \texttt{StatsPlots.jl}: Statistička vizualizacija
\end{itemize}

\subsubsection{Napredni primjer: Metapopulacijski model s genetskom strukturom}

\begin{lstlisting}[language=C, caption=Julia implementacija genetski strukturiranog metapopulacijskog modela]
	using DifferentialEquations, Plots, LinearAlgebra, Random, Statistics
	using DataFrames, CSV, StatsBase
	
	# Definicija strukture za metapopulacijski model
	struct GeneticMetapopulationModel{T}
	n_patches::Int64                    # broj patch-eva
	n_alleles::Int64                    # broj alela po lokusu
	connectivity_matrix::Matrix{T}      # matrica povezanosti
	carrying_capacity::Vector{T}        # nosivost svakog patch-a
	migration_rates::Matrix{T}          # matrica migracije
	selection_coefficients::Vector{T}   # selekcijski koeficijenti
	mutation_rate::T                    # stopa mutacije
	genetic_drift_strength::T           # jačina genetskog drifta
	end
	
	function create_metapopulation_model(;
	n_patches::Int = 10,
	n_alleles::Int = 2,
	landscape_size::Float64 = 100.0,
	dispersal_kernel::Float64 = 10.0,
	carrying_capacity_mean::Float64 = 1000.0,
	carrying_capacity_cv::Float64 = 0.3,
	selection_strength::Float64 = 0.01,
	mutation_rate::Float64 = 1e-6,
	genetic_drift_strength::Float64 = 1.0
	)
	"""
	Kreiranje prostorno-eksplicitnog genetski strukturiranog metapopulacijskog modela
	"""
	
	# Generiranje nasumičnih pozicija patch-eva
	Random.seed!(42)
	patch_positions = rand(n_patches, 2) .* landscape_size
	
	# Računanje matrice udaljenosti
	distance_matrix = zeros(n_patches, n_patches)
	for i in 1:n_patches
	for j in 1:n_patches
	if i != j
	distance_matrix[i, j] = sqrt(sum((patch_positions[i, :] .- patch_positions[j, :]).^2))
	end
	end
	end
	
	# Matrica povezanosti (eksponencijalno opadanje s udaljenošću)
	connectivity_matrix = exp.(-distance_matrix ./ dispersal_kernel)
	connectivity_matrix[diagind(connectivity_matrix)] .= 0.0
	
	# Normalizacija po redovima
	for i in 1:n_patches
	row_sum = sum(connectivity_matrix[i, :])
	if row_sum > 0
	connectivity_matrix[i, :] ./= row_sum
	end
	end
	
	# Nosivost patch-eva (log-normalna distribucija)
	carrying_capacity = exp.(randn(n_patches) .* carrying_capacity_cv .+ log(carrying_capacity_mean))
	
	# Matrica migracije (proporcionalna povezanosti)
	base_migration_rate = 0.1
	migration_rates = connectivity_matrix .* base_migration_rate
	
	# Selekcijski koeficijenti za svaki alel
	selection_coefficients = randn(n_alleles) .* selection_strength
	selection_coefficients[1] = 0.0  # referentni alel
	
	return GeneticMetapopulationModel(
	n_patches,
	n_alleles,
	connectivity_matrix,
	carrying_capacity,
	migration_rates,
	selection_coefficients,
	mutation_rate,
	genetic_drift_strength
	)
	end
	
	function metapopulation_dynamics!(du, u, p, t)
	"""
	Sustav diferencijalnih jednadžbi za genetski strukturiranu metapopulaciju
	
	State vektor organiziran kao:
	[N₁, p₁₁, p₁₂, ..., p₁ₐ, N₂, p₂₁, p₂₂, ..., p₂ₐ, ...]
	gdje je Nᵢ ukupna populacija u patch-u i, a pᵢⱼ frekvencija alela j u patch-u i
	"""
	
	model = p
	n_patches = model.n_patches
	n_alleles = model.n_alleles
	state_per_patch = n_alleles + 1  # N + frekvencije alela
	
	# Inicijalizacija derivativa
	fill!(du, 0.0)
	
	for i in 1:n_patches
	# Indeksi za trenutni patch
	start_idx = (i-1) * state_per_patch + 1
	N_idx = start_idx
	allele_start = start_idx + 1
	allele_end = start_idx + n_alleles
	
	# Trenutno stanje
	N_i = max(u[N_idx], 0.0)
	allele_freqs = u[allele_start:allele_end]
	
	# Normalizacija frekvencija alela
	total_freq = sum(allele_freqs)
	if total_freq > 0
	allele_freqs ./= total_freq
	else
	allele_freqs .= 1.0 / n_alleles
	end
	
	# Populacijska dinamika s logističkim rastom
	carrying_capacity = model.carrying_capacity[i]
	
	# Średnja prilagodba (weighted average fitness)
	mean_fitness = sum(allele_freqs .* (1.0 .+ model.selection_coefficients))
	
	# Populacijski rast
	intrinsic_growth = 0.1 * mean_fitness
	dN_local = intrinsic_growth * N_i * (1 - N_i / carrying_capacity)
	
	# Migracija populacije
	dN_migration = 0.0
	for j in 1:n_patches
	if i != j
	other_N_idx = (j-1) * state_per_patch + 1
	N_j = max(u[other_N_idx], 0.0)
	
	# Imigracija iz patch-a j u patch i
	immigration = model.migration_rates[j, i] * N_j
	
	# Emigracija iz patch-a i u patch j
	emigration = model.migration_rates[i, j] * N_i
	
	dN_migration += immigration - emigration
	end
	end
	
	du[N_idx] = dN_local + dN_migration
	
	# Dinamika frekvencija alela
	for a in 1:n_alleles
	allele_idx = allele_start + a - 1
	current_freq = allele_freqs[a]
	
	# Selekcija
	fitness_a = 1.0 + model.selection_coefficients[a]
	dfreq_selection = current_freq * (fitness_a - mean_fitness)
	
	# Mutacija
	mutation_in = model.mutation_rate * (1.0 - current_freq) / (n_alleles - 1)
	mutation_out = model.mutation_rate * current_freq
	dfreq_mutation = mutation_in - mutation_out
	
	# Genetski drift (stohastička komponenta)
	if N_i > 0
	drift_variance = current_freq * (1 - current_freq) / (2 * N_i)
	drift_strength = model.genetic_drift_strength
	dfreq_drift = sqrt(drift_variance) * drift_strength * randn()
	else
	dfreq_drift = 0.0
	end
	
	# Migracija alela
	dfreq_migration = 0.0
	for j in 1:n_patches
	if i != j
	other_allele_idx = (j-1) * state_per_patch + 1 + a
	other_freq = u[other_allele_idx]
	other_N_idx = (j-1) * state_per_patch + 1
	N_j = max(u[other_N_idx], 0.0)
	
	if N_i > 0 && N_j > 0
	# Gene flow weighted by migration
	migration_weight = model.migration_rates[j, i]
	dfreq_migration += migration_weight * (other_freq - current_freq)
	end
	end
	end
	
	du[allele_idx] = dfreq_selection + dfreq_mutation + dfreq_drift + dfreq_migration
	end
	end
	
	return nothing
	end
	
	function run_genetic_metapopulation_simulation(;
	model_params = Dict(),
	simulation_time = 1000.0,
	n_timepoints = 1000,
	initial_population_fraction = 0.5,
	initial_allele_diversity = 0.1
	)
	"""
	Pokretanje simulacije genetski strukturirane metapopulacije
	"""
	
	println("Kreiranje metapopulacijskog modela...")
	model = create_metapopulation_model(; model_params...)
	
	n_patches = model.n_patches
	n_alleles = model.n_alleles
	state_per_patch = n_alleles + 1
	total_states = n_patches * state_per_patch
	
	# Početni uvjeti
	println("Postavljanje početnih uvjeta...")
	u0 = zeros(total_states)
	
	for i in 1:n_patches
	start_idx = (i-1) * state_per_patch + 1
	N_idx = start_idx
	allele_start = start_idx + 1
	
	# Početna populacija
	u0[N_idx] = model.carrying_capacity[i] * initial_population_fraction
	
	# Početne frekvencije alela (blago je varijabilne)
	base_freq = 1.0 / n_alleles
	for a in 1:n_alleles
	allele_idx = allele_start + a - 1
	variation = (rand() - 0.5) * initial_allele_diversity
	u0[allele_idx] = max(0.01, base_freq + variation)
	end
	
	# Normalizacija frekvencija
	allele_sum = sum(u0[allele_start:allele_start + n_alleles - 1])
	u0[allele_start:allele_start + n_alleles - 1] ./= allele_sum
	end
	
	# Definicija vremenskog raspona
	tspan = (0.0, simulation_time)
	t_eval = range(0.0, simulation_time, length=n_timepoints)
	
	# Problem setup
	println("Postavljanje i rješavanje diferencijalnih jednadžbi...")
	prob = ODEProblem(metapopulation_dynamics!, u0, tspan, model)
	
	# Rješavanje s adaptive time stepping
	sol = solve(prob, Tsit5(), saveat=t_eval, reltol=1e-6, abstol=1e-8)
	
	return model, sol
	end
	
	function analyze_genetic_metapopulation_results(model, solution)
	"""
	Analiza rezultata genetski strukturirane metapopulacije
	"""
	
	n_patches = model.n_patches
	n_alleles = model.n_alleles
	state_per_patch = n_alleles + 1
	times = solution.t
	
	results = Dict()
	
	# Izvlačenje vremenskih serija
	total_populations = zeros(length(times), n_patches)
	allele_frequencies = zeros(length(times), n_patches, n_alleles)
	
	for (t_idx, t) in enumerate(times)
	state = solution.u[t_idx]
	
	for i in 1:n_patches
	start_idx = (i-1) * state_per_patch + 1
	N_idx = start_idx
	allele_start = start_idx + 1
	allele_end = start_idx + n_alleles
	
	total_populations[t_idx, i] = max(state[N_idx], 0.0)
	allele_freqs = state[allele_start:allele_end]
	
	# Normalizacija
	total_freq = sum(allele_freqs)
	if total_freq > 0
	allele_frequencies[t_idx, i, :] = allele_freqs ./ total_freq
	else
	allele_frequencies[t_idx, i, :] .= 1.0 / n_alleles
	end
	end
	end
	
	# Ukupna metapopulacija
	total_metapopulation = sum(total_populations, dims=2)[:, 1]
	
	# Globalne frekvencije alela (weighted by population size)
	global_allele_freqs = zeros(length(times), n_alleles)
	
	for t_idx in 1:length(times)
	total_pop = total_metapopulation[t_idx]
	if total_pop > 0
	for a in 1:n_alleles
	weighted_freq = sum(total_populations[t_idx, :] .* allele_frequencies[t_idx, :, a])
	global_allele_freqs[t_idx, a] = weighted_freq / total_pop
	end
	else
	global_allele_freqs[t_idx, :] .= 1.0 / n_alleles
	end
	end
	
	# Heterozigotnost i Fst analize
	expected_heterozygosity = zeros(length(times))
	observed_heterozygosity = zeros(length(times))
	fst_values = zeros(length(times))
	
	for t_idx in 1:length(times)
	# Expected heterozygosity (Hardy-Weinberg)
	global_freqs = global_allele_freqs[t_idx, :]
	He = 1 - sum(global_freqs.^2)
	expected_heterozygosity[t_idx] = He
	
	# Observed heterozygosity and Fst (simplified calculation)
	patch_He = zeros(n_patches)
	patch_weights = zeros(n_patches)
	
	for i in 1:n_patches
	if total_populations[t_idx, i] > 0
	local_freqs = allele_frequencies[t_idx, i, :]
	patch_He[i] = 1 - sum(local_freqs.^2)
	patch_weights[i] = total_populations[t_idx, i]
	end
	end
	
	total_weight = sum(patch_weights)
	if total_weight > 0
	patch_weights ./= total_weight
	Hs = sum(patch_weights .* patch_He)  # average within-patch diversity
	observed_heterozygosity[t_idx] = Hs
	
	# Fst = (Ht - Hs) / Ht
	if He > 0
	fst_values[t_idx] = (He - Hs) / He
	end
	end
	end
	
	# Populacijska stabilnost
	cv_populations = std(total_populations, dims=1)[:, 1] ./ mean(total_populations, dims=1)[:, 1]
	
	results["times"] = times
	results["total_populations"] = total_populations
	results["total_metapopulation"] = total_metapopulation
	results["allele_frequencies"] = allele_frequencies
	results["global_allele_frequencies"] = global_allele_freqs
	results["expected_heterozygosity"] = expected_heterozygosity
	results["observed_heterozygosity"] = observed_heterozygosity
	results["fst"] = fst_values
	results["population_cv"] = cv_populations
	
	return results
	end
	
	function visualize_genetic_metapopulation(model, results)
	"""
	Vizualizacija rezultata genetski strukturirane metapopulacije
	"""
	
	times = results["times"]
	
	# Layout za subplot
	l = @layout [a b c; d e f]
	
	# Plot 1: Ukupna metapopulacija
	p1 = plot(times, results["total_metapopulation"], 
	linewidth=2, color=:blue,
	xlabel="Vrijeme", ylabel="Ukupna populacija",
	title="Dinamika metapopulacije", grid=true)
	
	# Plot 2: Populacije po patch-evima
	p2 = plot(xlabel="Vrijeme", ylabel="Populacija po patch-u",
	title="Lokalne populacije", grid=true)
	
	n_patches_to_show = min(5, model.n_patches)
	for i in 1:n_patches_to_show
	plot!(p2, times, results["total_populations"][:, i], 
	linewidth=1, alpha=0.7, label="Patch $i")
	end
	
	# Plot 3: Globalne frekvencije alela
	p3 = plot(xlabel="Vrijeme", ylabel="Frekvencija alela",
	title="Globalna evolucija alela", grid=true)
	
	colors = [:red, :green, :blue, :orange, :purple]
	for a in 1:model.n_alleles
	color = colors[mod(a-1, length(colors)) + 1]
	plot!(p3, times, results["global_allele_frequencies"][:, a], 
	linewidth=2, color=color, label="Alel $a")
	end
	
	# Plot 4: Genetska raznolikost
	p4 = plot(times, results["expected_heterozygosity"], 
	linewidth=2, color=:red, label="Expected (Ht)",
	xlabel="Vrijeme", ylabel="Heterozigotnost",
	title="Genetska raznolikost", grid=true)
	plot!(p4, times, results["observed_heterozygosity"], 
	linewidth=2, color=:blue, label="Observed (Hs)")
	
	# Plot 5: Fst kroz vrijeme
	p5 = plot(times, results["fst"], 
	linewidth=2, color=:purple,
	xlabel="Vrijeme", ylabel="Fst",
	title="Genetska diferencijacija", grid=true)
	
	# Plot 6: Stabilnost populacija
	p6 = bar(1:model.n_patches, results["population_cv"],
	xlabel="Patch ID", ylabel="Koeficijent varijacije",
	title="Stabilnost lokalnih populacija", 
	color=:lightblue, alpha=0.7)
	
	# Kombinacija svih plotova
	final_plot = plot(p1, p2, p3, p4, p5, p6, layout=l, size=(1200, 800))
	
	return final_plot
	end
	
	# Pokretanje primjera
	function main_genetic_metapopulation_example()
	"""
	Glavni primjer pokretanja genetski strukturirane metapopulacije
	"""
	
	println("=== GENETSKI STRUKTURIRANA METAPOPULACIJA ===")
	
	# Parametri simulacije
	model_params = Dict(
	:n_patches => 8,
	:n_alleles => 3,
	:landscape_size => 50.0,
	:dispersal_kernel => 8.0,
	:carrying_capacity_mean => 800.0,
	:carrying_capacity_cv => 0.4,
	:selection_strength => 0.02,
	:mutation_rate => 1e-5,
	:genetic_drift_strength => 0.5
	)
	
	# Pokretanje simulacije
	model, solution = run_genetic_metapopulation_simulation(
	model_params = model_params,
	simulation_time = 500.0,
	n_timepoints = 500,
	initial_population_fraction = 0.6,
	initial_allele_diversity = 0.2
	)
	
	# Analiza rezultata
	println("Analiza rezultata...")
	results = analyze_genetic_metapopulation_results(model, solution)
	
	# Ispis ključnih rezultata
	println("\n=== KLJUČNI REZULTATI ===")
	println("Konačna ukupna populacija: $(round(Int, results["total_metapopulation"][end]))")
	
	final_global_freqs = results["global_allele_frequencies"][end, :]
	for (i, freq) in enumerate(final_global_freqs)
	println("Konačna frekvencija alela $i: $(round(freq, digits=4))")
	end
	
	final_fst = results["fst"][end]
	println("Konačni Fst: $(round(final_fst, digits=4))")
	
	final_heterozygosity = results["expected_heterozygosity"][end]
	println("Konačna heterozigotnost: $(round(final_heterozygosity, digits=4))")
	
	# Vizualizacija
	println("Kreiranje vizualizacija...")
	final_plot = visualize_genetic_metapopulation(model, results)
	display(final_plot)
	
	return model, solution, results, final_plot
	end
	
	# Pozivanje glavnog primjera
	# main_genetic_metapopulation_example()
\end{lstlisting}

\subsubsection{Instalacija i upravljanje paketima}

\paragraph{Instalacija Julia}

\begin{lstlisting}[language=bash, caption=Instalacija Julia]
	# Download binaries from https://julialang.org/downloads/
	
	# Linux - using juliaup (recommended)
	curl -fsSL https://install.julialang.org | sh
	
	# MacOS - using Homebrew
	brew install julia
	
	# Alternativno, direktno preuzimanje
	wget https://julialang-s3.julialang.org/bin/linux/x64/1.9/julia-1.9.3-linux-x86_64.tar.gz
	tar -xzf julia-1.9.3-linux-x86_64.tar.gz
	sudo mv julia-1.9.3 /opt/
	sudo ln -s /opt/julia-1.9.3/bin/julia /usr/local/bin/julia
\end{lstlisting}

\paragraph{Upravljanje paketima u Julia}

\begin{lstlisting}[language=C, caption=Julia package management]
	# Ulazak u Pkg REPL mod (pritisnuti ']' u Julia REPL)
	] 
	
	# Dodavanje paketa
	add DifferentialEquations
	add Plots DataFrames CSV
	
	# Dodavanje specifične verzije
	add DifferentialEquations@6.15
	
	# Dodavanje iz GitHub repozitorija
	add https://github.com/user/PackageName.jl
	
	# Ažuriranje paketa
	update
	
	# Status instaliranih paketa
	status
	
	# Uklanjanje paketa
	remove PackageName
	
	# Kreiranje projektnog okruženja
	activate .
	instantiate  # instaliraj sve pakete iz Project.toml
	
	# Testiranje paketa
	test DifferentialEquations
	
	# Izlazak iz Pkg moda (pritisnuti backspace)
\end{lstlisting}

\paragraph{Project.toml za reproducibilnost}

\begin{lstlisting}[language=C, caption=Project.toml za ekološki projekt]
	name = "EcologicalModeling"
	version = "0.1.0"
	
	[deps]
	DifferentialEquations = "0c46a032-eb83-5123-abaf-570d42b7fbaa"
	Plots = "91a5bcdd-55d7-5caf-9e0b-520d859cae80"
	DataFrames = "a93c6f00-e57d-5684-b7b6-d8193f3e46c0"
	CSV = "336ed68f-0bac-5ca0-87d4-7b16caf5d00b"
	StatsBase = "2913bbd2-ae8a-5f71-8c99-4fb6c76f3a91"
	Distributions = "31c24e10-a181-5473-b8eb-7969acd0382f"
	LinearAlgebra = "37e2e46d-f89d-539d-b4ee-838fcccc9c8e"
	Random = "9a3f8284-a2c9-5f02-9a11-845980a1fd5c"
	Statistics = "10745b16-79ce-11e8-11f9-7d13ad32a3b2"
	
	[compat]
	julia = "1.8"
	DifferentialEquations = "7"
	Plots = "1.35"
	DataFrames = "1.4"
\end{lstlisting}

\subsection{NetLogo}

\subsubsection{Uvod u NetLogo}

NetLogo je programsko okruženje specijalno dizajnirano za agent-based modeliranje (ABM). Razvio ga je Uri Wilensky na Northwestern University. NetLogo omogućava modeliranje složenih sustava koji nastaju iz interakcija mnoštva individualnih agenata, što ga čini idealnim za ekološko modeliranje na razini jedinki i populacija.

\subsubsection{Ključne značajke NetLogo-a}

\begin{itemize}
	\item \textbf{Agent-based pristup}: Modeliranje individualnih organizama
	\item \textbf{Prostorna eksplicitnost}: Ugrađena 2D mrežna struktura
	\item \textbf{Vizualna priroda}: Real-time vizualizacija simulacija
	\item \textbf{Jednostavnost}: Relativno lako učenje
	\item \textbf{Biblioteca modela}: Opsežna kolekcija gotovih modela
\end{itemize}

\subsubsection{Struktura NetLogo modela}

NetLogo model se sastoji od četiri glavna dijela:

\begin{enumerate}
	\item \textbf{Interface}: Korisničko sučelje s kontrolama i grafikonima
	\item \textbf{Code}: Programski kod modela
	\item \textbf{Info}: Dokumentacija modela
	\item \textbf{Procedures}: Funkcije i procedure
\end{enumerate}

\subsubsection{Primjer NetLogo koda: Predator-Prey model}

\begin{lstlisting}[language=C, caption=Agent-based predator-prey model]
	; Globalne varijable
	globals [
	grass-regrowth-time        ; vrijeme potrebno za ponovni rast trave
	sheep-gain-from-food       ; energija koju ovca dobiva od trave
	wolf-gain-from-food        ; energija koju vuk dobiva od ovce
	sheep-reproduce            ; vjerojatnost reprodukcije ovaca
	wolf-reproduce             ; vjerojatnost reprodukcije vukova
	
	; Varijable za praćenje populacija
	initial-number-sheep
	initial-number-wolves
	]
	
	; Definiranje tipova agenata
	breed [sheep a-sheep]
	breed [wolves wolf]
	
	; Svojstva patch-eva (ćelija mreže)
	patches-own [
	countdown                  ; vrijeme do ponovnog rasta trave
	]
	
	; Svojstva ovaca
	sheep-own [
	energy                     ; energija ovce
	]
	
	; Svojstva vukova
	wolves-own [
	energy                     ; energija vuka
	]
	
	; Procedura za pokretanje modela
	to setup
	clear-all
	
	; Postavljanje parametara
	set grass-regrowth-time 30
	set sheep-gain-from-food 4
	set wolf-gain-from-food 20
	set sheep-reproduce 4
	set wolf-reproduce 5
	
	; Kreiranje početne populacije ovaca
	create-sheep initial-number-sheep [
	setxy random-xcor random-ycor
	set shape "sheep"
	set color white
	set size 1.5
	set energy random (2 * sheep-gain-from-food)
	]
	
	; Kreiranje početne populacije vukova
	create-wolves initial-number-wolves [
	setxy random-xcor random-ycor
	set shape "wolf"
	set color black
	set size 2
	set energy random (2 * wolf-gain-from-food)
	]
	
	; Inicijalizacija trave
	ask patches [
	set pcolor green
	set countdown grass-regrowth-time
	]
	
	reset-ticks
	end
\end{lstlisting}

\subsection{Reproducibilnost i najbolje prakse}

\subsubsection{Principi reproducibilne znanosti}

Reproducibilnost je temeljan zahtjev znanstvenog rada. U ekološkom modeliranju, reproducibilnost znači da treći ljudi mogu ponoviti analize i dobiti iste rezultate koristeći iste podatke i metode.

\paragraph{Ključni elementi reproducibilnosti:}

\begin{itemize}
	\item \textbf{Verzioniranje koda}: Korištenje Git-a ili sličnih sustava
	\item \textbf{Dokumentacija}: Jasna dokumentacija svih koraka
	\item \textbf{Upravljanje ovisnostima}: Specificiranje verzija svih paketa
	\item \textbf{Kontejnerizacija}: Docker, Singularity za potpunu reproducibilnost
	\item \textbf{Dijeljenje podataka}: Otvoreni podaci u standardnim formatima
\end{itemize}

\subsubsection{Git verzioniranje za ekološke projekte}

\begin{lstlisting}[language=bash, caption=Git workflow za ekološke projekte]
	# Inicijalizacija repozitorija
	git init ecology_project
	cd ecology_project
	
	# Kreiranje .gitignore datoteke
	echo "*.Rdata
	*.Rhistory
	.RData
	.Ruserdata
	__pycache__/
	*.pyc
	*.pyo
	data/raw/
	results/temp/
	.ipynb_checkpoints/" > .gitignore
	
	# Struktura direktorija
	mkdir -p {data/{raw,processed},scripts,results,docs,figures}
	
	# Prvi commit
	git add .
	git commit -m "Initial project structure"
	
	# Kreiranje brancha za novu analizu
	git checkout -b feature/population-analysis
	
	# Dodavanje remote repozitorija
	git remote add origin https://github.com/username/ecology_project.git
	
	# Push to remote
	git push -u origin main
\end{lstlisting}

\subsubsection{Struktura reproducibilnog projekta}

\begin{lstlisting}[caption=Preporučena struktura direktorija]
	ecology_project/
	├── README.md                 # Opis projekta i upute
	├── LICENSE                   # Licenca
	├── environment.yml           # Conda environment (Python)
	├── renv.lock                # R environment
	├── requirements.txt          # Python paketi
	├── Dockerfile               # Container definicija
	├── Makefile                 # Automatizacija workflow-a
	├── .gitignore               # Git ignore datoteka
	├── data/
	│   ├── raw/                 # Sirovi podaci (read-only)
	│   ├── processed/           # Obrađeni podaci
	│   └── external/            # Eksterni podaci
	├── scripts/
	│   ├── 01_data_preparation.R
	│   ├── 02_exploratory_analysis.py
	│   ├── 03_modeling.R
	│   └── 04_visualization.py
	├── src/                     # Izvorni kod funkcija
	│   ├── R/
	│   └── python/
	├── results/
	│   ├── figures/
	│   ├── tables/
	│   └── models/
	├── docs/
	│   ├── methodology.md
	│   └── analysis_notes.md
	└── tests/                   # Unit testovi
	├── test_functions.R
	└── test_models.py
\end{lstlisting}

\subsubsection{Docker kontejneri za ekološko modeliranje}

\begin{lstlisting}[language=bash, caption=Dockerfile za R + Python okruženje]
	# Bazna slika s R-om
	FROM rocker/verse:4.3.0
	
	# Metapodaci
	LABEL maintainer="your.email@institution.org"
	LABEL description="Reproducible environment for ecological modeling"
	
	# Systemski paketi
	RUN apt-get update && apt-get install -y \
	python3 \
	python3-pip \
	python3-venv \
	gdal-bin \
	libgdal-dev \
	libgeos-dev \
	libproj-dev \
	&& rm -rf /var/lib/apt/lists/*
	
	# R paketi
	RUN install2.r --error \
	dismo \
	vegan \
	mgcv \
	lme4 \
	sf \
	raster \
	ggplot2 \
	dplyr \
	tidyr \
	knitr \
	rmarkdown
	
	# Python okruženje
	RUN python3 -m venv /opt/venv
	ENV PATH="/opt/venv/bin:$PATH"
	
	# Python paketi
	COPY requirements.txt /tmp/
	RUN pip install --no-cache-dir -r /tmp/requirements.txt
	
	# Kopiranje projekta
	WORKDIR /home/rstudio
	COPY . .
	
	# Postavljanje dozvola
	RUN chown -R rstudio:rstudio /home/rstudio
	
	# Default naredba
	CMD ["R"]
\end{lstlisting}

\subsubsection{Automatizacija s Makefile}

\begin{lstlisting}[language=bash, caption=Makefile za automatizaciju analize]
	# Makefile za ekološki modelski projekt
	
	# Varijable
	R_SCRIPTS = scripts
	PYTHON_SCRIPTS = scripts
	DATA_DIR = data
	RESULTS_DIR = results
	FIGURES_DIR = $(RESULTS_DIR)/figures
	
	# Glavni cilj
	all: data analysis figures report
	
	# Priprema podataka
	data: $(DATA_DIR)/processed/cleaned_data.csv
	
	$(DATA_DIR)/processed/cleaned_data.csv: $(DATA_DIR)/raw/field_data.csv
	Rscript $(R_SCRIPTS)/01_data_preparation.R
	
	# Eksplorativna analiza
	analysis: $(RESULTS_DIR)/exploratory_analysis.html
	
	$(RESULTS_DIR)/exploratory_analysis.html: $(DATA_DIR)/processed/cleaned_data.csv
	python $(PYTHON_SCRIPTS)/02_exploratory_analysis.py
	
	# Modeliranje
	models: $(RESULTS_DIR)/models/population_model.rds
	
	$(RESULTS_DIR)/models/population_model.rds: $(DATA_DIR)/processed/cleaned_data.csv
	Rscript $(R_SCRIPTS)/03_modeling.R
	
	# Figure
	figures: $(FIGURES_DIR)/population_dynamics.png
	
	$(FIGURES_DIR)/population_dynamics.png: $(RESULTS_DIR)/models/population_model.rds
	python $(PYTHON_SCRIPTS)/04_visualization.py
	
	# Finalni izvještaj
	report: $(RESULTS_DIR)/final_report.pdf
	
	$(RESULTS_DIR)/final_report.pdf: $(FIGURES_DIR)/population_dynamics.png
	Rscript -e "rmarkdown::render('docs/report.Rmd', output_dir='$(RESULTS_DIR)')"
	
	# Čišćenje
	clean:
	rm -rf $(RESULTS_DIR)/*
	rm -rf $(DATA_DIR)/processed/*
	
	# Testiranje
	test:
	Rscript tests/test_functions.R
	python -m pytest tests/
	
	# Docker build
	docker-build:
	docker build -t ecology-project .
	
	# Docker run
	docker-run:
	docker run -it --rm -v $(PWD):/home/rstudio ecology-project
	
	.PHONY: all data analysis models figures report clean test docker-build docker-run
\end{lstlisting}

\subsubsection{Upravljanje ovisnostima}

\paragraph{R renv sustav}

\begin{lstlisting}[language=R, caption=Upravljanje R ovisnostima s renv]
	# Inicijalizacija renv projekta
	renv::init()
	
	# Instalacija paketa
	install.packages("dismo")
	renv::snapshot()
	
	# Ažuriranje lockfile-a nakon promjena
	renv::snapshot()
	
	# Vraćanje na verzije iz lockfile-a
	renv::restore()
	
	# Ažuriranje svih paketa
	renv::update()
	
	# Status trenutnog okruženja
	renv::status()
	
	# Kreiranje portable library
	renv::isolate()
\end{lstlisting}

\paragraph{Python virtualenv i pip-tools}

\begin{lstlisting}[language=bash, caption=Python dependency management]
	# Kreiranje virtualnog okruženja
	python -m venv ecology_env
	source ecology_env/bin/activate
	
	# Instalacija pip-tools
	pip install pip-tools
	
	# Kreiranje requirements.in
	echo "numpy>=1.20.0
	scipy>=1.7.0
	pandas>=1.3.0
	matplotlib>=3.4.0
	scikit-learn>=1.0.0" > requirements.in
	
	# Generiranje lockovanog requirements.txt
	pip-compile requirements.in
	
	# Instalacija točnih verzija
	pip-sync requirements.txt
	
	# Ažuriranje ovisnosti
	pip-compile --upgrade requirements.in
\end{lstlisting}

\subsection{Integracijska prilazi i interoperabilnost}

\subsubsection{R i Python integracija}

\paragraph{reticulate paket}

\begin{lstlisting}[language=R, caption=Pozivanje Python-a iz R-a]
	library(reticulate)
	
	# Specificiranje Python okruženja
	use_virtualenv("ecology_env")
	
	# Import Python modula
	np <- import("numpy")
	plt <- import("matplotlib.pyplot")
	
	# Korištenje Python funkcija
	data <- np$random$normal(0, 1, 1000L)
	hist(py_to_r(data))
	
	# Izvršavanje Python koda
	py_run_string("
	import numpy as np
	result = np.mean([1, 2, 3, 4, 5])
	")
	
	# Pristup Python varijablama
	py$result
\end{lstlisting}

\subsection{Zaključak i preporuke}

\subsubsection{Izbor alata prema vrsti problema}

% Napomena: Potreban je \usepackage{longtable} u preambuli dokumenta

\begin{longtable}{|l|l|l|l|}
	\caption{Preporučeni alati prema vrsti modeliranja} \label{tab:alati-modeliranje} \\
	\hline
	\textbf{Vrsta modeliranja} & \textbf{Primarni alat} & \textbf{Alternativni alati} & \textbf{Specijalizirani slučajevi} \\
	\hline
	\endfirsthead
	
	\multicolumn{4}{c}%
	{{\bfseries Tablica \thetable{} -- nastavak s prethodne stranice}} \\
	\hline
	\textbf{Vrsta modeliranja} & \textbf{Primarni alat} & \textbf{Alternativni alati} & \textbf{Specijalizirani slučajevi} \\
	\hline
	\endhead
	
	\hline 
	\multicolumn{4}{|r|}{{Nastavlja se na sljedećoj stranici}} \\ 
	\hline
	\endfoot
	
	\hline
	\endlastfoot
	
	Statistička analiza & R & Python (statsmodels) & Julia (StatsBase.jl) \\
	\hline
	Prostorno modeliranje & R (sf, raster) & Python (geopandas) & Julia (GeoStats.jl) \\
	\hline
	Strojno učenje & Python & R (caret, tidymodels) & Julia (MLJ.jl, Flux.jl) \\
	\hline
	Agent-based modeli & NetLogo & Python (Mesa), R & Julia (Agents.jl) \\
	\hline
	Kompleksni ODE sustavi & Julia (DiffEq.jl) & MATLAB, Python (scipy) & GNU Octave \\
	\hline
	Stohastičke diferencijalne jednadžbe & Julia (StochasticDiffEq.jl) & MATLAB & Python (scipy) \\
	\hline
	Bayesovska analiza & R (Stan) & Python (PyMC) & Julia (Turing.jl) \\
	\hline
	Numerička optimizacija & Julia (Optim.jl) & MATLAB (Optimization) & GNU Octave, Python \\
	\hline
	Visokomperformantno računanje & Julia & Rust, C/C++, Fortran & Python (Numba), MATLAB \\
	\hline
	Paralelno računanje & Julia & Python (multiprocessing) & MATLAB (Parallel), R \\
	\hline
	MATLAB kod migracija & GNU Octave & Julia & Python (MATLAB Engine) \\
	\hline
	Vizualizacija & R (ggplot2) & Python (matplotlib, plotly) & Julia (Plots.jl, Makie.jl) \\
	\hline
	Interaktivna analiza & Python (Jupyter) & R (RMarkdown) & Julia (Pluto.jl) \\
	\hline
	Reproducibilnost & R (renv) & Python (conda/pip) & Julia (Pkg.jl) \\
	\hline
	Genetička algoritmi & Python (DEAP) & R (GA) & Julia (Evolutionary.jl) \\
	\hline
	Mrežna analiza & R (igraph) & Python (NetworkX) & Julia (Graphs.jl) \\
	\hline
	Vremenske serije & R (forecast) & Python (statsmodels) & Julia (TimeSeries.jl) \\
	\hline
	Prostorna statistika & R (gstat, sp) & Python (geostatspy) & Julia (GeoStats.jl) \\
	\hline
	Monte Carlo simulacije & Julia & Python, R & MATLAB, GNU Octave \\
	\hline
	Simboličko računanje & Python (SymPy) & MATLAB (Symbolic) & Julia (Symbolics.jl) \\
	
\end{longtable}

\vspace{1em}

\subsubsection{Objašnjenja i smjernice za izbor}

\paragraph{GNU Octave specijalizacije:}
\begin{itemize}
	\item \textbf{MATLAB migracija}: 95\%+ sintaksna kompatibilnost omogućava direktno pokretanje postojećeg MATLAB koda
	\item \textbf{Obrazovne institucije}: Besplatan pristup MATLAB-ovalnim mogućnostima
	\item \textbf{Numerička stabilnost}: Koristi iste BLAS/LAPACK biblioteke kao MATLAB
	\item \textbf{Ograničenja}: Sporiji od MATLAB-a, manje toolbox-ova, slabija 3D grafika
\end{itemize}

\paragraph{Julia specijalizacije:}
\begin{itemize}
	\item \textbf{Diferencijalne jednadžbe}: DifferentialEquations.jl najnapredniji je ekosustav za DE
	\item \textbf{Visokomperformantno računanje}: "Two language problem" rješavanje - brzina C-a s jednostavnošću Python-a
	\item \textbf{Paralelizam}: Ugrađena podrška za distribuirano računanje i GPU
	\item \textbf{Znanstvena područja}: Posebno snažan u fizici, biologiji, ekonomiji
	\item \textbf{Ograničenja}: Mlađa zajednica, manji broj paketa od Python-a/R-a
\end{itemize}

\paragraph{Hibridni pristupi:}
\begin{itemize}
	\item \textbf{R + Julia}: R za statistiku i vizualizaciju, Julia za računalno zahtjevne simulacije
	\item \textbf{Python + Julia}: Python za ML i podatkovnu analizu, Julia za numeričko modeliranje  
	\item \textbf{MATLAB/Octave + R}: MATLAB/Octave za inženjerske simulacije, R za statističku analizu
	\item \textbf{NetLogo + R/Python}: NetLogo za ABM prototipiranje, R/Python za skalabilne implementacije
\end{itemize}

\subsubsection{Ključne preporuke}

\begin{enumerate}
	\item \textbf{Kombiniraj alate}: Koristi prednosti svakog alata
	
	Nijedan alat nije savršen za sve situacije. R je izvrsno za statistiku i vizualizaciju, Python za strojno učenje i skalabilnost, MATLAB za numeričke simulacije, NetLogo za agent-based modele. Kombiniraj ih strategijski - koristi R za početnu analizu i čišćenje podataka, prebaci u Python za složene algoritme strojnog učenja, a rezultate vizualiziraj u R-u s ggplot2. Takav hibridni pristup omogućava maksimalnu iskoristivost specifičnih prednosti svakog alata.
	
	\item \textbf{Dokumentiraj sve}: Kod, podatke, pretpostavke, ograničenja
	
	Dobra dokumentacija je temelj reproducibilne znanosti. Svaki kod trebao bi imati jasne komentare koji objašnjavaju logiku, svaki dataset opise varijabli i metoda prikupljanja, svaki model eksplicitne pretpostavke. Kreiraj README datoteke za projekte, koristi comment tagove u kodu, dokumentiraj sve odluke o čišćenju podataka. Dokumentiraj također ograničenja modela, nesigurnosti u podacima i pretpostavke koje utječu na interpretaciju rezultata.
	
	\item \textbf{Testiraj redovito}: Unit testovi, validacija modela
	
	Sistematsko testiranje sprječava greške i povećava pouzdanost rezultata. Piši unit testove koji provjeravaju da funkcije rade kako je namijenjeno, testiraj modele na poznatim dataset s poznatim rezultatima, provjeravaj granične slučajeve. U R-u koristi pakete testthat ili tinytest, u Python-u pytest. Također validiraj model performanse kroz cross-validation, bootstrap analize i osjetljivost analize.
	
	\item \textbf{Verzitrajte promjene}: Git za kod, DVC za podatke
	
	Verzianje omogućava praćenje promjena i vraćanje na prethodne verzije ako nešto pođe po zlu. Git je standard za verzioniranje koda- \textit{"Commitajte"} često s opisnim porukama, koristite \textit{"branching"} za eksperimente, \textit{"maintainajte"} čist glavni \textit{"branch"}. Za velike skupove podataka neprikladne za Git, koristite DVC (\textit{Data Version Control}) kako bi pratili promjene podataka i omogućili reproduciranje točnih verzija podataka za svaki eksperiment.
	
	\item \textbf{Dijelite otvoreno}: Kod, podaci, rezultati
	
	Otvorena znanost povećava transparentnost i omogućava drugima da grade na vašem radu. Objavljujte kod na GitHub/GitLab s jasnim licencama, dijelite skupove podataka na repozitorijima poput Dryad ili Zenodo, napravite preprinte dostupnima na bioRxiv. Koristite FAIR princip (\textit{Findable, Accessible, Interoperable, Reusable}) za sve digitalne objekte jer tako ne samo što pomažete znanosti već često rezultira boljim citiranjem vaših radova.
	
	\item \textbf{Planirajte skalabilnost}: Optimizacija za veće probleme
	
	Dizajnirajte kod koji može rasti s vašim potrebama. Koristite vektorske operacije umjesto petlji, leveraging paralelizaciju za simulacije, razmislite o \textit{memory-efficient algoritmima} za velike skupove podataka. Ako trenutno radite s 1000 podatkovnih točaka, zapitajte se  sebe hoćete li kod raditi s milijun točaka? Koristite \textit{profiling} alate za identificiranje \textit{bottleneck-a} i optimizirajte kritične dijelove koda. \textit{Cloud computing} platfome mogu omogućiti skaliranje izvan lokalnih ograničenja.
	
	\item \textbf{Radite sigurnosne kopije}: Redundantno pohranjivanje
	
	Gubitak podataka i koda može uništiti mjesece rada. Implementirajte "3-2-1" pravilo: 3 kopije (izvornik + 2 sigurnosne), na 2 različita media (lokalni disk + cloud), s 1 kopijom geografski odvojenom. Automatskim \textit{cloud sync} za aktivne projekte, periodični \textit{backup} za sve skupova epodataka, \textit{offsite storage} za kritične dugotrajne projekt. Git repozitoriji na GitHub/GitLab automatski pružaju \textit{distributed backup} za kod.
	
	\item \textbf{Učici kontinuirano}: Novi alati, tehnike, najbolje prakse
	
	Ekološko modeliranje je dinamično polje s konstantnim novostima. Prati znanstvene journal, blog (R-bloggers, Towards Data Science), attend konference i webinar. Eksperimentiraj s novim paketima i tehnologiji na malim projekt prije implementacije u važan rad. Networking s drugim ekolozim-modelerima kroz Twitter, ResearchGate ili profesionalna udruženja omogućava razmjenu znanja i kolaboracije. Izdvoj vrijeme za redovno osvježavanje vještina.
	
\end{enumerate}
	
	% =====================================================
	\part{Populacije, zajednice i ekosustavi}
	% =====================================================
	
\chapter{Modeli populacijske dinamike}

\section{Jednostavni modeli}

\subsection{Uvod u populacijsku dinamiku}

Populacijska dinamika proučava promjene u veličini i strukturi populacija kroz vrijeme. Temelji se na osnovnim demografskim procesima: rađanju, smrti, imigraciji i emigraciji. Matematičko modeliranje populacijske dinamike omogućava razumijevanje faktora koji utječu na rast ili opadanje populacija te predviđanje budućih trendova.

Opća bilancijska jednadžba za populaciju može se zapisati kao:

\begin{equation}
	\frac{dN}{dt} = \text{Rođenja} - \text{Smrti} + \text{Imigracija} - \text{Emigracija}
\end{equation}

\subsection{Eksponencijalni rast}

\subsubsection{Kontinuirani model eksponencijalnog rasta}

Najjednostavniji model populacijske dinamike pretpostavlja da je stopa rasta populacije proporcionalna trenutnoj veličini populacije:

\begin{equation}
	\frac{dN}{dt} = rN
\end{equation}

gdje je:
\begin{itemize}
	\item $N(t)$ - veličina populacije u vremenu $t$
	\item $r$ - intrinsična stopa rasta (per capita)
\end{itemize}

\textbf{Rješenje diferencijalne jednadžbe:}

\begin{equation}
	N(t) = N_0 e^{rt}
\end{equation}

gdje je $N_0$ početna veličina populacije u $t = 0$.

\subsubsection{Interpretacija parametra $r$}

Intrinsična stopa rasta $r$ može se razložiti na komponente:

\begin{equation}
	r = b - d + i - e
\end{equation}

gdje su:
\begin{itemize}
	\item $b$ - stopa rođenja (births per capita)
	\item $d$ - stopa smrti (deaths per capita)  
	\item $i$ - stopa imigracije
	\item $e$ - stopa emigracije
\end{itemize}

Za zatvorenu populaciju (bez migracije): $r = b - d$

\subsubsection{Vrijeme udvostručavanja}

Vrijeme potrebno da se populacija udvostručuje:

\begin{equation}
	t_{\text{double}} = \frac{\ln(2)}{r} \approx \frac{0.693}{r}
\end{equation}

\subsubsection{Diskretni eksponencijalni model}

Za populacije s diskretnim generacijama:

\begin{equation}
	N_{t+1} = \lambda N_t
\end{equation}

gdje je $\lambda$ konačna stopa rasta (finite rate of increase).

Veza između $r$ i $\lambda$:
\begin{equation}
	\lambda = e^r \quad \text{ili} \quad r = \ln(\lambda)
\end{equation}

\textbf{Opće rješenje:}
\begin{equation}
	N_t = N_0 \lambda^t
\end{equation}

\subsection{Logistički rast}

\subsubsection{Kontinuirani logistički model}

Eksponencijalni rast ne može se nastaviti beskonačno zbog ograničenosti resursa. Logistički model uključuje koncept nosivosti staništa ($K$):

\begin{equation}
	\frac{dN}{dt} = rN\left(1 - \frac{N}{K}\right)
\end{equation}

gdje je $K$ nosivost staništa (carrying capacity).

\subsubsection{Analiza logističke jednadžbe}

\textbf{Ravnotežne točke:}
\begin{itemize}
	\item $N^* = 0$ (nestabilna)
	\item $N^* = K$ (stabilna)
\end{itemize}

\textbf{Rješenje logističke jednadžbe:}

Koristeći separaciju varijabli:

\begin{equation}
	N(t) = \frac{K}{1 + \left(\frac{K-N_0}{N_0}\right)e^{-rt}}
\end{equation}

ili ekvivalentno:

\begin{equation}
	N(t) = \frac{K N_0 e^{rt}}{K + N_0(e^{rt} - 1)}
\end{equation}

\subsubsection{Karakteristike logističke krivulje}

\textbf{Točka infleksije:}

Maksimalna stopa rasta postiže se kada je $N = K/2$:

\begin{equation}
	\left.\frac{d^2N}{dt^2}\right|_{N=K/2} = 0
\end{equation}

Maksimalna stopa rasta:
\begin{equation}
	\left.\frac{dN}{dt}\right|_{\max} = \frac{rK}{4}
\end{equation}

\textbf{Asimptotsko ponašanje:}
\begin{align}
	\lim_{t \to \infty} N(t) &= K \\
	\lim_{t \to -\infty} N(t) &= 0
\end{align}

\subsection{Diskretni modeli rasta}

\subsubsection{Diskretni logistički model}

\begin{equation}
	N_{t+1} = N_t + rN_t\left(1 - \frac{N_t}{K}\right)
\end{equation}

ili ekvivalentno:

\begin{equation}
	N_{t+1} = N_t\left(1 + r\left(1 - \frac{N_t}{K}\right)\right)
\end{equation}

\textbf{Ravnotežna točka:}
\begin{equation}
	N^* = K
\end{equation}

\textbf{Stabilnost:} Ovisi o vrijednosti $r$
\begin{itemize}
	\item $0 < r < 2$: stabilan pristup ravnoteži
	\item $r > 2$: oscilacije ili kaos
\end{itemize}

\subsubsection{Ricker model}

Ricker model posebno je pogodan za populacije s preklapajućim generacijama i gustoćno ovisnom reprodukcijom:

\begin{equation}
	N_{t+1} = N_t e^{r\left(1 - \frac{N_t}{K}\right)}
\end{equation}

\textbf{Linearizacija oko ravnoteže:}

Oko $N^* = K$:
\begin{equation}
	N_{t+1} - K \approx -r(N_t - K)
\end{equation}

\textbf{Uvjeti stabilnosti:}
\begin{itemize}
	\item $0 < r < 2$: monotonski pristup
	\item $2 < r < 2.526$: oscilirajući pristup
	\item $r > 2.526$: kaotična dinamika
\end{itemize}

\subsubsection{Beverton-Holt model}

Model posebno korišten u ribarstvu:

\begin{equation}
	N_{t+1} = \frac{aN_t}{1 + bN_t}
\end{equation}

gdje su $a$ i $b$ pozitivni parametri.

\textbf{Veza s logističkim parametrima:}
\begin{align}
	a &= e^r \\
	b &= \frac{e^r - 1}{K}
\end{align}

\textbf{Ravnotežna točka:}
\begin{equation}
	N^* = \frac{a-1}{b}
\end{equation}

\textbf{Stabilnost:} Beverton-Holt model uvijek konvergira monotonski prema ravnoteži za $a > 1$.

\subsection{Usporedba diskretnih modela}

\begin{longtable}{|l|l|l|l|}
	\caption{Usporedba diskretnih populacijskih modela} \\
	\hline
	\textbf{Model} & \textbf{Jednadžba} & \textbf{Karakteristike} & \textbf{Primjena} \\
	\hline
	\endfirsthead
	\multicolumn{4}{c}{\textbf{Tablica} -- nastavak} \\
	\hline
	\textbf{Model} & \textbf{Jednadžba} & \textbf{Karakteristike} & \textbf{Primjena} \\
	\hline
	\endhead
	\hline
	\endfoot
	\hline
	\endlastfoot
	
	Eksponencijalni & $N_{t+1} = \lambda N_t$ & Neograničen rast & Početni rast populacije \\
	\hline
	Diskretni logistički & $N_{t+1} = N_t(1 + r(1-N_t/K))$ & Može biti nestabilan & Jednostavni modeli \\
	\hline
	Ricker & $N_{t+1} = N_t e^{r(1-N_t/K)}$ & Prekomjerna kompenzacija & Ribe, insekti \\
	\hline
	Beverton-Holt & $N_{t+1} = \frac{aN_t}{1 + bN_t}$ & Uvijek stabilan & Ribarstvo \\
	\hline
\end{longtable}

\subsection{Procjena parametara}

\subsubsection{Linearizacija za procjenu parametara}

\textbf{Eksponencijalni model:}
\begin{equation}
	\ln(N_t) = \ln(N_0) + rt
\end{equation}

Linearna regresija $\ln(N_t)$ na $t$ daje procjenu $r$.

\textbf{Logistički model:}

Gompertz transformacija:
\begin{equation}
	\ln\left(\frac{K-N_t}{N_t}\right) = \ln\left(\frac{K-N_0}{N_0}\right) - rt
\end{equation}

\subsubsection{Nelinearna regresija}

Za direktnu procjenu parametara logističke jednadžbe koristi se nelinearna regresija s funkcijom:

\begin{equation}
	N(t) = \frac{K}{1 + ae^{-rt}}
\end{equation}

gdje je $a = (K-N_0)/N_0$.

\textbf{Početne vrijednosti za optimizaciju:}
\begin{itemize}
	\item $K$: maksimalna opažena vrijednost $\times 1.2$
	\item $r$: gradijent u točki infleksije
	\item $a$: iz početnih uvjeta
\end{itemize}

\section{Dobno/stanovno strukturirani modeli}

\subsection{Uvod u strukturirane modele}

Jednostavni populacijski modeli tretiraju sve jedinke kao identične. U stvarnosti, demografi parametri (rođenja, smrti) uvelike ovise o dobi, veličini ili reproduktivnom stanju jedinki. Strukturirani modeli eksplicitno modeliraju ove razlike.

\subsection{Leslie matrica}

\subsubsection{Teorijski okvir}

Leslie matrica, koju je razvio P.H. Leslie 1945., model je za dobno strukturirane populacije s diskretnim vremenskim koracima.

Populacija se dijeli na $n$ dobnih klasa. Stanje populacije u vremenu $t$ opisuje vektor:

\begin{equation}
	\mathbf{n}_t = \begin{pmatrix} n_1(t) \\ n_2(t) \\ \vdots \\ n_n(t) \end{pmatrix}
\end{equation}

gdje je $n_i(t)$ broj jedinki u $i$-toj dobnoj klasi.

\subsubsection{Konstrukcija Leslie matrice}

Leslie matrica ima oblik:

\begin{equation}
	\mathbf{L} = \begin{pmatrix}
		F_1 & F_2 & F_3 & \cdots & F_{n-1} & F_n \\
		S_1 & 0 & 0 & \cdots & 0 & 0 \\
		0 & S_2 & 0 & \cdots & 0 & 0 \\
		\vdots & \vdots & \ddots & \ddots & \vdots & \vdots \\
		0 & 0 & 0 & \cdots & S_{n-1} & 0
	\end{pmatrix}
\end{equation}

gdje su:
\begin{itemize}
	\item $F_i$ - plodnost $i$-te dobne klase (broj mladih po jedinki)
	\item $S_i$ - preživljavanje iz $i$-te u $(i+1)$-u dobnu klasu
\end{itemize}

\textbf{Populacijska projekcija:}
\begin{equation}
	\mathbf{n}_{t+1} = \mathbf{L} \mathbf{n}_t
\end{equation}

\subsubsection{Asimptotsko ponašanje}

Za ergodičnu Leslie matricu (sve $F_i, S_i \geq 0$, $F_i > 0$ za neki $i$):

\textbf{Dominantna svojstvena vrijednost $\lambda_1$:}
\begin{equation}
	\lambda_1 = \text{najveća svojstvena vrijednost od } \mathbf{L}
\end{equation}

\textbf{Asimptotska stopa rasta:}
\begin{equation}
	r = \ln(\lambda_1)
\end{equation}

\textbf{Stabilna dobna distribucija:}

Odgovarajući svojstveni vektor $\mathbf{w}$:
\begin{equation}
	\mathbf{L} \mathbf{w} = \lambda_1 \mathbf{w}
\end{equation}

Normalizirano: $\sum w_i = 1$

\textbf{Reproduktivna vrijednost:}

Lijevi svojstveni vektor $\mathbf{v}$:
\begin{equation}
	\mathbf{v}^T \mathbf{L} = \lambda_1 \mathbf{v}^T
\end{equation}

$v_i$ predstavlja budući reproduktivni potencijal jedinke u dobnoj klasi $i$.

\subsubsection{Euler-Lotka jednadžba}

Za kontinuirane dobne distribucije, karakteristična jednadžba Leslie matrice postaje:

\begin{equation}
	1 = \int_0^{\infty} e^{-ra} l(a) m(a) da
\end{equation}

gdje su:
\begin{itemize}
	\item $l(a)$ - vjerojatnost preživljavanja do dobi $a$
	\item $m(a)$ - maternalna funkcija (rođenja po jedinki dobi $a$)
	\item $r$ - intrinsična stopa rasta
\end{itemize}

\subsection{Lefkovitch matrica}

\subsubsection{Općeniti okvir}

Lefkovitch matrica generalizacija je Leslie matrice za populacije strukturirane prema bilo kojoj karakteristici (veličina, stadij razvoja, reproduktivno stanje).

\begin{equation}
	\mathbf{A} = \begin{pmatrix}
		a_{11} & a_{12} & \cdots & a_{1n} \\
		a_{21} & a_{22} & \cdots & a_{2n} \\
		\vdots & \vdots & \ddots & \vdots \\
		a_{n1} & a_{n2} & \cdots & a_{nn}
	\end{pmatrix}
\end{equation}

gdje $a_{ij}$ predstavlja doprinos jedinki iz klase $j$ klasi $i$ u sljedećem vremenskom koraku.

\subsubsection{Interpretacija elemenata matrice}

\textbf{Dijagonalni elementi ($a_{ii}$):} Ostanak u istoj klasi
\textbf{Iznad dijagonale:} Napredovanje u veće klase  
\textbf{Ispod dijagonale:} Regresija u manje klase
\textbf{Prvi red:} Reprodukcija (novi potomci)

\textbf{Ograničenja:}
\begin{align}
	a_{ij} &\geq 0 \quad \forall i,j \\
	\sum_{i=2}^{n} a_{ij} &\leq 1 \quad \forall j \geq 2 \text{ (bez prvog reda)}
\end{align}

\subsection{Analiza osjetljivosti i elastičnosti}

\subsubsection{Osjetljivost (Sensitivity)}

Osjetljivost $\lambda$ na promjene u elementi matrice $a_{ij}$:

\begin{equation}
	s_{ij} = \frac{\partial \lambda}{\partial a_{ij}} = \frac{v_i w_j}{\langle \mathbf{v}, \mathbf{w} \rangle}
\end{equation}

gdje je $\langle \mathbf{v}, \mathbf{w} \rangle = \sum v_i w_i$ skalarni produkt.

\textbf{Svojstva osjetljivosti:}
\begin{itemize}
	\item $s_{ij} > 0$ za sve elemente
	\item $\sum_{i,j} s_{ij} a_{ij} = \lambda$ (sumiranje po svim elementima)
\end{itemize}

\subsubsection{Elastičnost (Elasticity)}

Relativan utjecaj proporcionalnih promjena:

\begin{equation}
	e_{ij} = \frac{a_{ij}}{\lambda} \frac{\partial \lambda}{\partial a_{ij}} = \frac{a_{ij} s_{ij}}{\lambda}
\end{equation}

\textbf{Interpretacija:}
$e_{ij}$ predstavlja proporcionalnu promjenu u $\lambda$ koja rezultira iz 1\% promjene u $a_{ij}$.

\textbf{Svojstva elastičnosti:}
\begin{align}
	\sum_{i,j} e_{ij} &= 1 \\
	0 \leq e_{ij} &\leq 1
\end{align}

\subsubsection{LTRE analiza (Life Table Response Experiment)}

LTRE kvantificira razlike u $\lambda$ između populacija:

\begin{equation}
	\lambda^{(1)} - \lambda^{(2)} \approx \sum_{i,j} s_{ij} (a_{ij}^{(1)} - a_{ij}^{(2)})
\end{equation}

gdje gornji indeksi označavaju različite populacije ili tretmane.

\textbf{Doprinos elementa $a_{ij}$:}
\begin{equation}
	c_{ij} = s_{ij} (a_{ij}^{(1)} - a_{ij}^{(2)})
\end{equation}

\subsection{Napredni koncepti strukturiranih modela}

\subsubsection{Gustoćno ovisni strukturirani modeli}

\begin{equation}
	\mathbf{n}_{t+1} = \mathbf{A}(N_t) \mathbf{n}_t
\end{equation}

gdje $\mathbf{A}(N_t)$ ovisi o ukupnoj populaciji $N_t = \sum n_i(t)$.

\textbf{Primjer - logistička regulacija reprodukcije:}
\begin{equation}
	F_i(N_t) = F_i^{\max} \left(1 - \frac{N_t}{K}\right)
\end{equation}

\subsubsection{Stohastički strukturirani modeli}

\textbf{Elementarna stohastičnost:}
\begin{equation}
	\mathbf{n}_{t+1} = \mathbf{A}_t \mathbf{n}_t
\end{equation}

gdje su elementi $\mathbf{A}_t$ slučajne varijable.

\textbf{Stohastička stopa rasta:}
\begin{equation}
	r_s = \lim_{t \to \infty} \frac{1}{t} \ln(N_t)
\end{equation}

Za nekorelirana okruženja:
\begin{equation}
	r_s \approx E[\ln(\lambda_t)] - \frac{1}{2} \text{Var}[\ln(\lambda_t)]
\end{equation}

\subsubsection{Perioidičnost i sezonalnost}

Za sezonske modele s $k$ sezona:

\begin{equation}
	\mathbf{n}_{t+k} = \mathbf{A}_k \mathbf{A}_{k-1} \cdots \mathbf{A}_1 \mathbf{n}_t = \mathbf{B} \mathbf{n}_t
\end{equation}

\textbf{Godišnja stopa rasta:}
$\lambda_{\text{annual}}$ = dominantna svojstvena vrijednost od $\mathbf{B}$

\section{Metapopulacijski modeli}

\subsection{Uvod u metapopulacijsku ekologiju}

Metapopulacija je skup lokalnih populacija povezanih migracijom. Koncept je uveo Richard Levins 1969. za opisivanje dinamike populacija u fragmentiranim staništima. U metapopulacijskim modelima lokalne populacije mogu izumrijeti, ali ponovno se mogu uspostaviti kolonizacijom iz drugih lokalnih populacija.

\subsubsection{Ključni koncepti}

\textbf{Lokalne populacije:} Grupe jedinki u diskretnim fragmentima staništa
\textbf{Migracija:} Kretanje jedinki između lokalnih populacija
\textbf{Kolonizacija:} Uspostavljanje novih lokalnih populacija
\textbf{Lokalno izumiranje:} Nestanak lokalnih populacija
\textbf{Povezanost:} Mogućnost migracije između fragmenata

\subsection{Levinsov klasični model}

\subsubsection{Osnovni okvir}

Levinsov model tretira dinamiku udjela okupiranih staništa umjesto praćenja brojnosti jedinki. Neka je $p(t)$ udio okupiranih fragmenata u vremenu $t$.

\textbf{Osnovna jednadžba:}
\begin{equation}
	\frac{dp}{dt} = cp(1-p) - ep
\end{equation}

gdje su:
\begin{itemize}
	\item $c$ - stopa kolonizacije per okupiranog fragmenta
	\item $e$ - stopa lokalnog izumiranja per okupiranog fragmenta
	\item $p$ - udio okupiranih fragmenata
	\item $(1-p)$ - udio neokupiranih fragmenata dostupnih za kolonizaciju
\end{itemize}

\subsubsection{Analiza Levinsovog modela}

\textbf{Ravnotežne točke:}

Postavljanjem $dp/dt = 0$:
\begin{align}
	p^* &= 0 \quad \text{(uvijek postoji)} \\
	p^* &= 1 - \frac{e}{c} \quad \text{(ako } c > e\text{)}
\end{align}

\textbf{Uvjeti za opstanak metapopulacije:}
\begin{equation}
	c > e \quad \text{ili} \quad \frac{c}{e} > 1
\end{equation}

Omjer $c/e$ naziva se \textbf{osnovni reprodukcijski broj metapopulacije} $R_0$.

\textbf{Stabilnost ravnoteže:}

Linearizacija oko $p^* = 1 - e/c$:
\begin{equation}
	\frac{d}{dt}(p - p^*) = -e(p - p^*)
\end{equation}

Ravnoteža je stabilna jer je $e > 0$.

\subsubsection{Dinamika Levinsovog modela}

\textbf{Rješenje diferencijalne jednadžbe:}

Za $c > e$:
\begin{equation}
	p(t) = \frac{(c-e)p_0}{c-e + ep_0 + (c-e-cp_0)e^{-(c-e)t}}
\end{equation}

gdje je $p_0 = p(0)$ početni udio okupiranih fragmenata.

\textbf{Asimptotsko ponašanje:}
\begin{align}
	\lim_{t \to \infty} p(t) &= 1 - \frac{e}{c} \quad \text{(ako } c > e\text{)} \\
	\lim_{t \to \infty} p(t) &= 0 \quad \text{(ako } c \leq e\text{)}
\end{align}

\subsection{Prostorno eksplicitni metapopulacijski modeli}

\subsubsection{Model s prostornim rasporedom}

Prostorna udaljenost utječe na vjerojatnost kolonizacije. Za $n$ fragmenata:

\begin{equation}
	\frac{dS_i}{dt} = C_i(1-S_i) - E_i S_i
\end{equation}

gdje je $S_i$ vjerojatnost da je fragment $i$ okupiran.

\textbf{Stopa kolonizacije fragmenta $i$:}
\begin{equation}
	C_i = \sum_{j \neq i} c_{ij} S_j
\end{equation}

gdje je $c_{ij}$ stopa kolonizacije fragmenta $i$ iz fragmenta $j$.

\textbf{Funkcija dispersije:}

Najčešće se koristi eksponencijalni kernel:
\begin{equation}
	c_{ij} = c_0 e^{-d_{ij}/\alpha}
\end{equation}

gdje su:
\begin{itemize}
	\item $d_{ij}$ - udaljenost između fragmenata $i$ i $j$
	\item $\alpha$ - srednja udaljenost dispersije
	\item $c_0$ - bazalna stopa kolonizacije
\end{itemize}

\subsubsection{Metapopulacijska kapaciteta}

Hanski je uveo koncept metapopulacijske kapacitete $\lambda_M$:

\begin{equation}
	\lambda_M = \text{dominantna svojstvena vrijednost od } \mathbf{M}
\end{equation}

gdje je $\mathbf{M}$ matrica migracije s elementima:
\begin{equation}
	M_{ij} = \begin{cases}
		\frac{c_{ij}}{E_i} & \text{ako } i \neq j \\
		0 & \text{ako } i = j
	\end{cases}
\end{equation}

\textbf{Uvjet za opstanak:}
\begin{equation}
	\lambda_M > 1
\end{equation}

\subsection{Modeli s heterogenošću fragmenata}

\subsubsection{Varijacije u kvaliteti staništa}

Fragmenti mogu imati različite kapacitete, što utječe na stope izumiranja:

\begin{equation}
	E_i = \frac{e_0}{A_i^z}
\end{equation}

gdje su:
\begin{itemize}
	\item $A_i$ - površina fragmenta $i$
	\item $e_0$ - bazalna stopa izumiranja
	\item $z$ - eksponent površine (obično $z \approx 0.5$)
\end{itemize}

\textbf{Teorija biogeografije otoka:}

MacArthur-Wilson model povezuje veličinu i izoliranost:
\begin{align}
	C_i &= c_0 D_i^{-a} \\
	E_i &= e_0 A_i^{-z}
\end{align}

gdje je $D_i$ mjera izoliranosti fragmenta $i$.

\subsubsection{Rescue efekt}

Imigracija može spasiti lokalne populacije od izumiranja:

\begin{equation}
	E_i(I_i) = \frac{E_i^0}{1 + \beta I_i}
\end{equation}

gdje je $I_i$ stopa imigracije u fragment $i$.

\subsection{Diskretni metapopulacijski modeli}

\subsubsection{Diskretna verzija Levinsovog modela}

\begin{equation}
	p_{t+1} = p_t + cp_t(1-p_t) - ep_t
\end{equation}

ili:
\begin{equation}
	p_{t+1} = p_t(1 + c(1-p_t) - e)
\end{equation}

\textbf{Mapiranje:}
\begin{equation}
	p_{t+1} = p_t(1 + c - e - cp_t) = p_t(R_0 - cp_t)
\end{equation}

gdje je $R_0 = (c-e+1)$.

\subsubsection{Stohastički diskretni model}

Uključuje demografsku stohastičnost u malim lokalnim populacijama:

\begin{equation}
	P(S_{i,t+1} = 1) = S_{i,t}(1-E_i) + (1-S_{i,t})C_i
\end{equation}

gdje su $S_{i,t} \in \{0,1\}$ indikatori okupiranosti.

\subsection{Fragmentacija staništa}

\subsubsection{Utjecaj fragmentacije}

Fragmentacija utječe na metapopulacijsku dinamiku kroz:

\textbf{Smanjenje površine staništa:}
\begin{equation}
	H = \sum_{i=1}^{n} A_i
\end{equation}

\textbf{Povećanje izoliranosti:}
\begin{equation}
	I = \frac{1}{n(n-1)} \sum_{i=1}^{n} \sum_{j \neq i} d_{ij}
\end{equation}

\textbf{Smanjenje povezanosti:}
\begin{equation}
	\Gamma = \frac{1}{n} \sum_{i=1}^{n} \sum_{j \neq i} e^{-d_{ij}/\alpha}
\end{equation}

\subsubsection{Kritični prag fragmentacije}

Postoji kritična razina fragmentacije ispod koje metapopulacija ne može postojati:

\begin{equation}
	H_{\text{crit}} = \frac{e}{c} \cdot \frac{A_{\text{total}}}{\Gamma_{\text{max}}}
\end{equation}

\textbf{Perkolacijski model:}

Za regularne mreže, kritični prag perkolacije je:
\begin{itemize}
	\item 2D kvadratna mreža: $p_c \approx 0.593$
	\item 2D heksagonalna mreža: $p_c = 0.5$
	\item 2D trokutasta mreža: $p_c = 0.5$
\end{itemize}

\subsection{Empirijski primjer: Metapopulacija leptira}

\subsubsection{Melitaea cinxia na Ålandskim otocima}

Hanski i suradnici proučavali su metapopulaciju leptira \textit{Melitaea cinxia}:

\textbf{Empirijski model:}
\begin{equation}
	C_i = \frac{c \sum_{j \neq i} S_j e^{-d_{ij}/\alpha}}{1 + c \sum_{j \neq i} S_j e^{-d_{ij}/\alpha}}
\end{equation}

\begin{equation}
	E_i = \frac{e_0}{A_i^z}
\end{equation}

\textbf{Procijenjeni parametri:}
\begin{itemize}
	\item $\alpha \approx 2$ km (srednja udaljenost dispersije)
	\item $z \approx 0.5$ (eksponent površine-izumiranje)
	\item $c \approx 1.5$ (kolonizacijski parametar)
	\item $e_0 \approx 0.3$ (bazalna stopa izumiranja)
\end{itemize}

\subsubsection{Prediktivni uspjeh}

Model je uspješno predvidio:
\begin{itemize}
	\item 70\% točnost za kolonizaciju
	\item 85\% točnost za izumiranje
	\item Dugoročne trendove populacije
\end{itemize}

\subsection{Praktične primjene metapopulacijskih modela}

\subsubsection{Dizajn rezervata}

\textbf{SLOSS debata:} Single Large or Several Small

Metapopulacijski modeli pokazuju da optimalni dizajn ovisi o:
\begin{itemize}
	\item Dispersijskim mogućnostima vrste
	\item Koreleiranosti lokalnih izumiranja
	\item Trade-off između veličine i broja fragmenata
\end{itemize}

\textbf{Optimizacijski problem:}
\begin{equation}
	\max_{\{A_i\}} \lambda_M \quad \text{s.t.} \quad \sum A_i \leq A_{\text{total}}
\end{equation}

\subsubsection{Minimalna životna populacija}

Za metapopulacije, MVP se definira kao:

\begin{equation}
	\text{MVP} = \min\{N : P(\text{izumiranje u } T \text{ godina}) < \alpha\}
\end{equation}

gdje se uključuje i prostorna komponenta rizika.

\textbf{Prostorno eksplicitni MVP:}
\begin{equation}
	N_{\text{MVP}} = k \cdot \frac{H_{\text{min}}}{\bar{A}} \cdot \bar{n}
\end{equation}

gdje su:
\begin{itemize}
	\item $H_{\min}$ - minimalna površina staništa
	\item $\bar{A}$ - srednja površina fragmenta
	\item $\bar{n}$ - srednja veličina lokalne populacije
	\item $k$ - sigurnosni faktor (obično $k = 2-5$)
\end{itemize}
	
\chapter{Međuspecijske interakcije i modeli zajednica}

\section{Grabljivac--plijen i funkcijski odgovori}

\subsection{Uvod u predator-prey dinamiku}

Interakcije između grabljivaca i plijena temeljne su za razumijevanje dinamike ekoloških zajednica. Ove interakcije karakteriziraju se složenim povratnim spregama: povećanje broja plijena omogućava rast populacije grabljivaca, što dovodi do smanjenja populacije plijena, što zauzvrat smanjuje populaciju grabljivaca.

\subsubsection{Klasifikacija trofnih interakcija}

\begin{longtable}{|p{2.8cm}|p{1.5cm}|p{1.5cm}|p{5.5cm}|}
	\caption{Klasifikacija međuspecijskih interakcija} \\
	\hline
	\textbf{Interakcija} & \textbf{Sp. 1} & \textbf{Sp. 2} & \textbf{Primjer} \\
	\hline
	\endfirsthead
	\multicolumn{4}{c}{\textbf{Tablica} -- nastavak} \\
	\hline
	\textbf{Interakcija} & \textbf{Sp. 1} & \textbf{Sp. 2} & \textbf{Primjer} \\
	\hline
	\endhead
	\hline
	\endfoot
	\hline
	\endlastfoot
	
	Predacija & $+$ & $-$ & Lav-zebra \\
	\hline
	Parazitizam & $+$ & $-$ & Virus-domaćin \\
	\hline
	Mutualism & $+$ & $+$ & Pčela-cvijet \\
	\hline
	Komenzalizam & $+$ & $0$ & Morski pas-riba \\
	\hline
	Amenzalizam & $-$ & $0$ & Alelopatija \\
	\hline
	Konkurencija & $-$ & $-$ & Dva predatora \\
	\hline
	Neutralizam & $0$ & $0$ & Udaljene vrste \\
	\hline
\end{longtable}

\subsection{Lotka-Volterra modeli}

\subsubsection{Osnovni Lotka-Volterra model}

Alfredo Lotka (1925) i Vito Volterra (1926) neovisno su razvili prvi matematički model predator-prey dinamike. Model pretpostavlja:

\begin{itemize}
	\item Eksponencijalni rast plijena u odsutnosti predatora
	\item Stopa predacije proporcionalna umnošku abundancija
	\item Smrt predatora exponencijalna u odsutnosti plijena
	\item Konverzijska efikasnost konstanta
\end{itemize}

\textbf{Osnovni sustav jednadžbi:}

\begin{align}
	\frac{dN}{dt} &= rN - aNP \label{eq:lv-prey} \\
	\frac{dP}{dt} &= eaNP - mP \label{eq:lv-predator}
\end{align}

gdje su:
\begin{itemize}
	\item $N(t)$ - gustoća populacije plijena
	\item $P(t)$ - gustoća populacije predatora  
	\item $r$ - intrinsična stopa rasta plijena
	\item $a$ - stopa napada predatora
	\item $e$ - efikasnost konverzije (0 < e < 1)
	\item $m$ - stopa smrti predatora
\end{itemize}

\subsubsection{Analiza ravnotežnih točaka}

\textbf{Nullclines:}

Postavljanjem derivacija na nulu:

\begin{align}
	\frac{dN}{dt} = 0 &: \quad N(r - aP) = 0 \\
	\frac{dP}{dt} = 0 &: \quad P(eaN - m) = 0
\end{align}

\textbf{Ravnotežne točke:}

\begin{align}
	&\text{Trivijalna: } (N^*, P^*) = (0, 0) \\
	&\text{Koegzistencijska: } (N^*, P^*) = \left(\frac{m}{ea}, \frac{r}{a}\right)
\end{align}

\subsubsection{Linearna stabilnostna analiza}

Jacobian matrica oko koegzistencijske ravnoteže:

\begin{equation}
	\mathbf{J} = \begin{pmatrix}
		0 & -aN^* \\
		eaP^* & 0
	\end{pmatrix} = \begin{pmatrix}
		0 & -m/e \\
		er & 0
	\end{pmatrix}
\end{equation}

\textbf{Karakteristična jednadžba:}
\begin{equation}
	\det(\mathbf{J} - \lambda\mathbf{I}) = \lambda^2 + \frac{mer}{e} = \lambda^2 + mr = 0
\end{equation}

\textbf{Svojstvene vrijednosti:}
\begin{equation}
	\lambda_{1,2} = \pm i\sqrt{mr}
\end{equation}

Čisto imaginarni svojstvene vrijednosti označavaju \textbf{neutralno stabilnu ravnotežu} s periodičnim oscilacijama.

\subsubsection{Konzervirana količina}

Lotka-Volterra sustav ima konzerviraną količinu (Hamiltonijan):

\begin{equation}
	H(N, P) = eaN - m\ln(N) + aP - r\ln(P) = \text{konstanta}
\end{equation}

\textbf{Izvod:}
\begin{align}
	\frac{dH}{dt} &= \frac{\partial H}{\partial N}\frac{dN}{dt} + \frac{\partial H}{\partial P}\frac{dP}{dt} \\
	&= \left(ea - \frac{m}{N}\right)(rN - aNP) + \left(a - \frac{r}{P}\right)(eaNP - mP) \\
	&= 0
\end{align}

\subsubsection{Perioda oscilacija}

Period oscilacija oko ravnoteže:

\begin{equation}
	T = \frac{2\pi}{\sqrt{mr}}
\end{equation}

\textbf{Amplitude oscilacija} ovise o početnim uvjetima i određene su konzerviranom količinom.

\subsection{Funkcijski odgovori}

\subsubsection{Definicija i tipovi}

Funkcijski odgovor opisuje vezu između gustoće plijena i stope konzumacije po predatoru. C.S. Holling (1959) identificirao je tri osnovna tipa:

\textbf{Tip I - Linearan:}
\begin{equation}
	f(N) = aN
\end{equation}

\textbf{Tip II - Hiperbolički (Michaelis-Menten):}
\begin{equation}
	f(N) = \frac{aN}{1 + ahN}
\end{equation}

\textbf{Tip III - Sigmoidalni:}
\begin{equation}
	f(N) = \frac{aN^2}{1 + ahN^2}
\end{equation}

gdje je $h$ handling time (vrijeme potrebno za hvatanje i konzumaciju jedne jedinke plijena).

\subsubsection{Izvod Tip II funkcijskog odgovora}

\textbf{Pretpostavke:}
\begin{itemize}
	\item Ukupno vrijeme = traženja + handling
	\item $T = T_s + T_h$
	\item Stopa susreta proporcionalna $N$: attack rate = $a$
	\item Handling time po plijenu: $h$
\end{itemize}

\textbf{Izvod:}

Broj uhvaćenih jedinki plijena u vremenu $T$:
\begin{equation}
	N_e = aT_s N
\end{equation}

Vrijeme handling:
\begin{equation}
	T_h = hN_e
\end{equation}

Ukupno vrijeme:
\begin{equation}
	T = T_s + hN_e = T_s + haT_s N
\end{equation}

Rješavanjem za $T_s$:
\begin{equation}
	T_s = \frac{T}{1 + haN}
\end{equation}

Stopa konzumacije:
\begin{equation}
	f(N) = \frac{N_e}{T} = \frac{aT_s N}{T} = \frac{aN}{1 + ahN}
\end{equation}

\subsubsection{Svojstva funkcijskih odgovora}

\textbf{Tip I:}
\begin{itemize}
	\item Linearna veza do zasićenja
	\item $f'(N) = a > 0$ (konstanta)
	\item Nema handling time ograničenja
	\item Rijetko u prirodi
\end{itemize}

\textbf{Tip II:}
\begin{itemize}
	\item $\lim_{N \to \infty} f(N) = 1/h$ (maksimalna stopa)
	\item $f'(N) = \frac{a}{(1 + ahN)^2} > 0$ (uvijek pozitivna derivacija)
	\item $f''(N) = \frac{-2a^2hN}{(1 + ahN)^3} < 0$ (konkavna)
	\item Može destabilizirati predator-prey dinamiku
\end{itemize}

\textbf{Tip III:}
\begin{itemize}
	\item S-oblik s infleksijskim točkom
	\item $f'(0) = 0$ (nulta derivacija kod $N = 0$)
	\item Infleksijska točka kod $N^* = 1/(\sqrt{ah})$
	\item Omogućava refuge za plijen pri niskim gustoćama
	\item Stabilizira predator-prey dinamiku
\end{itemize}

\subsection{Modifikacije Lotka-Volterra modela}

\subsubsection{Model s Tip II funkcijskim odgovorom}

\begin{align}
	\frac{dN}{dt} &= rN - \frac{aNP}{1 + ahN} \\
	\frac{dP}{dt} &= \frac{eaNP}{1 + ahN} - mP
\end{align}

\textbf{Ravnotežne točke:}

Koegzistencijska ravnoteža:
\begin{align}
	N^* &= \frac{m}{ea - mah} \\
	P^* &= \frac{r(1 + ahN^*)}{a}
\end{align}

\textbf{Uvjet za postojanje:}
\begin{equation}
	ea > mah \quad \text{ili} \quad \frac{e}{h} > m
\end{equation}

\subsubsection{Rosenzweig-MacArthur model}

Dodavanje logističkog rasta plijena:

\begin{align}
	\frac{dN}{dt} &= rN\left(1 - \frac{N}{K}\right) - \frac{aNP}{1 + ahN} \\
	\frac{dP}{dt} &= \frac{eaNP}{1 + ahN} - mP
\end{align}

\textbf{Paradoks obogaćivanja:}

Povećanje $K$ (nosivost staništa) može destabilizirati sustav i dovesti do ekstremnih oscilacija ili izumiranja predatora.

\subsubsection{Model s Tip III funkcijskim odgovorom}

\begin{align}
	\frac{dN}{dt} &= rN - \frac{aN^2P}{1 + ahN^2} \\
	\frac{dP}{dt} &= \frac{eaN^2P}{1 + ahN^2} - mP
\end{align}

\textbf{Stabiliziranje svojstvo:}

Tip III funkcijski odgovor stvara \textbf{refuge efekt} za plijen pri niskim gustoćama, što stabilizira dinamiku.

\subsection{Prostorni predator-prey modeli}

\subsubsection{Reakcijska-difuzija modeli}

\begin{align}
	\frac{\partial N}{\partial t} &= rN - aNP + D_N \nabla^2 N \\
	\frac{\partial P}{\partial t} &= eaNP - mP + D_P \nabla^2 P
\end{align}

gdje su $D_N$ i $D_P$ difuzijski koeficijenti.

\subsubsection{Prostorna stabilizacija}

Prostorna heterogenost može stabilizirati inače nestabilne predator-prey dinamike kroz:

\begin{itemize}
	\item \textbf{Asinkrone oscilacije} između lokaliteta
	\item \textbf{Source-sink dinamiku}
	\item \textbf{Prostorne refugije} za plijen
\end{itemize}

\section{Konkurencija i koegzistencija}

\subsection{Teorija konkurencije}

\subsubsection{Tipovi konkurencije}

\textbf{Interference konkurencija:} Direktne agresivne interakcije

\textbf{Exploitation konkurencija:} Konkurencija za ograničene resurse

\textbf{Apparent konkurencija:} Posredovana zajedničkim predatorima

\subsubsection{Gause-ov princip}

G.F. Gause (1934) postulirao je \textbf{competitive exclusion principle}:

\textit{``Dvije vrste s identičnim ekološkim potrebama ne mogu dugoročno koegzistirati u istom staništu.''}

\subsection{Lotka-Volterra model konkurencije}

\subsubsection{Dvospecijski model}

\begin{align}
	\frac{dN_1}{dt} &= r_1 N_1 \left(1 - \frac{N_1 + \alpha_{12}N_2}{K_1}\right) \\
	\frac{dN_2}{dt} &= r_2 N_2 \left(1 - \frac{N_2 + \alpha_{21}N_1}{K_2}\right)
\end{align}

gdje su:
\begin{itemize}
	\item $\alpha_{12}$ - utjecaj vrste 2 na vrstu 1 (competition coefficient)
	\item $\alpha_{21}$ - utjecaj vrste 1 na vrstu 2  
	\item $K_1, K_2$ - nosivosti staništa kada su vrste same
\end{itemize}

\subsubsection{Isokline i fazni portret}

\textbf{Nullclines (isokline):}

\begin{align}
	\frac{dN_1}{dt} = 0 &: \quad N_1 = K_1 - \alpha_{12}N_2 \\
	\frac{dN_2}{dt} = 0 &: \quad N_2 = K_2 - \alpha_{21}N_1
\end{align}

\textbf{Ravnotežne točke:}

\begin{enumerate}
	\item $(0, 0)$ - nestabilna
	\item $(K_1, 0)$ - vrste 1 pobjeđuje
	\item $(0, K_2)$ - vrste 2 pobjeđuje  
	\item $(N_1^*, N_2^*)$ - koegzistencija (ako postoji)
\end{enumerate}

\textbf{Koegzistencijska ravnoteža:}
\begin{align}
	N_1^* &= \frac{K_1 - \alpha_{12}K_2}{1 - \alpha_{12}\alpha_{21}} \\
	N_2^* &= \frac{K_2 - \alpha_{21}K_1}{1 - \alpha_{12}\alpha_{21}}
\end{align}

\subsubsection{Uvjeti za koegzistenciju}

\textbf{Uvjet postojanja:}
\begin{equation}
	N_1^*, N_2^* > 0
\end{equation}

To zahtijeva:
\begin{align}
	K_1 &> \alpha_{12}K_2 \\
	K_2 &> \alpha_{21}K_1
\end{align}

\textbf{Uvjet stabilnosti:}

Jacobian oko koegzistencijske ravnoteže:
\begin{equation}
	\mathbf{J} = \begin{pmatrix}
		-\frac{r_1 N_1^*}{K_1} & -\frac{r_1 \alpha_{12} N_1^*}{K_1} \\
		-\frac{r_2 \alpha_{21} N_2^*}{K_2} & -\frac{r_2 N_2^*}{K_2}
	\end{pmatrix}
\end{equation}

\textbf{Uvjeti stabilnosti (Routh-Hurwitz):}
\begin{align}
	\text{tr}(\mathbf{J}) &< 0 \quad \text{(uvijek zadovoljeno)} \\
	\det(\mathbf{J}) &> 0
\end{align}

Uvjet $\det(\mathbf{J}) > 0$ ekvivalentan je:
\begin{equation}
	\alpha_{12}\alpha_{21} < 1
\end{equation}

\subsubsection{Scenariji konkurencije}

\textbf{1. Stabilna koegzistencija:}
\begin{equation}
	\alpha_{12} < \frac{K_1}{K_2} \quad \text{i} \quad \alpha_{21} < \frac{K_2}{K_1}
\end{equation}

\textbf{2. Vrste 1 uvijek pobjeđuje:}
\begin{equation}
	\alpha_{12} > \frac{K_1}{K_2} \quad \text{i} \quad \alpha_{21} < \frac{K_2}{K_1}
\end{equation}

\textbf{3. Vrste 2 uvijek pobjeđuje:}
\begin{equation}
	\alpha_{12} < \frac{K_1}{K_2} \quad \text{i} \quad \alpha_{21} > \frac{K_2}{K_1}
\end{equation}

\textbf{4. Prioritetni efekt (bistabilnost):}
\begin{equation}
	\alpha_{12} > \frac{K_1}{K_2} \quad \text{i} \quad \alpha_{21} > \frac{K_2}{K_1}
\end{equation}

\subsection{Moderna teorija koegzistencije}

\subsubsection{Chesson okvir}

Peter Chesson razvio je sveobuhvatan okvir koji dijeli mehanizme koegzistencije na:

\textbf{Stabilizarne mehanizme:} Povećavaju negativnu povratnu spregu
\textbf{Equaliziranje mehanizmi:} Smanjuju fitness razlike između vrsta

\subsubsection{Invazijski rast}

Vrsta $i$ može invadirati zajednicu ako je njena dugoročna stopa rasta pozitivna kada je rijetka:

\begin{equation}
	\bar{r}_i = r_i - \sum_{j \neq i} \alpha_{ij} \bar{N}_j > 0
\end{equation}

\textbf{Recipročna invazivnost} - uvjet za stabilnu koegzistenciju.

\subsubsection{Modern coexistence mechanisms}

\textbf{1. Storage efekt:}
\begin{equation}
	\text{Kovarijanca}(\text{rast}, \text{konkurencija}) < 0
\end{equation}

Koegzistencija zbog vremenskih varijacija u uvjetima.

\textbf{2. Speed-accuracy trade-off:}

Kompromis između brzine rasta i efikasnosti korištenja resursa.

\textbf{3. Relative nonlinearity:}

Nelinearne interakcije koje pogoduju vrsti kada je rijetka.

\subsection{Resource partitioning i niche teorija}

\subsubsection{MacArthur-ov model resursa}

\begin{equation}
	\frac{dN_i}{dt} = N_i \left(\sum_{j=1}^{m} c_{ij} R_j - d_i\right)
\end{equation}

\begin{equation}
	\frac{dR_j}{dt} = S_j - \sum_{i=1}^{n} c_{ij} N_i R_j
\end{equation}

gdje su:
\begin{itemize}
	\item $R_j$ - abundancija resursa $j$
	\item $c_{ij}$ - consumption rate vrste $i$ na resurs $j$
	\item $S_j$ - supply rate resursa $j$
	\item $d_i$ - death rate vrste $i$
\end{itemize}

\subsubsection{Limiting similarity}

MacArthur i Levins (1967) pokazali su da postoji minimalna razlika između niša potrebna za koegzistenciju.

\textbf{Gaussian utiliziranje resursa:}
\begin{equation}
	u_i(x) = \exp\left(-\frac{(x - \mu_i)^2}{2\sigma^2}\right)
\end{equation}

\textbf{Uvjet koegzistencije:}
\begin{equation}
	\frac{|\mu_1 - \mu_2|}{\sigma} > \sigma_{\text{min}}
\end{equation}

gdje je $\sigma_{\text{min}} \approx 1$ za Gaussian niše.

\subsubsection{Character displacement}

Evolucijski odgovor na konkurenciju koji povećava razlike u trait-ima.

\textbf{Quantitative genetski model:}
\begin{equation}
	\frac{d\bar{z}_i}{dt} = h_i^2 \frac{\partial \bar{w}_i}{\partial \bar{z}_i}
\end{equation}

gdje je $h_i^2$ heritabilnost, $\bar{z}_i$ srednji trait, i $\bar{w}_i$ fitness.

\section{Ekološke mreže i stabilnost zajednica}

\subsection{Uvod u ekološke mreže}

Ekološke zajednice najbolje se razumiju kao složene mreže interakcija između vrsta. Mrežna analiza omogućava kvantificiranje strukture zajednica i ispitivanje veze između kompleksnosti i stabilnosti.

\subsubsection{Tipovi ekoloških mreža}

\textbf{Food webs:} Tko koga jede (trofne interakcije)
\textbf{Mutualistička mreže:} Beneficial interakcije (oprašivanje, rasipanje sjemenki)
\textbf{Host-parazitska mreže:} Parazitsko-host veze
\textbf{Mreže za čišćenje:} Cleaning symbioses

\subsection{Topologija hranidbenih mreža}

\subsubsection{Osnovni parametri mreže}

\textbf{Connectance (C):}
\begin{equation}
	C = \frac{L}{S^2}
\end{equation}

gdje je $L$ broj veza, $S$ broj vrsta.

\textbf{Link density:}
\begin{equation}
	\frac{L}{S}
\end{equation}

\textbf{Degree distribucija:}
\begin{equation}
	P(k) = \text{vjerojatnost da vrsta ima } k \text{ veza}
\end{equation}

\subsubsection{Trofni nivoi}

\textbf{Shortweighted trophic level:}
\begin{align}
	\text{TL}_i &= 1 + \frac{\sum_j \text{TL}_j \cdot w_{ji}}{\sum_j w_{ji}}
\end{align}

gdje je $w_{ji}$ težina veze od vrste $j$ do vrste $i$.

\textbf{Omnivory index:}
\begin{align}
	\text{OI}_i &= \sum_j (\text{TL}_j - \text{TL}_i + 1)^2 \cdot \frac{w_{ji}}{\sum_k w_{ki}}
\end{align}

\subsubsection{Modularnost}

Newman-ova modularnost:
\begin{align}
	Q &= \frac{1}{2m} \sum_{ij} \left(A_{ij} - \frac{k_i k_j}{2m}\right) \delta(c_i, c_j)
\end{align}

gdje su:
\begin{itemize}
	\item $A_{ij}$ - adjacency matrica
	\item $k_i$ - degree čvor $i$ 
	\item $m$ - ukupni broj veza
	\item $c_i$ - zajednica čvor $i$
\item $\delta(c_i, c_j) = 1$ ako $c_i = c_j$, inače $0$
\end{itemize}

\subsection{May-ova analiza stabilnosti}

\subsubsection{Random matrix pristup}

Robert May (1972) koristio je random matrix teoriju za analizu stabilnosti velikih ekoloških mreža.

\textbf{Model:}
\begin{equation}
	\frac{dx_i}{dt} = x_i \left(r_i + \sum_{j=1}^{S} a_{ij} x_j\right)
\end{equation}

Linearizacija oko ravnoteže:
\begin{equation}
	\frac{d\xi_i}{dt} = \sum_{j=1}^{S} A_{ij} \xi_j
\end{equation}

gdje je $A_{ij} = a_{ij} x_j^*$.

\subsubsection{Random matrica svojstva}

\textbf{Elementi community matrice:}
\begin{itemize}
	\item $A_{ii} = -1$ (self-regulation)
	\item $A_{ij} \sim N(0, \sigma^2)$ with probability $C$
	\item $A_{ij} = 0$ with probability $(1-C)$
\end{itemize}

\textbf{May-ov stabilnostni uvjet:}

Za velike mreže, uvjet stabilnosti je:
\begin{equation}
	\sigma \sqrt{SC} < 1
\end{equation}

ili ekvivalentno:
\begin{equation}
	\sigma \sqrt{L} < 1
\end{equation}

\subsubsection{Interpretacija May-ovog rezultata}

May-ov rezultat sugerira da kompleksnost (veći $S$ ili $C$) destabilizira zajednice, što je kontradiktorno empirijskim opažanjima.

\textbf{Paradoks složenosti-stabilnosti:}

- Prirodne zajednice su složene i stabilne
- Random modeli predviđaju da složenost destabilizira
- Razlika nastaje zbog struktura koja nije random

\subsection{Strukturni aspekti stabilnosti}

\subsubsection{Trofna koherentnost}

Kramer i suradnici pokazali su da trofno koherentne mreže (s jasnom hijerarhijskom strukturom) mogu biti stabilne unatoč velikoj kompleksnosti.

\textbf{Trofna koherentnost:}
\begin{equation}
	q = \frac{1}{L} \sum_{i \to j} |h_i - h_j - 1|^2
\end{equation}

gdje je $h_i$ trofni nivo vrste $i$.

\subsubsection{Modularna struktura}

Zajednice organizirane u module stabilnije su od random mreža.

\textbf{Within-module vs. between-module veze:}
\begin{equation}
	\frac{\sigma_{\text{within}}}{\sigma_{\text{between}}}
\end{equation}

\subsection{Kaskadni učinci}

\subsubsection{Trofni kaskadni učinci}

Promjene na vrhu hranidbene mreže mogu se propagirati prema dolje kroz trofne nivoe.

\textbf{Klasični trofni kaskadni učinak:}
\begin{equation}
	\text{Top predatori} \uparrow \Rightarrow \text{Mesopredatori} \downarrow \Rightarrow \text{Herbivori} \uparrow \Rightarrow \text{Primarni proizvođači} \downarrow
\end{equation}

\subsubsection{Sekundarni izumiranje}

Izumiranje jedne vrste može dovesti do kaskadnih izumiranja.

\textbf{Robusnost mreže:}
\begin{equation}
	R = \frac{S - S_{\text{extinct}}}{S}
\end{equation}

gdje je $S_{\text{extinct}}$ broj vrsta koje izumiru nakon perturbacije.

\subsubsection{Keystone vrste}

Vrste čije uklanjanje ima disproportionalno veliki utjecaj na strukturu zajednice.

\textbf{Keystoneness index:}
\begin{equation}
	K_i = \frac{\Delta D_i}{\Delta B_i}
\end{equation}

gdje je $\Delta D_i$ promjena u raznolikosti, $\Delta B_i$ promjena u biomasi vrste $i$.

\subsection{Dinamička analiza mreža}

\subsubsection{Perturbation analiza}

\textbf{Press perturbation:}

Trajna promjena u parametru:
\begin{equation}
	\Delta \mathbf{x}^* = -\mathbf{A}^{-1} \Delta \mathbf{r}
\end{equation}

\textbf{Pulse perturbation:}

Trenutačna promjena u gustoći:
\begin{equation}
	\mathbf{x}(t) = e^{\mathbf{A}t} \mathbf{x}(0)
\end{equation}

\subsubsection{Return time}

Vrijeme potrebno za povratak na ravnotežu nakon perturbacije:
\begin{equation}
	T_{\text{return}} = -\frac{1}{\text{Re}(\lambda_{\max})}
\end{equation}

gdje je $\lambda_{\max}$ najveća (najmanje negativna) svojstvena vrijednost.

\subsection{Empirijski primjeri}

\subsubsection{Yellowstone vukovi}

Reintrodukcija vukova u Yellowstone (1995) dovela je do:

\begin{itemize}
	\item Smanjenja populacije jelena
	\item Oporavka vegetacije (vrbe, topole)
	\item Povratka bobera
	\item Promjena tijeka rijeka (zbog vegetacije)
\end{itemize}

\textbf{Trofomorfic cascades:} Predatori oblikuju fizičku strukturu ekosustava.

\subsubsection{Morska vidra i kelp šume}

\textbf{Interakcijski lanac:}
\begin{equation}
	\text{Morska vidra} \to \text{Morski ježevi} \to \text{Kelp alge}
\end{equation}

Gubitak morskih vidara doveo je do:
- Eksplozije populacije morskih ježeva  
- Degradacije kelp šuma
- Gubitka staništa za mnoge vrste

\subsection{Mrežne metrike i bioraznolikost}

\subsubsection{Species importance indices}

\textbf{Betweenness centrality:}
\begin{equation}
	g_i = \sum_{s \neq t} \frac{\sigma_{st}(i)}{\sigma_{st}}
\end{equation}

gdje je $\sigma_{st}(i)$ broj najkraćih putanja između $s$ i $t$ koji prolaze kroz $i$.

\textbf{Closeness centrality:}
\begin{equation}
	c_i = \frac{1}{\sum_{j \neq i} d_{ij}}
\end{equation}

gdje je $d_{ij}$ najkraća udaljenost između $i$ i $j$.

\subsubsection{Network motifs}

Mali subgrafovi koji se javljaju često:

\textbf{3-node motifs:}
\begin{itemize}
	\item Omnivory loop
	\item Apparent competition  
	\item Three-species chain
\end{itemize}

\textbf{Z-score:}
\begin{equation}
	Z = \frac{N_{\text{real}} - N_{\text{random}}}{\sigma_{\text{random}}}
\end{equation}

\subsection{Prostorno-eksplicitni mrežni modeli}

\subsubsection{Metahranidbene mreže}

Kombinacija metapopulacijske dinamike i trofnih interakcija:

\begin{equation}
	\frac{dS_{ij}}{dt} = C_{ij}(1 - S_{ij}) - E_{ij} S_{ij}
\end{equation}

gdje $S_{ij}$ označava okupiranost vrste $i$ u patch-u $j$.

\subsubsection{Prostorna stabilnost hranidbenih mreža}

Prostorna struktura može stabilizirati hranidbene mreže kroz:

\begin{itemize}
	\item \textbf{asinkronu dinamiku} između zakrpa staništa
	\item \textbf{rekolonizaciju} nakon lokalnih izumiranja
	\item \textbf{prostorne dotoke} između staništa
\end{itemize}

\textbf{Prostorni stabilizacijski uvjet:}
\begin{equation}
	\sigma \sqrt{SC} < \sqrt{1 + \frac{D}{\sigma^2}}
\end{equation}

gdje je $D$ stopa disperzije.

	
\chapter{Modeli ekosustava i bioenergetika}

\section{Kruženje tvari i energije}

\subsection{Osnovni principi i bilance tvari}

Ekosustavi funkcioniraju kao složeni sustavi koji obrađuju energiju i materijale prema fundamentalnim fizikalnim zakonima. Prvi zakon termodinamike nalaže da se energija ne može stvarati niti uništavati, već se samo može transformirati iz jednog oblika u drugi. Za bilo koji ekosustav možemo postaviti temeljnu jednadžbu bilance:

\begin{equation}
	\frac{dS}{dt} = I - O - T - D
\end{equation}

gdje je:
\begin{itemize}
	\item $S$ = zaliha tvari ili energije u sustavu
	\item $I$ = ulazni tok (input)
	\item $O$ = izlazni tok (output) 
	\item $T$ = transformacija unutar sustava
	\item $D$ = degradacija ili disipacija
\end{itemize}

\subsection{Biogeokemijski ciklusi}

\subsubsection{Ciklus ugljika}

Model ugljičnog ciklusa može se opisati sustavom diferencijalnih jednadžbi za različite rezervoare. Za jednostavan model s tri rezervoara (atmosfera, biomasa, tlo):

\begin{align}
	\frac{dC_a}{dt} &= \gamma \cdot R_{soil} + \gamma \cdot R_{bio} - GPP + F_{fossil}\\
	\frac{dC_{bio}}{dt} &= (1-\alpha) \cdot GPP - R_{bio} - M\\
	\frac{dC_{soil}}{dt} &= \alpha \cdot GPP + M - R_{soil}
\end{align}

gdje je:
\begin{itemize}
	\item $C_a, C_{bio}, C_{soil}$ = ugljik u atmosferi, biomasi i tlu
	\item $GPP$ = bruto primarna produktivnost (Gross Primary Productivity)
	\item $R_{bio}, R_{soil}$ = respiracija biomase i tla
	\item $\alpha$ = frakcija GPP koja ide u tlo
	\item $M$ = smrtnost biomase
	\item $\gamma$ = konverzijski faktor CO₂ ↔ C
	\item $F_{fossil}$ = emisije iz fosilnih goriva
\end{itemize}

Respiracija se često modelira kao funkcija temperature koristeći Q₁₀ odnos:

\begin{equation}
	R(T) = R_{ref} \cdot Q_{10}^{(T-T_{ref})/10}
\end{equation}

gdje je tipično $Q_{10} = 2-3$.

\subsubsection{Ciklus dušika}

Dušični ciklus uključuje multiple kemijske oblike. Osnovni model uključuje:

\begin{align}
	\frac{dNH_4^+}{dt} &= k_{min} \cdot N_{org} - k_{nit} \cdot NH_4^+ - U_{NH4}\\
	\frac{dNO_3^-}{dt} &= k_{nit} \cdot NH_4^+ - k_{denit} \cdot NO_3^- - U_{NO3}\\
	\frac{dN_{org}}{dt} &= -k_{min} \cdot N_{org} + I_{org} + D_{mort}
\end{align}

gdje su:
\begin{itemize}
	\item $k_{min}, k_{nit}, k_{denit}$ = konstante brzine mineralizacije, nitrifikacije i denitrifikacije
	\item $U_{NH4}, U_{NO3}$ = biljno usvajanje amonija i nitrata
	\item $I_{org}$ = ulaz organske tvari
	\item $D_{mort}$ = dušik iz smrtnosti biomase
\end{itemize}

\subsection{Energetski tokovi kroz ekosustav}

Energetski tok kroz trofične razine slijedi eksponencijalno slabljenje:

\begin{equation}
	E_n = E_0 \cdot \epsilon^n
\end{equation}

gdje je:
\begin{itemize}
	\item $E_n$ = energija na trofičnoj razini $n$
	\item $E_0$ = energija primarnih producenata
	\item $\epsilon$ = efikasnost transfera (tipično 0.1-0.2)
\end{itemize}

\subsubsection{Lindeman-ova piramida efikasnosti}

Efikasnost transfera energije između trofičnih razina:

\begin{equation}
	TE_n = \frac{P_n}{P_{n-1}} = \frac{A_n \cdot AE_n \cdot PE_n}{P_{n-1}}
\end{equation}

gdje je:
\begin{itemize}
	\item $TE_n$ = efikasnost transfera na razinu $n$
	\item $P_n$ = produkcija na razini $n$
	\item $A_n$ = asimilacija na razini $n$
	\item $AE_n$ = efikasnost asimilacije
	\item $PE_n$ = efikasnost produkcije
\end{itemize}

\subsection{Ohmov zakon za ekologiju}

Po analogiji s Ohmovim zakonom u elektrotehnici, možemo definirati "otpor" ekosustava prema toku tvari:

\begin{equation}
	J = \frac{\Delta C}{R_{eco}}
\end{equation}

gdje je:
\begin{itemize}
	\item $J$ = tok tvari
	\item $\Delta C$ = razlika koncentracija
	\item $R_{eco}$ = ekološki otpor
\end{itemize}

\section{Bioenergetski i DEB modeli}

\subsection{Osnove DEB teorije}

Dynamic Energy Budget (DEB) teorija pruža kvantitativni okvir za metabolizam organizma tijekom cijelog životnog ciklusa. Temelji se na nekoliko ključnih pretpostavki:

\begin{enumerate}
	\item \textbf{Kappa pravilo}: Energija se dijeli između rasta/održavanja ($\kappa$) i reprodukcije/sazrijevanja ($(1-\kappa)$)
	\item \textbf{Surface area law}: Brzina hranjenja proporcionalna je površini
	\item \textbf{Volume-specific somatic maintenance}: Održavanje je proporcionalno volumenu
\end{enumerate}

\subsubsection{Osnovne varijable stanja}

DEB model koristi tri varijable stanja:
\begin{itemize}
	\item $E$ = energija u rezervi (J)
	\item $V$ = strukturalni volumen (cm³)
	\item $E_H$ = energija uložena u sazrijevanje (J)
\end{itemize}

\subsubsection{Temeljne jednadžbe}

\textbf{Brzina hranjenja:}
\begin{equation}
	\dot{p}_X = \{p_{Xm}\} \cdot f \cdot V^{2/3}
\end{equation}

gdje je:
\begin{itemize}
	\item $\{p_{Xm}\}$ = maksimalna specifična brzina hranjenja
	\item $f$ = funkcionalni odgovor (0 ≤ f ≤ 1)
	\item $V^{2/3}$ = površina
\end{itemize}

\textbf{Asimilacija:}
\begin{equation}
	\dot{p}_A = \{p_{Am}\} \cdot f \cdot V^{2/3}
\end{equation}

\textbf{Mobilizacija iz rezerve:}
\begin{equation}
	\dot{p}_C = \frac{E \cdot \dot{v}}{V}
\end{equation}

gdje je $\dot{v}$ = energijska provodljivost.

\textbf{Kappa pravilo za alokaciju:}
\begin{align}
	\dot{p}_G &= \kappa \cdot \dot{p}_C - \dot{p}_M\\
	\dot{p}_R &= (1-\kappa) \cdot \dot{p}_C - \dot{p}_J
\end{align}

gdje su:
\begin{itemize}
	\item $\dot{p}_G$ = rast
	\item $\dot{p}_R$ = reprodukcija
	\item $\dot{p}_M$ = somatsko održavanje = $[\dot{p}_M] \cdot V$
	\item $\dot{p}_J$ = sazrijevanje
\end{itemize}

\subsubsection{Diferencijalne jednadžbe DEB modela}

\textbf{Rezerva energije:}
\begin{equation}
	\frac{dE}{dt} = \dot{p}_A - \dot{p}_C
\end{equation}

\textbf{Strukturalni volumen:}
\begin{equation}
	\frac{dV}{dt} = \frac{\dot{p}_G}{[E_G]}
\end{equation}

gdje je $[E_G]$ = specifična cijena strukture.

\textbf{Energija sazrijevanja:}
\begin{equation}
	\frac{dE_H}{dt} = \dot{p}_J = (1-\kappa) \cdot \dot{p}_C - \dot{p}_J
\end{equation}

\subsection{Standardni DEB model}

Za standardni DEB model možemo izvesti analitička rješenja u slučaju konstantnih uvjeta:

\textbf{Ultimativna duljina:}
\begin{equation}
	L_\infty = \frac{\{p_{Am}\} \cdot f}{[\dot{p}_M] \cdot \dot{v}} \cdot \delta_M
\end{equation}

\textbf{von Bertalanffy-jeva jednadžba rasta:}
\begin{equation}
	L(t) = L_\infty \cdot (1 - e^{-k \cdot t})
\end{equation}

gdje je von Bertalanffy-jeva konstanta:
\begin{equation}
	k = \frac{\dot{v}}{3 \cdot \delta_M \cdot L_\infty}
\end{equation}

\subsection{Funkcionalni odgovor u DEB kontekstu}

Funkcionalni odgovor povezuje gustoću hrane s brzinom hranjenja:

\begin{equation}
	f = \frac{X}{X + K}
\end{equation}

gdje je:
\begin{itemize}
	\item $X$ = gustoća hrane
	\item $K$ = polu-saturacijska konstanta
\end{itemize}

Za više tipova hrane:
\begin{equation}
	f_i = \frac{X_i}{X_i + K_i} \cdot \prod_{j \neq i} \frac{K_j}{X_j + K_j}
\end{equation}

\subsection{Povezivanje DEB modela s populacijskim dinamikama}

DEB modeli se mogu proširiti na populacijsku razinu kroz strukturirane populacijske modele:

\begin{equation}
	\frac{\partial n(a,t)}{\partial t} + \frac{\partial n(a,t)}{\partial a} = -\mu(a,E(a)) \cdot n(a,t)
\end{equation}

s rubnim uvjetom:
\begin{equation}
	n(0,t) = \int_0^\infty R(a,E(a)) \cdot n(a,t) \, da
\end{equation}

gdje su:
\begin{itemize}
	\item $n(a,t)$ = gustoća jedinki dobi $a$ u vremenu $t$
	\item $\mu(a,E(a))$ = stopa smrtnosti ovisna o dobi i energetskom stanju
	\item $R(a,E(a))$ = stopa reprodukcije
\end{itemize}

\section{Ecosystem services}

\subsection{Klasifikacija i kvantifikacija}

Usluge ekosustava dijele se u četiri glavne kategorije:

\begin{enumerate}
	\item \textbf{Pružateljske usluge} (hrana, voda, drvo)
	\item \textbf{Regulacijske usluge} (klimatska regulacija, čišćenje vode)
	\item \textbf{Kulturne usluge} (rekreacija, estetska vrijednost)
	\item \textbf{Potporne usluge} (kruženje hranjivih tvari, fotosinteza)
\end{enumerate}

\subsubsection{Kvantifikacija pomoću produkcijskih funkcija}

Opća forma produkcijske funkcije za uslugu ekosustava:

\begin{equation}
	ES = f(B_1, B_2, ..., B_n, C_1, C_2, ..., C_m)
\end{equation}

gdje su:
\begin{itemize}
	\item $ES$ = razina usluge ekosustava
	\item $B_i$ = biotički čimbenici
	\item $C_j$ = abiotički čimbenici
\end{itemize}

\textbf{Primjer - Sekvestracija ugljika:}
\begin{equation}
	C_{seq} = \alpha \cdot LAI \cdot PAR \cdot LUE \cdot (1 - R_a - R_h)
\end{equation}

gdje je:
\begin{itemize}
	\item $LAI$ = indeks listne površine
	\item $PAR$ = fotosintetski aktivno zračenje
	\item $LUE$ = efikasnost korištenja svjetlosti
	\item $R_a, R_h$ = autotrófna i heterotrófna respiracija
\end{itemize}

\subsection{Ekonomska vrednovanje}

\subsubsection{Metoda zamjenskih troškova}

Za uslugu čišćenja vode:
\begin{equation}
	V_{water} = C_{treatment} \cdot V_{water} \cdot E_{removal}
\end{equation}

gdje je:
\begin{itemize}
	\item $V_{water}$ = ekonomska vrijednost
	\item $C_{treatment}$ = trošak tehnološke obrade
	\item $V_{water}$ = volumen vode
	\item $E_{removal}$ = efikasnost uklanjanja onečišćenja
\end{itemize}

\subsubsection{Hedonistic pricing}

Cijena nekretnine kao funkcija usluga ekosustava:
\begin{equation}
	P = \alpha + \sum_{i} \beta_i X_i + \sum_{j} \gamma_j ES_j + \epsilon
\end{equation}

gdje su:
\begin{itemize}
	\item $P$ = cijena nekretnine
	\item $X_i$ = strukturne karakteristike
	\item $ES_j$ = usluge ekosustava
	\item $\gamma_j$ = marginalne vrijednosti usluga
\end{itemize}

\subsection{Trade-off analize}

\subsubsection{Pareto granica}

Za dvije usluge ekosustava možemo definirati Pareto granicu:
\begin{equation}
	ES_2^{max} = f(ES_1) \quad \text{subject to} \quad g(ES_1, ES_2) \leq 0
\end{equation}

\textbf{Optimizacijski problem:}
\begin{align}
	&\max \sum_{i} w_i \cdot ES_i\\
	&\text{subject to:} \quad \sum_{j} a_{ij} x_j \leq b_i
\end{align}

gdje su:
\begin{itemize}
	\item $w_i$ = težine usluga
	\item $x_j$ = upravljačke varijable
	\item $a_{ij}, b_i$ = ograničenja
\end{itemize}

\subsubsection{Elastičnost supstitucije}

Elastičnost supstitucije između dvije usluge:
\begin{equation}
	\sigma = \frac{d \ln(ES_2/ES_1)}{d \ln(MRS)}
\end{equation}

gdje je $MRS$ = marginalna stopa supstitucije.

\subsection{Prostorno modeliranje usluga}

\subsubsection{InVEST modeli}

\textbf{Model opskrbe vodom:}
\begin{equation}
	Y(x) = (1 - AET(x)/P(x)) \cdot P(x)
\end{equation}

gdje je:
\begin{itemize}
	\item $Y(x)$ = godišnja opskrba vodom u pikselu $x$
	\item $AET(x)$ = stvarna evapotranspiracija
	\item $P(x)$ = oborine
\end{itemize}

\textbf{Model polinizacije:}
\begin{equation}
	PS_j = \sum_{n} \frac{A_n}{1 + (D_{jn}/\alpha_n)^2} \cdot e^{-D_{jn}/\alpha_n}
\end{equation}

gdje je:
\begin{itemize}
	\item $PS_j$ = usluga polinizacije na lokaciji $j$
	\item $A_n$ = površina staništa $n$
	\item $D_{jn}$ = udaljenost između $j$ i $n$
	\item $\alpha_n$ = prosječna udaljenost leta za vrstu $n$
\end{itemize}

\subsection{Dinamičko modeliranje usluga}

Za usluge koje se mijenjaju kroz vrijeme:
\begin{equation}
	\frac{dES_i}{dt} = r_i \cdot ES_i \cdot \left(1 - \frac{ES_i}{K_i}\right) - h_i \cdot ES_i - \sum_{j \neq i} \alpha_{ij} \cdot ES_j
\end{equation}

gdje je:
\begin{itemize}
	\item $r_i$ = intrinsična stopa rasta usluge $i$
	\item $K_i$ = nosivost za uslugu $i$  
	\item $h_i$ = stopa žetve/korištenja
	\item $\alpha_{ij}$ = interakcijski koeficijenti
\end{itemize}

\subsubsection{Optimalno korištenje kroz vrijeme}

Hamiltonijan za optimalno korištenje:
\begin{equation}
	H = \sum_i \left[ U_i(h_i) + \lambda_i \left( f_i(ES_i) - h_i \right) \right]
\end{equation}

Prvi red uvjeti:
\begin{align}
	\frac{\partial U_i}{\partial h_i} &= \lambda_i\\
	\dot{\lambda_i} &= \rho \lambda_i - \lambda_i \frac{\partial f_i}{\partial ES_i}
\end{align}

gdje je $\rho$ = diskontna stopa.

Ovaj sveobuhvatan pristup modeliranju ekosustava i bioenergetike omogućuje kvantitativno razumijevanje složenih interakcija između biotičkih i abiotičkih komponenti, što je ključno za održivo upravljanje prirodnim resursima i uslugama ekosustava.
	
	% =====================================================
	\part{Prostor, predviđanje i AI}
	% =====================================================
	
\chapter{Prostorno-ekološko modeliranje}

Prostorno-ekološko modeliranje predstavlja jedan od najdinamičnijih i najbrže rastućih područja suvremene ekologije. Ovo interdisciplinarno polje spaja teoretsku ekologiju, geomatiku, statistiku i računalne znanosti kako bi razumjelo i predvidjelo prostorne obrasce distribucije organizama, njihove interakcije s okolišem te procese koji oblikuju bioraznolikost na različitim prostornim skalama.

\section{GIS, daljinska istraživanja i prostorne skale}

\subsection{Geografski informacijski sustavi (GIS) u ekologiji}

Geografski informacijski sustavi predstavljaju tehnološku osnovu prostorno-ekološkog modeliranja. GIS omogućuje pohranu, manipulaciju, analizu i vizualizaciju prostornih podataka, što je nezamjenjivo za razumijevanje ekoloških procesa koji se odvijaju u prostoru i vremenu.

\textbf{Osnovni principi GIS-a u ekologiji:}

\begin{enumerate}
	\item \textbf{Rasterski podatci}: Prostor je podijeljen u mrežu pravokutnih ćelija (piksela), gdje svaka ćelija nosi informaciju o određenoj varijabli (temperatura, padaline, nadmorska visina, tip vegetacije).
	
	\item \textbf{Vektorski podatci}: Prostorni objekti su predstavljeni kao točke (lokacije vrsta), linije (rijeke, ceste) ili poligoni (šume, zaštićena područja).
	
	\item \textbf{Prostorna rezolucija}: Određuje najmanji prostorni element koji može biti razlučen u analizi. Viša rezolucija znači manje ćelije i detaljniju informaciju, ali i veće računalne zahtjeve.
\end{enumerate}

\textbf{Matematički okvir prostornih podataka:}

Za rasterske podatke, prostorna lokacija $(i,j)$ u mreži odgovara geografskim koordinatama $(x,y)$:

\begin{align}
	x &= x_0 + i \times \Delta x \\
	y &= y_0 + j \times \Delta y
\end{align}

gdje su $x_0$ i $y_0$ koordinate ishodišta mreže, a $\Delta x$ i $\Delta y$ su prostorne rezolucije u $x$ i $y$ smjeru.

\subsection{Daljinska istraživanja i satelitski podatci}

Daljinska istraživanja pružaju kontinuirane vremenske serije podataka o stanju Zemljine površine, što je ključno za praćenje promjena u ekološkim sustavima.

\textbf{Glavni izvori satelitskih podataka:}

\begin{enumerate}
	\item \textbf{Landsat serija}: Prostorna rezolucija 15-120 m, vremenske serije od 1972. godine
	\item \textbf{MODIS (Moderate Resolution Imaging Spectroradiometer)}: Dnevni podatci globalne pokrivenosti, 250-1000 m rezolucija
	\item \textbf{Sentinel-2}: 10-60 m rezolucija, 5-dnevni ciklus ponovnog snimanja
\end{enumerate}

\textbf{Vegetacijski indeksi:}

Najčešće korišten je Normalizirani vegetacijski indeks (NDVI):

\begin{equation}
	\text{NDVI} = \frac{\text{NIR} - \text{Red}}{\text{NIR} + \text{Red}}
\end{equation}

gdje su NIR (near-infrared) i Red reflektance vrijednosti u odgovarajućim spektralskim pojasevima. NDVI vrijednosti kreću se od -1 do +1, pri čemu više vrijednosti označavaju gušću i zdraviju vegetaciju.

\subsection{Prostorne skale u ekologiji}

Konceptualna razina analize ključna je za razumijevanje ekoloških procesa jer se različiti procesi manifestiraju na različitim prostornim skalama.

\subsubsection{Lokalna skala (<1 km)}

Na lokalnoj skali dominiraju mikroklima i mikrostanišni uvjeti. Ova skala je relevantna za:

\begin{itemize}
	\item \textbf{Populacijsku dinamiku} malih organizama
	\item \textbf{Međuspecijske interakcije} (konkurencija, predacija)
	\item \textbf{Stanišne preferencije} pojedinačnih vrsta
\end{itemize}

\textbf{Matematički pristup lokalnoj skali:}

Za modeliranje na lokalnoj skali često koristimo funkcije kernela za opisivanje prostornih interakcija:

\begin{equation}
	K(d) = \exp\left(-\frac{d^2}{2\sigma^2}\right)
\end{equation}

gdje je $d$ udaljenost između lokacija, a $\sigma$ parametar koji kontrolira prostorni doseg interakcije.

\subsubsection{Krajobrazna skala (1-100 km)}

Krajobrazna skala obuhvaća mozaik različitih staništa i tipova korištenja zemljišta. Na ovoj skali proučavamo:

\begin{itemize}
	\item \textbf{Fragmentaciju staništa}
	\item \textbf{Koridore i ekološku povezanost}
	\item \textbf{Metapopulacijske dinamike}
\end{itemize}

\textbf{Metrike krajobrazne strukture:}

1. \textbf{Indeks fragmentacije}:
\begin{equation}
	\text{FI} = 1 - \frac{A_{\text{core}}}{A_{\text{total}}}
\end{equation}
gdje je $A_{\text{core}}$ površina osnovnih staništa, a $A_{\text{total}}$ ukupna površina staništa.

2. \textbf{Indeks povezanosti}:
\begin{equation}
	\text{PC} = \frac{\sum_i \sum_j (a_i \times a_j \times p^*_{ij})}{A^2_L}
\end{equation}
gdje su $a_i$ i $a_j$ površine fragmenata $i$ i $j$, $p^*_{ij}$ je vjerojatnost disperzije između fragmenata, a $A_L$ je ukupna površina krajolika.

\subsubsection{Regionalna skala (100-1000 km)}

Regionalna skala odgovara biogeografskim regijama i karakterizirana je:

\begin{itemize}
	\item \textbf{Klimatskim gradijentima}
	\item \textbf{Historijskim biogeografskim procesima}
	\item \textbf{Velikim ekološkim koridorima}
\end{itemize}

\subsubsection{Kontinentalna skala (>1000 km)}

Na kontinentalnoj skali dominiraju:

\begin{itemize}
	\item \textbf{Makroklimatski uvjeti}
	\item \textbf{Biogeografske barijere}
	\item \textbf{Evolucijski procesi}
\end{itemize}

\subsection{Prostorna autokorelacija}

Fundamentalni princip prostorne ekologije glasi da su lokacije koje su bliže jedna drugoj sličnije od onih koje su udaljenije (Toblerovo prvo pravilo geografije).

\textbf{Moranovo I:}

Globalna prostorna autokorelacija mjeri se Moranovim I indeksom:

\begin{equation}
	I = \frac{n}{S_0} \times \frac{\sum_i \sum_j w_{ij}(x_i - \bar{x})(x_j - \bar{x})}{\sum_i (x_i - \bar{x})^2}
\end{equation}

gdje su:
\begin{itemize}
	\item $n$ broj lokacija
	\item $w_{ij}$ prostorni težinski elementi
	\item $S_0 = \sum_i \sum_j w_{ij}$
	\item $x_i$ vrijednost varijable na lokaciji $i$
	\item $\bar{x}$ prosjek varijable
\end{itemize}

Vrijednosti $I$ kreću se od -1 (savršena negativna autokorelacija) do +1 (savršena pozitivna autokorelacija).

\section{Modeli rasprostiranja vrsta (SDM/ENM)}

Modeli rasprostiranja vrsta (Species Distribution Models - SDM) ili modeli ekoloških niša (Ecological Niche Models - ENM) predstavljaju jedan od najvažnijih alata u suvremenoj ekologiji i biologiji konzervacije.

\subsection{Konceptualni okvir}

\textbf{Fundamentalni vs. realizirani ekološki prostor:}

Hutchinsonov koncept ekološke niše razlikuje:

\begin{enumerate}
	\item \textbf{Fundamentalna niša} (n-dimenzionalni hipervolumen): svi uvjeti okoliša u kojima vrsta može preživjeti i razmnožavati se
	\item \textbf{Realizirana niša}: dio fundamentalne niše koje vrsta stvarno nastanjuje
\end{enumerate}

\textbf{Matematički zapis niše:}

Ako je okoliš opisan s $m$ varijabli $\mathbf{E} = (E_1, E_2, \ldots, E_m)$, onda je fundamentalna niša:

\begin{equation}
	N_f = \{\mathbf{E} \in \mathbb{R}^m \mid \text{fitness}(\mathbf{E}) > 0\}
\end{equation}

a realizirana niša:

\begin{equation}
	N_r = N_f \cap A \cap B
\end{equation}

gdje $A$ predstavlja geografski dostupan prostor, a $B$ biotičke interakcije.

\subsection{Klimatske varijable i bioklimatski slojevi}

\textbf{WorldClim bioklimatske varijable:}

Standardni skup od 19 bioklimatskih varijabli (BIO1-BIO19) karakterizira temperaturne i oborinske uvjete:

\textbf{Temperaturne varijable:}
\begin{itemize}
	\item BIO1: Godišnja srednja temperatura
	\item BIO2: Srednji dnevni raspon temperatura (mjesečni prosjek(max temp - min temp))
	\item BIO3: Izotermnost (BIO2/BIO7) × 100
	\item BIO4: Temperaturna sezonalnost (standardna devijacija × 100)
	\item BIO5: Maksimalna temperatura najtopilijeg mjeseca
	\item BIO6: Minimalna temperatura najhladnijeg mjeseca
	\item BIO7: Godišnji temperaturni raspon (BIO5-BIO6)
\end{itemize}

\textbf{Oborinske varijable:}
\begin{itemize}
	\item BIO12: Godišnji oborinski ukupak
	\item BIO13: Oborinski ukupak najkišnijeg mjeseca
	\item BIO14: Oborinski ukupak najsušljeg mjeseca
	\item BIO15: Oborinska sezonalnost (koeficijent varijacije)
\end{itemize}

\textbf{Kombinacijske varijable:}
\begin{itemize}
	\item BIO16: Oborinski ukupak najkišnije četvrtine
	\item BIO17: Oborinski ukupak najsušlje četvrtine
	\item BIO18: Oborinski ukupak najtopiije četvrtine
	\item BIO19: Oborinski ukupak najhladnije četvrtine
\end{itemize}

\textbf{Izračun sezonalnosti:}

Temperaturna sezonalnost:
\begin{equation}
	\text{BIO4} = \sigma(T_1, T_2, \ldots, T_{12}) \times 100
\end{equation}

Oborinska sezonalnost:
\begin{equation}
	\text{BIO15} = \frac{\sigma(P_1, P_2, \ldots, P_{12})}{\mu(P_1, P_2, \ldots, P_{12})} \times 100
\end{equation}

gdje su $T_i$ i $P_i$ mjesečne temperature i oborini.

\subsection{Algoritmi SDM modeliranja}

\subsubsection{Maximum Entropy (MaxEnt)}

MaxEnt je jedan od najšire korištenih algoritma za SDM. Temelji se na principu maksimalne entropije - pronalaženje distribucije koja je najpristranija prema dostupnim podacima.

\textbf{Matematički okvir MaxEnt:}

Za skup okolišnih varijabli $\mathbf{X} = (x_1, x_2, \ldots, x_m)$ i skup značajki $\mathbf{f} = (f_1, f_2, \ldots, f_k)$, MaxEnt traži distribuciju $q$ koja maksimizira entropiju:

\begin{equation}
	H(q) = -\int q(\mathbf{x}) \log q(\mathbf{x}) d\mathbf{x}
\end{equation}

uz ograničenja:
\begin{equation}
	E_q[f_j] = \hat{f}_j \quad (j = 1, \ldots, k)
\end{equation}

gdje je $\hat{f}_j$ empirijski prosjek značajke $f_j$ iz podataka o prisutnosti.

\textbf{Exponential family format:}

Rješenje MaxEnt problema ima oblik:

\begin{equation}
	q(\mathbf{x}) = \frac{\exp\left(\sum_j \lambda_j f_j(\mathbf{x})\right)}{Z}
\end{equation}

gdje su $\lambda_j$ Lagrangeovi multiplikatori, a $Z$ normalizacijska konstanta:

\begin{equation}
	Z = \int \exp\left(\sum_j \lambda_j f_j(\mathbf{x})\right) d\mathbf{x}
\end{equation}

\textbf{Logistički format:}

Za praktičnu primjenu, MaxEnt koristi logistički format:

\begin{equation}
	P(\mathbf{x}) = \frac{\exp\left(\sum_j \lambda_j f_j(\mathbf{x})\right)}{1 + \exp\left(\sum_j \lambda_j f_j(\mathbf{x})\right)}
\end{equation}

\subsubsection{Generalizirani linearni modeli (GLM)}

GLM pristup koristi logističku regresiju za modeliranje prisutnosti/odsutnosti:

\begin{equation}
	\text{logit}(p) = \beta_0 + \sum_i \beta_i x_i
\end{equation}

gdje je $p$ vjerojatnost prisutnosti vrste, a $x_i$ su okolišne varijable.

\textbf{Likelihood funkcija:}

Za $n$ lokacija s poznatim statusom prisutnosti $y_i \in \{0,1\}$:

\begin{equation}
	L(\boldsymbol{\beta}) = \prod_{i=1}^n p(\mathbf{x}_i)^{y_i} (1-p(\mathbf{x}_i))^{1-y_i}
\end{equation}

\subsubsection{Random Forest}

Random Forest kombinira mnoge stabla odlučivanja:

\begin{equation}
	\hat{P}(\mathbf{x}) = \frac{1}{B} \sum_{b=1}^B T_b(\mathbf{x})
\end{equation}

gdje je $B$ broj stabala, a $T_b(\mathbf{x})$ predviđanje $b$-tog stabla.

\subsection{Evaluacija modela}

\textbf{AUC (Area Under the ROC Curve):}

ROC krivulja prikazuje osjetljivost (True Positive Rate) u odnosu na 1-specifičnost (False Positive Rate):

\begin{align}
	\text{Sensitivity} &= \frac{\text{TP}}{\text{TP} + \text{FN}} \\
	\text{Specificity} &= \frac{\text{TN}}{\text{TN} + \text{FP}}
\end{align}

AUC vrijednosti:
\begin{itemize}
	\item 0.5: random model
	\item 0.7-0.8: dobar model
	\item 0.8-0.9: vrlo dobar model
	\item >0.9: izvrsno (možda overfitting)
\end{itemize}

\textbf{TSS (True Skill Statistic):}

\begin{equation}
	\text{TSS} = \text{Sensitivity} + \text{Specificity} - 1
\end{equation}

TSS kreće se od -1 do +1, gdje vrijednosti >0.4 označavaju dobre modele.

\subsection{Pristranost uzorkovanja}

Pristranost uzorkovanja predstavlja jedan od najvećih izazova u SDM modeliranju.

\textbf{Prostorna pristranost:}

Opažanja su često koncentrirana uz ceste, naselja ili dostupna područja. Ovo može dovesti do:

\begin{enumerate}
	\item \textbf{Geografske pristranosti}: područja nisu jednoliko uzorkovana
	\item \textbf{Okolišne pristranosti}: određeni tip staništa je predviđen zbog dostupnosti
\end{enumerate}

\textbf{Metode korekcije pristranosti:}

1. \textbf{Spatial filtering}: uklanjanje blisko lokaliziranih točaka
\begin{equation}
	\text{min\_distance} = \sigma \times \sqrt{\frac{A}{n}}
\end{equation}
gdje je $A$ ukupna površina studijskog područja, $n$ broj lokacija, $\sigma$ scaling faktor.

2. \textbf{Target-group background}: korištenje lokacija drugih vrsta kao pozadinske točke

3. \textbf{Kernel density estimation}: težinski background uzorčavanje prema distribuciji opažanja

\begin{equation}
	w(\mathbf{x}) = \frac{K(\|\mathbf{x} - \mathbf{x}_i\|/h)}{\sum_j K(\|\mathbf{x} - \mathbf{x}_j\|/h)}
\end{equation}

gdje je $K$ kernel funkcija (obično Gaussian), $h$ bandwidth parametar.

\section{Individualno temeljeni modeli (IBM)}

Individualno temeljeni modeli (Individual-Based Models - IBM) predstavljaju bottom-up pristup modeliranju gdje se ponašanje sustava izvodi iz interakcija između jedinki.

\subsection{Konceptualni okvir IBM-a}

\textbf{Osnovni principi:}

\begin{enumerate}
	\item \textbf{Diskretnost}: jedinke su diskretni entiteti s vlastitim svojstvima
	\item \textbf{Lokalnost}: interakcije se odvijaju lokalno u prostoru i vremenu
	\item \textbf{Emergentno}: svojstva populacije i zajednice nastaju iz interakcija među jedinkama
\end{enumerate}

\textbf{Struktura IBM modela:}

Svaka jedinka $i \in \{1, 2, \ldots, N\}$ karakterizirana je skupom atributa:
\begin{equation}
	I_i = (x_i, y_i, a_i, s_i, e_i, \ldots)
\end{equation}

gdje su:
\begin{itemize}
	\item $(x_i, y_i)$: prostorna lokacija
	\item $a_i$: dob
	\item $s_i$: spol
	\item $e_i$: energetsko stanje
\end{itemize}

\subsection{Prostorno ponašanje i movement}

\textbf{Random walk modeli:}

Jednostavan random walk u diskretnom vremenu:
\begin{align}
	x_i(t+1) &= x_i(t) + \Delta x \\
	y_i(t+1) &= y_i(t) + \Delta y
\end{align}

gdje su $\Delta x$ i $\Delta y$ slučajni pomaci iz određene distribucije.

\textbf{Correlated random walk:}

\begin{equation}
	\theta_i(t+1) = \theta_i(t) + \delta
\end{equation}

gdje je $\theta$ smjer kretanja, a $\delta \sim N(0, \sigma^2)$ slučajna promjena smjera.

\textbf{Biased random walk:}

Uključuje preferenciju prema određenim staništima:
\begin{equation}
	P(\text{move to cell } j) \propto \exp(\beta S_j)
\end{equation}

gdje je $S_j$ prikladnost staništa u ćeliji $j$.

\subsection{Energetski modeli}

\textbf{Simple energy budget:}

\begin{equation}
	\frac{dE}{dt} = I - M - G
\end{equation}

gdje je:
\begin{itemize}
	\item $E$: energetski sadržaj jedinke
	\item $I$: stopa unosa energije (hranjenje)
	\item $M$: stopa metabolizma
	\item $G$: stopa rasta/reprodukcije
\end{itemize}

\textbf{Foraging behaviour:}

Optimalno hranjenje prema Marginal Value Theorem:
\begin{equation}
	\frac{\partial G}{\partial t} = \lambda
\end{equation}

gdje je $\lambda$ prosječna stopa dobivanja energije u staništu.

\textbf{Functional response:}

Type II functional response za individual foraging:
\begin{equation}
	I = \frac{a \times R}{1 + a \times h \times R}
\end{equation}

gdje je:
\begin{itemize}
	\item $a$: attacking rate
	\item $R$: resource density
	\item $h$: handling time
\end{itemize}

\subsection{Reprodukcija i demografija}

\textbf{Probabilistic reproduction:}

Vjerojatnost reprodukcije ovisi o energetskom stanju:
\begin{equation}
	P(\text{reproduction}) = \frac{1}{1 + \exp(-\beta(E - E_{\text{threshold}}))}
\end{equation}

\textbf{Offspring numbers:}

Broj potomaka slijedi Poisson distribuciju:
\begin{equation}
	N_{\text{offspring}} \sim \text{Poisson}(\lambda_{\text{reproduction}} \times \text{fecundity})
\end{equation}

\subsection{Interakcije između jedinki}

\textbf{Competition for resources:}

Lokalna konkurencija modelirana kernel funkcijom:
\begin{equation}
	\text{Competition\_effect}_i = \sum_{j \neq i} K(d_{ij}) \times \text{resource\_overlap}
\end{equation}

gdje je $K(d) = \exp(-d^2/2\sigma^2)$ competition kernel.

\textbf{Predator-prey interactions:}

Vjerojatnost predacije:
\begin{equation}
	P(\text{predation}) = 1 - \exp(-\lambda \times \text{predator\_density} \times \text{exposure\_time})
\end{equation}

\subsection{Emergentna svojstva}

IBM modeli mogu proizvesti kompleksna emergentna ponašanja:

\textbf{Spatial aggregation:}

Koeficijent agregacije:
\begin{equation}
	IA = \frac{\sigma^2/\mu - 1}{n-1}
\end{equation}

gdje je $\sigma^2$ varijanca, $\mu$ prosjek gustoće, $n$ broj ćelija.

\textbf{Population cycles:}

Analizom vremenskih serija ukupne abundancije:
\begin{equation}
	N(t) = A + B \cos\left(\frac{2\pi t}{T} + \phi\right)
\end{equation}

gdje je $T$ period ciklusa.

\textbf{Pattern formation:}

Prostorni uzorci analizirani prostornim autokorelacijskim funkcijama:
\begin{equation}
	r(d) = \frac{\sum_i \sum_j (n_i - \bar{n})(n_j - \bar{n})}{\sigma^2 \times \text{number\_of\_pairs}}
\end{equation}

\subsection{Implementacija i softverski alati}

\textbf{NetLogo pseudokod:}

\begin{verbatim}
	to setup
	create-turtles initial-population [
	setxy random-xcor random-ycor
	set energy random 100
	set age 0
	]
	end
	
	to go
	ask turtles [
	move
	feed
	reproduce
	age
	if energy < 0 [ die ]
	]
	tick
	end
	
	to move
	rt random 360
	fd step-size
	end
\end{verbatim}

\textbf{Kalibracija i validacija:}

IBM modeli kalibriraju se pomoću:

\begin{enumerate}
	\item \textbf{Pattern-oriented modelling (POM)}: fitiranje multiplih prostornih i temporalnih uzoraka
	\item \textbf{Approximate Bayesian Computation (ABC)}:
	\begin{equation}
		d(S_{\text{obs}}, S_{\text{sim}}) < \varepsilon
	\end{equation}
	gdje je $d$ distance funkcija između opaženih ($S_{\text{obs}}$) i simuliranih ($S_{\text{sim}}$) uzoraka.
\end{enumerate}

IBM modeli predstavljaju moćan alat za razumijevanje kompleksnih ekoloških sustava, omogućujući nam da istražimo kako lokalne interakcije generiraju globalne obrasce u prostoru i vremenu.
	
	\chapter{Prediktivno modeliranje i scenariji}
	\section{Scenariji klimatskih promjena}
	Korištenje scenarija (npr. SSP/RCP) u ekološkim projekcijama.
	\section{Predviđanje invazija i rani sustavi upozorenja}
	Integracija podataka nadzora i modela rizika; pragovi upozorenja.
	\section{Spajanje mehanističkih i statističkih pristupa}
	Hibridni modeli i kalibracija.
	
	\chapter{Statistički i AI pristupi}
	\section{Klasična i Bayesovska statistika}
	GLM/GLMM, MCMC, kredibilni intervali.
	\section{Strojno učenje}
	RF, XGBoost, SVM; značajke, regularizacija, unakrsna provjera.
	\section{Duboko učenje i digitalni blizanci ekosustava}
	Grafovi, konvolucijske i sekvencijske mreže; digital twins za \emph{what-if} scenarije.
	
	% =====================================================
	\part{Validacija, nesigurnost i primjene}
	% =====================================================
	
	\chapter{Validacija, verifikacija i osjetljivost}
	\section{Kalibracija i verifikacija}
	Podjela podataka, \emph{out-of-sample} procjene, nezavisni skupovi.
	\section{Analize osjetljivosti i robustnosti}
	Lokalne i globalne metode; Monte Carlo; propagacija nesigurnosti.
	\section{Etika i komunikacija nesigurnosti}
	Transparentnost i odgovornost modelara.
	
	\chapter{Primjene u upravljanju okolišem}
	\section{Zaštita prirode i bioraznolikosti}
	Prioritizacija očuvanja, planovi upravljanja.
	\section{Poljoprivreda i šumarstvo}
	Produktivnost, štetnici, otpornost agroekosustava.
	\section{Ekotoksikologija i procjena rizika}
	LC/EC metri\linebreak ke, PNEC, scenariji izloženosti; integracija s populacijskim modelima.
	\section{Klimatske politike i održivi razvoj}
	Podrška odlučivanju, socio-ekonomske poveznice.
	
	% =====================================================
	\part{Studije slučaja i praktične vježbe}
	% =====================================================
	
	\chapter{Studije slučaja}
	\section{Gujavice u agroekosustavima}
	Dobno strukturirani modeli, elastičnost i scenariji upravljanja.
	\section{Širenje komaraca}
	SDM + meteorološki pogonjeni prediktori; rani sustavi upozorenja.
	\section{Eutrofikacija riječnog ekosustava}
	Kutije (box) modeli i validacija na nizvodnim mjerenjima.
	\section{Fragmentacija šuma i ptice metapopulacija}
	Povezanost staništa i pragovi propusnosti krajolika.
	\section{Bayesove mreže u ekotoksikologiji}
	Kauzalni grafikoni i inverzna inferencija.
	
	\chapter{Praktične vježbe (\R{} i \Python{})}
	\section{Uvod u \R{} i \Python{} alate}
	Instalacija, radna okolina, reproducibilnost.
	\section{Leslie matrica u \R{}}
	\begin{lstlisting}[language=R,caption={Leslie matrica i projekcija populacije u R-u}]
		F <- c(0, 0.3, 1.2, 1.5)
		P <- c(0.6, 0.7, 0.8)
		L <- matrix(c(F,
		P[1],0,0,0,
		0,P[2],0,0,
		0,0,P[3],0), nrow=4, byrow=TRUE)
		n0 <- c(50, 40, 20, 10)
		proj <- function(L, n, t=20){
			N <- matrix(NA, nrow=length(n), ncol=t+1)
			N[,1] <- n
			for(i in 1:t) N[,i+1] <- L %*% N[,i]
			N
		}
		N <- proj(L, n0, t=30)
		colSums(N) -> Tot
		plot(Tot, type="l", xlab="Vrijeme", ylab="Uk. veličina")
	\end{lstlisting}
	
	\section{Lotka--Volterra simulacija u \Python{}}
	\begin{lstlisting}[language=Python,caption={Jednostavni LV model u Pythonu (scipy.integrate)}]
		import numpy as np
		from scipy.integrate import solve_ivp
		import matplotlib.pyplot as plt
		
		def lv(t, z, r, a, e, m):
		N, P = z
		dN = r*N - a*N*P
		dP = e*a*N*P - m*P
		return [dN, dP]
		
		pars = dict(r=0.8, a=0.02, e=0.1, m=0.3)
		sol = solve_ivp(lambda t,z: lv(t,z,**pars),
		[0, 200], [40, 9], dense_output=True)
		t = np.linspace(0,200,1000)
		N, P = sol.sol(t)
		plt.plot(t, N, label="Plijen")
		plt.plot(t, P, label="Grabežljivac")
		plt.xlabel("Vrijeme"); plt.ylabel("Gustoća"); plt.legend(); plt.show()
	\end{lstlisting}
	
	\section{SDM u \R{} (skica s \texttt{dismo})}
	\begin{lstlisting}[language=R,caption={SDM skica s bioklimatskim varijablama}]
		library(dismo); library(raster)
		# bioclim <- getData('worldclim', var='bio', res=10) # primjer dohvaćanja
		# occ <- read.csv("occ_points.csv")  # popisi opažanja (lon, lat)
		# m <- maxent(bioclim, occ)
		# p <- predict(bioclim, m)
		# plot(p)
	\end{lstlisting}
	
	\section{Random Forest za invazije}
	\begin{lstlisting}[language=R,caption={RF klasifikator s unakrsnom provjerom}]
		library(tidymodels)
		set.seed(1)
		# df: response ~ predictors
		split <- initial_split(df, prop=0.8, strata=response)
		train <- training(split); test <- testing(split)
		rf_spec <- rand_forest(trees=500) %>% set_engine("ranger") %>% set_mode("classification")
		rec <- recipe(response ~ ., data=train) %>% step_zv(all_predictors())
		wf <- workflow() %>% add_model(rf_spec) %>% add_recipe(rec)
		res <- wf %>% fit_resamples(vfold_cv(train, v=5, strata=response), metrics=metric_set(roc_auc,accuracy))
		collect_metrics(res)
	\end{lstlisting}
	
	\section{DEB simulacije (skica)}
	\begin{lstlisting}[language=R,caption={Minimalna DEB skica (konceptualno)}]
		# Ovo je conceptual stub: definirajte parametre i tokove E, V...
		pars <- list(p_Am=1, v=0.02, kappa=0.8)
		state <- c(E=1, V=0.1)
		deb_model <- function(t, y, p){
			with(as.list(c(y,p)), {
				dE <- p_Am - v*E
				dV <- kappa*v*E - 0.01*V
				list(c(dE,dV))
			})
		}
		# solve s deSolve::ode(...)
	\end{lstlisting}
	
	% =====================================================
	\appendix
	% =====================================================
	
	\chapter{Matematički prilozi}
	\section{Osnovni populacijski modeli}
	Sažeti popis formula (eksponencijalni, logistički, Ricker, Beverton--Holt).
	
	\chapter{Instalacija softvera}
	Upute za \R/\Python{} okruženja, pakete i reproducibilnost.
	
	\chapter{Primjeri koda}
	Proširene skripte za poglavlja iz \emph{Praktičnih vježbi}.
	
	\chapter{Rječnik pojmova}
	Bioraznolikost, nosivost staništa, metapopulacija, ekološka niša, stohastičnost.
	
	\backmatter
	\printbibliography
	
\end{document}
